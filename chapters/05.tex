\documentclass[output=collectionpaper]{langsci/langscibook}

\title{Niger-Congo ``noun classes'' conflate gender with deriflection}
\author{Tom Güldemann\affiliation{Humboldt University Berlin and Max Planck Institute for the Science of Human History}\lastand Ines Fiedler\affiliation{Humboldt University Berlin}}


\abstract{
This paper reviews the treatment of gender systems in Niger-Congo languages. Our discussion is based on a consistent methodological approach, to be presented in \sectref{sec:Gueld:1}, which employs four analytical concepts, namely agreement class, gender, nominal form class, and deriflection and which, as we argue, are applicable within Niger-Congo and beyond. Due to the strong bias toward the reconstruction of Bantu and wider Benue-Congo, Niger-Congo gender systems tend to be analyzed by means of a philologically biased and partly inadequate approach that is outlined in \sectref{sec:Gueld:2}. This framework assumes in particular a consistent alliterative one-to-one mapping of agreement and nominal form classes conflated under the philological concept of ``noun class''. One result of this is that gender systems are recurrently deduced merely from the number-mapping of nominal form classes in the nominal deriflection system rather than from the agreement behavior of noun lexemes. We show, however, that gender and deriflection systems are in principle different, illustrating this in \sectref{sec:Gueld:3} with data from such \ili{Niger-Congo} subgroups as \ili{Potou-Akanic} and \ili{Ghana-Togo-Mountain}. Our conclusions given in \sectref{sec:Gueld:4} are not only relevant for the historical-comparative and typological assessment of Niger-Congo systems but also for the general approach to grammatical gender.
\medskip

\textbf{Keywords:}
  gender,
  Niger-Congo languages,
  agreement,
  noun classes,
  deriflection
}%

\maketitle
\begin{document}

\section{The cross-linguistic approach to gender}
\label{sec:Gueld:1}

Gender is understood here in terms of \citet{Corbett1991}, namely as systems of nominal classification (also called categorization) that are reflected by agreement. ``With about two thirds of all African languages [being] gender languages'' \citep[190]{Heine1982}, Africa is rightly identified by \citet[131]{Nichols1992} as a global hotbed of this phenomenon. At the same time, the majority of African languages belong to a single language family, \ili{Niger-Congo},%
\footnote{We will not deal here with the still controversial question of the exact composition of this language family. That there is a substantial core group of genealogically related languages has been shown by \citet{Westermann1935} with reference to gender, the very feature at issue, and the present discussion is concerned with languages that are robust members of this lineage (see Güldemann (\citeyear{Gueldemann2018}) for a detailed recent discussion of the genealogical classification of African languages and the status of \ili{Niger-Congo} in particular). While the discussion is also relevant for uncertain members of the group, we will not deal with them here.%
} %
which displays a cross-linguistically unusual type of nominal classification described in a particular philological tradition. The existing research bias toward this large family keeps influencing the treatment of noun classification not only in African linguistics but also in typology in general. This contribution approaches the typical gender systems of \ili{Niger-Congo} from a cross-linguistic perspective by subjecting them to an analysis that is universally applicable rather than one that is biased toward the special characteristics of this language group.

As mentioned above, according to the typologically most widespread approach, gender is the intersection of two domains, namely nominal classification and syntactic agreement, as the overt expression of a feature of a ``trigger'' (also called controller), usually a noun, on another word as the ``target''. Several complications for the analysis of gender arise from Corbett's (\citeyear{Corbett2006}) extensive cross-linguistic survey of agreement. Notably, a language may have more than one agreement system and, more importantly for our discussion, a system sensitive to gender need not be restricted to this feature but most often also concerns others like number, person, case, etc. The features that a noun trigger transfers to a target not only relate to properties of an abstract lexical item, which are recurrently semantic. They can also concern the formal properties of the concrete word form of a given noun in the agreement context. A sound understanding of a gender system thus presupposes an exhaustive analysis of the language's agreement system regarding all its agreement features and the subsequent ``subtraction'' of all factors but gender. If gender is only conflated with number, which is cross-linguistically frequent, it can be conceptualized as ``agreement minus number.'' This also holds for the \ili{Niger-Congo} systems at issue here.

The present contribution provides a novel analytical approach to gender. That is, we apply a strict distinction of four concepts, which are necessary whenever gender is reflected by syntactic agreement as well as nominal morpho-phonology, the latter implying some amount of what \citet{Corbett1991} calls formal class assignment. The four notions are:%
\footnote{Since genders and deriflections also establish sets of nouns, they could also be called ``gender \spterm{classes}'' and ``deriflection \spterm{classes}'', respectively. We use here the short versions.}

\begin{enumerate}
\item[a.] \spterm{agreement class} (to be abbreviated as AGR and numbered by Arabic numbers),

\item[b.] \spterm{gender} (to be occasionally labeled semantically or numbered by Roman numbers),

\item[c.]  \spterm{nominal form class} (to be abbreviated as NF and represented by the capitalized exponent), and

\item[d.] \spterm{deriflection} (see p.~\pageref{page:Gueld:def} for the definition of the term, to be represented by the relevant NF set).
\end{enumerate}

This approach is illustrated with the following example from the \ili{Bantu} language \ili{Swahili}, where agreement and nominal form classes are bold-faced in both vernacular and annotation line.

\ea%1
    \label{ex:Gueld:1}
\langinfo{Swahili}{}{personal knowledge}\\
\begin{xlist}
\ex
\gll \textbf{m}{}-toto  \textbf{yu}{}-le  \textbf{m}{}-moja  \textbf{a}{}-me-anguka\\
     \textbf{\textsc{m(w)}}{}-child(\textbf{1})  \textbf{1}{}-\textsc{d.dem}  \textbf{1}{}-one  \textbf{1}{}-\textsc{perf}{}-fall\\
\glt `that one child has fallen'
\ex
\gll \textbf{wa}{}-toto  \textbf{wa}{}-le  \textbf{wa}{}-wili  \textbf{wa}{}-me-anguka\\
     \textbf{\textsc{w(a)}}{}-child(\textbf{2})  \textbf{2}{}-\textsc{d.dem}  \textbf{2}{}-two  \textbf{2}{}-\textsc{perf}{}-fall\\
\glt `those two children have fallen'
\end{xlist}
\z

The subject nouns in (\ref{ex:Gueld:1}) trigger agreement on three targets: the demonstrative modifier \textit{{}-le}, the numeral modifiers \textit{{}-moja} and \textit{{}-wili}, and the verb \textit{{}-anguka} in the form of subject cross-reference. There are two different \spterm{agreement classes}, AGR1 and AGR2, that are associated with the noun forms \textit{m.toto} `child (SG)' in (\ref{ex:Gueld:1}a) and \textit{wa.toto} `children (PL)' in (\ref{ex:Gueld:1}b), respectively, and they are evident from two different sets of exponents across the three relevant agreement targets, namely \textit{yu-/m-/a-} vs.\ \textit{wa-/wa-/wa-}. An agreement class in the present conceptualization is thus a set of noun forms that share an identical behavior across all agreement contexts of a given system and thus equals what Corbett (\citeyear{Corbett1991}, \citeyear{Corbett2006}) calls a ``consistent agreement pattern'' (see this author's detailed discussion of the possible problems in establishing such an agreement class). (For schematic presentation, an agreement class is represented conventionally by the set of exponents of a single agreement target that involves the maximal class differentiation.) A crucial feature of our approach is that it is of no concern whether noun forms of one agreement class are of the same gender, number or any other feature, which differs from Corbett's approach inspired by \citet{Zaliznjak1964}. An agreement class in the present terms is thus an overt but normally conflated reflex of diverse grammatical features \textendash{} in \ili{Swahili}, concretely of gender and number (see below for more details about our analytical and terminological differences to Corbett's approach).

\spterm{Gender (classes)} are defined in line with Corbett's (\citeyear{Corbett1991}) cross-linguistic approach. Analytically, they are derived by abstracting from all other agreement features, which in the \ili{Swahili} system is only number. The majority of \ili{Swahili} nouns have a singular and a plural form so that a gender is instantiated by a particular pairing of the respective agreement classes. In (\ref{ex:Gueld:1}), these are singular AGR1 and plural AGR2, which is the regular agreement behavior for count nouns of the ``human'' gender, which includes the nominal lexeme \textit{{}-toto} `child'. The gender of transnumeral%
\footnote{The term ``transnumeral'' is used here neutrally to refer to nouns that do not partake in the normal number oppositions of a language. It must not be confused with ``general number'' in terms of \citet[9--19]{Corbett2000}, which refers to a feature value in the number system as opposed to the more common singular and plural. Typically, transnumeral nouns like infinitives, locatives and non-count nouns for masses, liquids, abstracts etc. do not have different number forms, while general number is a number value that applies to nouns that have an alternative singular and/or plural variant.%
} %
nouns outside the systems of number distinctions is accordingly discernible from a single agreement class.%
\footnote{In general, any agreement class that only encodes gender and no other agreement feature does not require a distinction between gender and agreement class. An entire system of this kind would represent ``ideal'' functionally transparent gender marking, because there is a straightforward relation between one form and one meaning. However, such cases turn out to be rare cross-linguistically; they are found, for example, in Australian languages.} %
Normally, genders as the ultimate goal of analysis here are thus classes of nouns in the lexicon. However, gender often transcends the lexicon and applies to a language's reference world more generally. That is, relevant systems can entail in addition such phenomena as nominal derivation and even the expression of grammatical relations. \ili{Swahili}, for instance, also has agreement patterns (and noun prefixes) for derivational diminutives, infinitives, and various locative notions. The nominal lexeme \textit{{}-toto} `child', for example, can also occur in the gender AGR7/AGR8 for diminutives, then appearing accordingly as \textit{ki-toto/vi-toto} `baby/babies'.

Example (\ref{ex:Gueld:1}) also shows the intimate interaction between nominal morphology and gender in \ili{Swahili}. The subject nouns as the agreement triggers again exhibit two morphologically distinct word forms rendered by prefixes, namely \textit{m-} and \textit{wa-}, which characterize NF \textit{M(W)-} and NF \textit{W(A)-}, respectively. This direct morphological reflex of gender on the noun is conventionally subsumed under ``overt gender'' (cf.\ \citealt[44, 62--63, 117--118]{Corbett1991}). That is, \spterm{nominal form classes} are established in the present approach by word forms with identical morphological or phonological properties; they represent the counterpart of agreement classes in the realm of morpho(phono)logy. As shown in the important work by \citet{Evans1997} and \citet{Evans1998}, nominal form classes (called there ``head classes'') can have an intricate relationship to agreement classes well beyond serving potentially as their triggers.

What is called here \spterm{deriflection (classes)} is the morpho(phono)logical counterpart of genders.
\label{page:Gueld:def}%
They are classes of form paradigms operating over nominal lexemes and established on account of identical formal variation that does not need but often does interact with such features as gender, number, etc. Our newly coined term ``deriflection'' (a blend of ``inflection'' and ``derivation'') thus refers here in a more narrow sense to relevant morphology or phonology that interacts with gender. In (\ref{ex:Gueld:1}) of \ili{Swahili} the two prefixes on \textit{{}-toto} `child' establish a specific type of number inflection typical for human nouns, namely \textit{M(W)-}/\textit{W(A)-}, which is the pairing of a singular and a plural nominal form class exponent. As with genders, deriflections in this context also entail other morpho(phono)logical phenomena to the extent these interact with the relevant nominal system.

\begin{table}[!htb]
\begin{tabularx}{\textwidth}{lQQ}
\lsptoprule

Relates to & Concrete noun in a morpho-syntactic context = word form & Abstract noun in the lexicon = lexeme\\
\midrule
Syntax & a. AGREEMENT CLASS & b. GENDER \\
 & (abbreviated as AGR and numbered by Arabic numbers) & (numbered by Roman numbers)\\
Morpho(phono)logy & c. NOMINAL FORM CLASS & d. DERIFLECTION\\
 & (abbreviated as NF) \\
\lspbottomrule
\end{tabularx}
%
\caption{The four concepts used for analyzing gender}
\label{tab:Gueld:1}
\end{table}

In general, agreement class and nominal form class are concepts that relate to a noun as a word form in a concrete morphosyntactic context, while gender and deriflection refer primarily to the more abstract domain of the nominal lexicon in a given language. At the same time, agreement class and gender are both syntactically defined phenomena and thus opposed to nominal form class and deriflection pertaining to the domain of morpho(phono)logy, so that the two concept pairs, although intimately related, are in principle independent from each other. The various interrelations between the four concepts are summarized in \tabref{tab:Gueld:1}, which also repeats the different notation principles applied for them here.


Corbett's (\citeyear{Corbett1991,Corbett2000,Corbett2006}) work has served as the primary reference point for the previous typological analyses of gender and related phenomena. As is to be discussed shortly, however, our framework also departs in some important respects from this author in order to better capture aspects that have subsequently emerged regarding the cross-linguistic diversity in this domain.

The framework outlined here draws on \citet{Gueldemann2000}, which dealt with gender systems in Southern African languages of the two non-\ili{Khoe} families \ili{Tuu} and \ili{Kx'a} (both traditionally attributed to a spurious \ili{Khoisan} lineage). The most important typological contribution of this work is that agreement classes in these languages are often multiply ambiguous regarding their gender and number value, unlike in many European languages, whose analysis has set the stage for the cross-linguistic research on gender and agreement.


\begin{figure}[!htb]
\centering
\begin{tabular}{llp{\llen}l}
\lsptoprule
AGR & SG & & \tknode{0} PL \\
\midrule
\padding
3 &\itshape ká \tknode{A1} & \tknode{A2} &\itshape \tknode{A3} ká\\
\padding
4 &\itshape hì \hfill \tknode{B1} & \tknode{B2} &\itshape \tknode{B3} hì\\
\padding
1 & \itshape ha \tknode{C1} & \tknode{C2} & \itshape \tknode{C3} ha\\
\padding
2 &  & \tknode{D2} & \itshape \tknode{D3} sì \\
\lspbottomrule
\end{tabular}

% lines connecting the nodes and labels
\tikz[remember picture, overlay] \draw[thick] (A1.center) -- (A2.center) node[anchor=south] {V} -- (A3.center);
\tikz[remember picture, overlay] \draw[thick] (B1.center) -- (B2.center) node[anchor=south] {IV} -- (B3.center);
\tikz[remember picture, overlay] \draw[thick] (C1.center) -- (B3.center) %
node[near start,anchor=south] {II};
\tikz[remember picture, overlay] \draw[thick] (C1.center) -- (C3.center) %
node[very near end,anchor=south] {III};
\tikz[remember picture, overlay] \draw[thick] (C1.center) -- (D3.center) %
node[very near end,anchor=south] {I};

{\small Note: agreement classes represented by anaphoric pronouns.}

\caption{Agreement classes and genders in Juǀ'hoan (based on \citealt{Gueldemann2000})}
\label{fig:Gueld:1}
\end{figure}


This can be observed in \figref{fig:Gueld:1}, which displays the gender system of the Juǀ’hoan dialect of \ili{Ju}, a member of the \ili{Kx'a} family. The schema shows how the four agreement classes 1--4 pattern across the two number categories singular (SG) and plural (PL) to yield five genders I--V. The numbering of classes and genders as well as their ordering in the schema are of no concern to the system: the former is an artifact of research history and the latter merely serves to yield a maximally simple representation of the system. The reader is referred to \citet{Gueldemann2000} for more details, for example, on the semantics of the genders. The only important point for the present discussion is the behavior of the agreement classes, for example, that AGR1 occurs in both number values, singular and plural, as well as in the three genders I-III. The non-sensitivity of an agreement class to number holds in \ili{Juǀ’hoan} for AGR1, AGR3, and AGR4. The majority of nouns falling into these classes are not transnumeral but possess different singular and plural forms. Recall from above that a system where the gender marking of nouns only involves one agreement class is as such functionally transparent (albeit typologically rare) in that agreement is here a ``non-conflated'' \emph{direct} reflex of gender.

The phenomenon that agreement classes are not dedicated to a single gender and/or number is also recurrent outside these Southern African languages, including \ili{Niger-Congo}. This justifies the strict descriptive and analytical separation of agreement class from any particular value of gender, number etc. This is opposed to Corbett's (\citeyear{Corbett1991}) approach, which, moreover, features more analytical concepts than our framework. He distinguishes on the one hand between ``controller gender'' and ``target gender'' (see his Section 6.3) and on the other hand between ``agreement class'' and ``consistent agreement pattern'' (see his Sections 6.2 and 6.4.5). Our approach, as we argue here, does not need all these notions, because it captures the same data by ascertaining just agreement class (= Corbett's ``consistent agreement pattern'') and gender (= Corbett's ``controller gender'') (our two additional concepts, nominal form class and deriflection, are irrelevant here, because they concern the form of nouns rather than agreement and gender).


\begin{figure}[htb]
\centering
\begin{tabular}{lrp{\llen}l}
\lsptoprule
AGR & SG \hspace{-.6em} \tknode{0} & & \tknode{0} PL \\
\midrule
\padding
1 & \itshape -\O \tknode{2A1} &&\\
\padding
2 & & &\itshape \tknode{2B3} -i \\
\padding
3 & \itshape -\u{a} \tknode{2C1} && \\
\padding
4 &  & & \itshape \tknode{2D3} -e \\
\lspbottomrule
\end{tabular}

% lines connecting the nodes and labels
\tikz[remember picture, overlay] \draw[thick] (2A1.center) -- (2B3.center) node[midway,anchor=south] {M};
\tikz[remember picture, overlay] \draw[thick] (2A1.center) -- (2D3.center) node[midway,anchor=west] {N};
\tikz[remember picture, overlay] \draw[thick] (2C1.center) -- (2D3.center) %
node[midway,anchor=north] {F};

\caption{Agreement of adjectives and genders in Romanian (based on \citealt[152]{Corbett1991})}
\label{fig:Gueld:2}
\end{figure}

\figref{fig:Gueld:2} takes up Corbett's (\citeyear[150--152]{Corbett1991}) example of \ili{Romanian} adjective agreement, which he uses to illustrate the necessity of his target gender notion. He states about this case that there are ``three agreement classes, and there is no reason not to recognize each as a gender [= the lines labeled semantically as masculine, neuter, and feminine]''%
\footnote{Although Corbett's identification of agreement class and gender is surprising, a detailed critical discussion would require a general assessment of his approach, which is beyond the purpose and limits of this paper.%
} %
as well as ``two target genders in both singular and plural ... [\textit{{}-}\textit{Ø}, \textit{{}-}\textit{ă} and \textit{{}-i}, \textit{{}-e}]''. Corbett's fourth concept, consistent agreement pattern, which we would call agreement class, is not dealt with in his discussion that concerns the exponents of only one agreement context; the notion is, however, relevant for a full description, because \ili{Romanian} has more than one agreement target (see \citealt[213--214]{Corbett1991} for further complications in \ili{Romanian} neuter agreement forms). In any case, Corbett's problem is that two of the four gender-number markers on adjectives are not dedicated to a single gender, \textit{{}-}\textit{Ø} encoding the singular of both masculine and neuter gender and \textit{{}-e} marking the plural of both neuter and feminine gender; the target gender concept seems to be invoked to solve this problem. However, applying the framework proposed here to the situation in \ili{Romanian}, we only need to recognize three genders and four agreement classes (representing them here by the four suffixal exponents on adjectives but assuming that other agreement targets do not contradict this picture).

A picture like \figref{fig:Gueld:2} is nothing special and even in a more extreme case, such as \ili{Juǀ’hoan} in \figref{fig:Gueld:1}, it does not require more elaborate analytical machinery. In the \ili{Juǀ’hoan} system, comprising five genders across two number values, \emph{three of four} agreement classes are unspecific regarding \emph{both} gender and/or number. As far as we can see, an additional concept like target gender restricted to a specific number category does not furnish any new and useful insight for the description of this and other gender systems. Since the present approach has also been applied with coherent results to a number of other languages with quite different and notoriously intricate gender systems (cf., e.g., \citealt{Neuhaus2008} on \ili{Krongo} of the \ili{Kadu} family, \citealt{Gueldemann2015} on \ili{Somali} of the \ili{Cushitic} family), we assume its wider applicability. The rest of the paper attempts to show its usefulness for the languages of \ili{Niger-Congo}, the world's largest language family featuring a historically deeply entrenched gender system.

\section{Niger-Congo gender systems and the philological ``noun class'' concept}
\label{sec:Gueld:2}

While the noun classification systems in \ili{Niger-Congo} have long been recognized as instances of grammatical gender, their special structural profile poses particular challenges to a cross-linguistically oriented analysis. To a large extent, this is due to the special morphological characteristics of gender systems in \ili{Bantu}, the resulting philological tradition of analyzing them, and the considerable research bias within \ili{Niger-Congo} studies toward this important subgroup (see Güldemann (\citeyear{Gueldemann2018}, Chapter 5) for more discussion).

The situation presented in \sectref{sec:Gueld:1} above with example (\ref{ex:Gueld:1}) from \ili{Swahili} is quite typical in \ili{Bantu} and many other \ili{Niger-Congo} languages and thus has crucially determined the philological tradition of describing their gender systems as a whole. In particular, it shows a one-to-one relationship between corresponding agreement classes and nominal form classes. As seen in (\ref{ex:Gueld:1}b), even the markers can be formally identical: \textit{wa-} (or an allomorph) is the formal exponent in both NF \textit{W(A)-} and all agreement contexts of AGR2. Such a biunique (and often even alliterative) relation between the form of the noun (representing the trigger) and any agreeing element (representing the target) is epitomized by the philological concept of ``noun class''. The notion of ``noun class'' is also behind the philological convention of a single class label by means of Arabic numbers, in opposition to our proposed distinction between agreement class and nominal form class (accordingly, in (\ref{ex:Gueld:1}) and subsequent \ili{Swahili} examples, the nominal form classes are not glossed by Arabic numbers, even in cases of biuniqueness and alliteration).

The conflation of nominal form classes and agreement classes is, as we argue, the reason for a major problem in the analysis of \ili{Niger-Congo} gender systems. The conceptually overloaded concept of ``noun class'' may account in many languages for a good portion of the relevant nominal domain, to the extent that the situation is as in (\ref{ex:Gueld:1}) of \ili{Swahili}. However, the concept cannot completely and adequately capture an entire system, because the characteristics implied in it are not universal. Example (\ref{ex:Gueld:1}a) with NF \textit{M(W)-} and AGR1 involving \textit{yu-/m-/a-} as its set of exponents has already shown alliteration not to be absolute. More importantly, however, the implied one-to-one relation between agreement classes and nominal form classes also has crucial exceptions so that one type of class is not always predictable from the other, which is shown in the following representative examples.

\protectedex{%
\ea%2
    \label{ex:Gueld:2}
\langinfo{Swahili}{}{personal knowledge}
\begin{xlist}
\ex
\gll rafiki  \textbf{yu}{}-le  \textbf{m}{}-moja  \textbf{a}{}-me-anguka\\
     \textbf{\textsc{Ø:}}friend(\textbf{1})  \textbf{1}{}-\textsc{d.dem}  \textbf{1}{}-one  \textbf{1}{}-\textsc{perf}{}-fall\\
\glt `that one friend has fallen'
\ex
\gll \textbf{ma}{}-rafiki  \textbf{wa}{}-le  \textbf{wa}{}-wili  \textbf{wa}{}-me-anguka\\
     \textbf{\textsc{ma}}{}-friend(\textbf{2})  \textbf{2}{}-\textsc{d.dem}  \textbf{2}{}-two  \textbf{2}{}-\textsc{perf}{}-fall\\
\glt `those two friends have fallen'
\end{xlist}
\z
}%

Example (\ref{ex:Gueld:2}) shows that \ili{Swahili} nouns of the human gender, as defined by the pairing AGR1/AGR2, can also appear with other number inflections, here \textit{Ø/MA} with \textit{rafiki} `friend' (see below for more discussion on prefixless nouns). That is, one agreement class goes with more than one nominal form class.

\ea%3
    \label{ex:Gueld:3}
\langinfo{Swahili}{}{personal knowledge}
\begin{xlist}
\ex
\gll \textbf{m}{}-ti  \textbf{u}{}-le  \textbf{m}{}-moja  \textbf{u}{}-me-anguka\\
     \textbf{\textsc{m(w)}}{}-tree(\textbf{3})  \textbf{3}{}-\textsc{d.dem}  \textbf{3}{}-one  \textbf{3}{}-\textsc{perf}{}-fall\\
\glt `that one tree has fallen'
\ex
\gll \textbf{mi}{}-ti  \textbf{i}{}-le  \textbf{mi}{}-wili  \textbf{i}{}-me-anguka\\
     \textbf{\textsc{mi}}{}-tree(\textbf{4})  \textbf{4}{}-\textsc{d.dem}  \textbf{4}{}-two  \textbf{4}{}-\textsc{perf}{}-fall\\
\glt `those two trees have fallen'
\end{xlist}
\z

Example (\ref{ex:Gueld:3}) illustrates that one nominal form class can also be associated with more than one agreement class \textendash{} the reverse case of the situation illustrated in (\ref{ex:Gueld:2}). As shown in (\ref{ex:Gueld:3}a), NF \textit{M(W)-} is not exclusively tied to AGR1 in the human gender AGR1/AGR2, as in (\ref{ex:Gueld:1}a), but is also relevant for singular forms of lexemes like \textit{{}-ti} `tree' in AGR3 belonging to the gender AGR3/AGR4. The matching of one nominal form class with more than one agreement class equally holds for NF \textit{MA-} in (\ref{ex:Gueld:2}b), because it is also found with plural count nouns of the gender AGR5/AGR6 and with transnumeral nouns for masses and liquids.

To reiterate the point, the philological ``noun class'' notion inadequately implies the universality of a one-to-one trigger-target mapping, thereby silently conflating the categories of agreement class and nominal form class that are in principle independent. Counterfeiting an ideal system, this concept recurrently decoys scholars into the analytical shortcut illustrated in the following.

Assume a language with gender and nominal deriflection where agreement and nominal form classes display a biunique mapping. Such a situation is represented in \figref{fig:Gueld:3} (which differs from figures focusing on gender and deriflection systems such as \ref{fig:Gueld:1} and \ref{fig:Gueld:2} above or \ref{fig:Gueld:4} below). In both domains, the classes are further assumed to map over number such that two apply to singular nouns and one to plural nouns. Such a system would allow one to predict AGR1, AGR2 and AGR3 from NF A, NF B and NF C, respectively, and vice versa \textendash{} a situation that implies a strong formal assignment of agreement (see \citealt[Chapter~3]{Corbett1991}).


\begin{figure}[htb]
\centering
\begin{tabular}{rp{\llen}ll}
\lsptoprule
AGR \tknode{0} & & \tknode{0} NF &  Number \\
\midrule
\padding
1 \tknode{3A1} & & \tknode{3A2} A & SG\\
\padding
2 \tknode{3B1} & & \tknode{3B2} B & SG\\
\padding
3 \tknode{3C1} & & \tknode{3C2} C & PL\\
\lspbottomrule
\end{tabular}

% lines connecting the nodes and labels
\tikz[remember picture, overlay] \draw[thick] (3A1.center) -- (3A2.center) ;
\tikz[remember picture, overlay] \draw[thick] (3B1.center) -- (3B2.center) ;
\tikz[remember picture, overlay] \draw[thick] (3C1.center) -- (3C2.center) ;

\caption{Full one-to-one mapping of agreement classes and nominal form classes}
\label{fig:Gueld:3}
\end{figure}


\figref{fig:Gueld:4} shows the resulting agreement-based gender system (left side) and the deriflection system based on nominal form classes (right side), which can also be inferred from each other. Here, both show convergence from two singular classes to one plural class. This predictability holds irrespective of whether the exponents in the system of agreement and nominal morphology display alliteration of the type recurrent in \ili{Niger-Congo} (cf.\ (\ref{ex:Gueld:1}b) from \ili{Swahili}).

\begin{figure}[htb]

\begin{minipage}{.4\textwidth}
\centering
\begin{tabular}{rp{\llen}l}
\lsptoprule
SG \tknode{0} & & \tknode{0} PL \\
\midrule
AGR1 \tknode{4A} \\
&&\tknode{4B} AGR3\\
AGR2 \tknode{4C}\\
\lspbottomrule
\end{tabular}

% lines connecting the nodes and labels
\tikz[remember picture, overlay] \draw[thick] (4A.center) -- (4B.center) ;
\tikz[remember picture, overlay] \draw[thick] (4C.center) -- (4B.center) ;

\end{minipage}
%
\begin{minipage}{.4\textwidth}
\centering
\begin{tabular}{rp{\llen}l}
\lsptoprule
SG \tknode{0} & & \tknode{0} PL \\
\midrule
NF A \tknode{4D} \\
&&\tknode{4E} NF C\\
NF B \tknode{4F}\\
\lspbottomrule
\end{tabular}

% lines connecting the nodes and labels
\tikz[remember picture, overlay] \draw[thick] (4D.center) -- (4E.center) ;
\tikz[remember picture, overlay] \draw[thick] (4F.center) -- (4E.center) ;

\end{minipage}

\caption{Gender system (left) vs.\ deriflection system (right) of the case in \figref{fig:Gueld:3}}
\label{fig:Gueld:4}
\end{figure}


In reality, however, an ``ideal'' trigger-target mapping as in \figref{fig:Gueld:3} is never universal in a language so that the ``noun class'' approach harbors the risk of misleading analysis. This can be illustrated by means of a rather well-behaved attested system, like that of \ili{Ikaan} (\ili{Benue-Kwa}). \figref{fig:Gueld:5} shows that there is only a single exception to a complete one-to-one mapping between agreement classes and prefixal nominal form classes, namely NF \textit{O-} that is associated with AGR1 \emph{and} AGR6. Hence, the system appears to be overall well described in terms of the canonical unitary concept of ``noun class'' involving both the forms of nominal prefixes and concords on agreement targets.
%
%AGR    NF  Number
%
%[Warning: Draw object ignored]6  \textit{nɔ:}    SG
%
%[Warning: Draw object ignored]1  \textit{jõ\`{} :}  \textit{O-}   SG
%
%[Warning: Draw object ignored]2  \textit{dà:}  \textit{A-}   TN, PL
%
%[Warning: Draw object ignored]3  \textit{dɔ:}  \textit{U-}   SG
%
%[Warning: Draw object ignored]4  \textit{dɛ:}  \textit{I-}   SG, PL
%
%[Warning: Draw object ignored]5  \textit{nɛ:}  \textit{E-}   SG
%
%Note: agreement classes represented by proximal demonstratives
%
%\begin{styleHeadingix}\begin{figure}
%\caption{Mapping of agreement and nominal form classes in Ikaan (based on \citealt{Borchardt2011}: 75-8)}
%\label{fig:Gueld:5}
%\end{figure}\end{styleHeadingix}

\begin{figure}[p]

\begin{tabular}{l@{\hspace{.1em}}rp{\llen}ll}
\lsptoprule
 &AGR \tknode{0} && \tknode{0} NF & Number \\
\midrule
\padding
6 & \textit{n\`{ɔ}:} \tknode{5A1}&& & SG \\
\padding
1 & \textit{j\`{õ}:} \tknode{5A2} && \tknode{5B2} \textit{O-} & SG \\
\padding
2 &  \textit{dà:} \tknode{5A3} && \tknode{5B3} \textit{A-} & TN, PL \\
\padding
3 &  \textit{d\`{ɔ}:} \tknode{5A4} && \tknode{5B4} \textit{U-} & SG \\
\padding
4 & \textit{d\`{ɛ}:} \tknode{5A5} && \tknode{5B5} \textit{I-} & SG, PL \\
\padding
5 & \textit{n\`{ɛ}:} \tknode{5A6} && \tknode{5B6} \textit{E-} & SG \\
\lspbottomrule
\end{tabular}

% lines connecting the nodes and labels
\tikz[remember picture, overlay] \draw[thick] (5A1.center) -- (5B2.center) ;
\tikz[remember picture, overlay] \draw[thick] (5A2.center) -- (5B2.center) ;
\tikz[remember picture, overlay] \draw[thick] (5A3.center) -- (5B3.center) ;
\tikz[remember picture, overlay] \draw[thick] (5A4.center) -- (5B4.center) ;
\tikz[remember picture, overlay] \draw[thick] (5A5.center) -- (5B5.center) ;
\tikz[remember picture, overlay] \draw[thick] (5A6.center) -- (5B6.center) ;

{\small Note: agreement classes represented by proximal demonstratives}

\caption{Mapping of agreement and nominal form classes in Ikaan (based on \citealt[75--78]{Borchardt2011})}
\label{fig:Gueld:5}
\end{figure}


With such a neat mapping one may be tempted to proceed according to the discussion revolving around Figures \ref{fig:Gueld:3} and \ref{fig:Gueld:4} and infer the agreement-based gender system from the morphological deriflection system based on nominal form classes (or vice versa). \figref{fig:Gueld:6} shows the two systems side by side for a better comparison. For the record, the two schemas also display a class of transnumeral (TN) nouns marked by circles, which cannot be assigned clearly to a single paired pattern and thus should be recognized as a separate gender. The nature of the various genders and deriflections, including their possible semantics, is largely irrelevant for the present topic and they are therefore not labeled or numbered \textendash{} a practice also relevant later on, especially in system schemas like those in \figref{fig:Gueld:6}.


\begin{figure}[p]

\begin{minipage}{.45\textwidth}
\centering

\begin{tabular}{ll>{\centering}p{\llen}l}
\lsptoprule
AGR & SG & TN & \tknode{0} PL \\
\midrule
\padding
1 & \textit{j\`{õ}:} \tknode{6A1} & \\
\padding
2 & & %
\tikz[remember picture,baseline=(6circ1.base)]\node[circle,inner sep=0pt,draw] (6circ1) {\textit{dà:}}; &%
 \tknode{6B2} \textit{dà:}   \\
\padding
3 &\textit{d\`{ɔ}:} \tknode{6A3} & \\
\padding
4 & \textit{d\`{ɛ}:} \tknode{6A4} & & \tknode{6B4} \textit{d\`{ɛ}:}\\
\padding
5 & \textit{n\`{ɛ}:} \tknode{6A5} & \\
\padding
6 & \textit{n\`{ɔ}:} \tknode{6A6} & \\
\lspbottomrule
\end{tabular}

% lines connecting the nodes and labels
\tikz[remember picture, overlay] \draw[thick] (6A1.center) -- (6B2.center) ;
\tikz[remember picture, overlay] \draw[thick] (6A3.center) -- (6B2.center) ;
\tikz[remember picture, overlay] \draw[thick] (6A4.center) -- (6B2.center) ;
\tikz[remember picture, overlay] \draw[thick] (6A5.center) -- (6B4.center) ;
\tikz[remember picture, overlay] \draw[ultra thick] (6A6.center) -- (6B4.center) ;

\end{minipage}
%
\begin{minipage}{.45\textwidth}
\centering

\begin{tabular}{r>{\centering}p{\llen}l}
\lsptoprule
 NF SG \tknode{0} & NF TN & \tknode{0} NF PL \\
\midrule
\padding
  \textit{O-} \tknode{6C1} & \\
\padding
  & %
\tikz[remember picture,baseline=(6circ1.base)]\node[circle,inner sep=0pt,draw] (6circ1) {\textit{A-}}; &%
 \tknode{6D2} \textit{A-}   \\
\padding
 \textit{U-} \tknode{6C3} & \\
\padding
  \textit{I-} \tknode{6C4} & & \tknode{6D4} \textit{I-}\\
\padding
  \textit{E-} \tknode{6C5} & \\
\padding
 & & \\
\lspbottomrule
\end{tabular}

% lines connecting the nodes and labels
\tikz[remember picture, overlay] \draw[thick] (6C1.center) -- (6D2.center) ;
\tikz[remember picture, overlay] \draw[ultra thick] (6C1.center) -- (6D4.center) ;
\tikz[remember picture, overlay] \draw[thick] (6C3.center) -- (6D2.center) ;
\tikz[remember picture, overlay] \draw[thick] (6C4.center) -- (6D2.center) ;
\tikz[remember picture, overlay] \draw[thick] (6C5.center) -- (6D4.center) ;


\end{minipage}

\caption{Gender system (left) vs.\ deriflection system (right) of Ikaan (based on \citealt{Borchardt2011}: 75--78)}
\label{fig:Gueld:6}
\end{figure}

\largerpage
The important observation from \figref{fig:Gueld:6} is that the single exception to a biunique class mapping in \figref{fig:Gueld:5} causes a clear structural divergence between the gender and deriflection systems, as marked by the two thick lines. The difference can be explained in terms of the typology for the mapping of classes across number categories originally proposed by \citet[196--198]{Heine1982} and elaborated by \citet[154--158]{Corbett1991}. There are three major types in the order of increasing complexity:

 
\begin{enumerate}
\item[a.] \spterm{parallel}: Singular and plural classes only show one-to-one mapping.

\item[b.] \spterm{convergent}: At least two agreement classes in one number converge to one class in the other number.

\item[c.] \spterm{crossed}: Class convergence exists in both directions.
\end{enumerate}


According to this typology, \ili{Ikaan}'s real gender system based on agreement is of the convergent type in that the conflation of classes only goes from singular to plural, while its deriflection system based on nominal form classes shows class convergence in both directions and is thus of the more complex crossed type.

In fact, the divergence between gender and deriflection system in \ili{Ikaan} is almost certainly greater, because the language will have prefixless nouns (e.g., proper names, loans), which are unfortunately not treated in the available sources. These establish an additional nominal form class that does not have a unique counterpart in the agreement system. Since such an additional unmarked Ø{}-nominal form class can be expected to be virtually universal, this phenomenon alone excludes the one-to-one class mapping and hence the identity of the gender and deriflection system from a general perspective.

The divergence between gender system and ``gender-like'' deriflection system holds to an even greater extent in \ili{Bantu} \textendash{} the very language group in which the problematic ``noun class'' concept was developed and from where it assumed its model role for the larger family. This can be illustrated by means of Proto-\ili{Bantu} for which there exists an elaborate reconstruction. Irrespective of its full historical adequacy, the detailed information of this proto-system allows a good approximation to the original situation regarding (a) the mapping of agreement classes and nominal form classes, (b) the gender system based on agreement classes, and (c) the deriflection system based on nominal form classes.

\begin{table}[b]
\begin{tabularx}{\textwidth}{XXXXXXXX}
\lsptoprule

``Noun

class'' & Number & AGR & \multicolumn{4}{l}{Different agreement targets} & NF\\
&  &  & CONC & NUM & SBJ & \multicolumn{1}{X}{OBJ} & \\
\midrule
*2 & PL & 2 & ba- & ba- & ba- & ba- & ba-\\
\emcell *1 & SG &\emcell  1 & ju- & u- ? & u-, a- & mu- & \multirow{3}{*}{\emcell}\\
\emcell *18 & TN &\emcell  18 & mu- & mu- & mu- & mu- & \emcell \\
\emcell *3 & SG &\emcell  3 & gu- & u- ? & gu- & \multicolumn{1}{X}{gu-} & \multirow{-3}{*}{\emcell mu-}\\
*4 & PL & 4 & gi- & i- ? & gi- & gi- & mi-\\
*5 & SG & 5 & di- & di- & di- & di- & i̜-\\
*6 & TN, PL & 6 & ga- & a- ? & ga- & ga- & ma-\\
*7 & SG & 7 & ki- & ki- & ki- & ki- & ki-\\
*8 & PL & 8 & bi̜- & bi̜- & bi̜- & bi̜- & bi̜-\\
\emcell *9 & SG &\emcell  9 & ji- & i- ? & ji- & ji- & \multirow{2}{*}{\emcell}\\
\emcell *10 & PL &\emcell 10 & ji̜- & i̜- & ji̜- & ji̜- & \multirow{-2}{*}{\emcell N-}\\
*11 & SG & 11 & du- & du- & du- & du- & du-\\
*12 & SG & 12 & ka- & ka- & ka- & ka- & ka-\\
*13 & PL & 13 & tu- & tu- & tu- & tu- & tu-\\
*14 & TN, SG & 14 & bu- & bu- & bu- & bu- & bu-\\
\emcell *15 & TN, SG & \multirow{2}{*}{\emcell} & \multirow{2}{*}{ku-} & \multirow{2}{*}{ku-} & \multirow{2}{*}{ku-} & \multirow{2}{*}{ku-} &  \multirow{2}{*}{\emcell }\\
\emcell *17 & TN & \multirow{-2}{*}{\emcell 15/17}  &  &  &  &  & \multirow{-2}{*}{\emcell ku-}\\
*16 & TN & 16 & pa- & pa- & pa- & pa- & pa-\\
*19 & SG & 19 & pi̜- & pi̜- & pi̜- & pi̜- & pi̜-\\
\lspbottomrule
\end{tabularx}

\caption{Proto-Bantu ``noun classes'' (conflating agreement classes and nominal form classes) (based on \citealt[96--99]{Meeussen1967})}
\label{tab:Gueld:2}
\end{table}

Excluding an uncertain proto-class *24, \tabref{tab:Gueld:2} presents Meeussen's (\citeyear[96--99]{Meeussen1967}) full reconstruction of the Proto-\ili{Bantu} ``noun classes'', which, as mentioned, conflate agreement and noun form. This framework is the outcome of specific developments in \ili{Bantu} philology, without much consideration of the typological treatment of gender. Hence, it comes as no surprise that it is multiply incompatible with the cross-linguistic approach proposed here.

The divergence between the above \ili{Bantu} reconstruction and our approach concerns in particular various mismatches between the philological ``noun class'' inventory in the leftmost column and our analysis that involves the agreement classes in the third column (followed by four columns displaying the exponents of major targets) and the nominal form classes in the rightmost column (we take over the philological class numbering 1--19 for our agreement classes, while nominal form classes are simply referred to by their reconstructed prefix).

The major differences between the \ili{Bantu} reconstruction and the present analysis, marked in \tabref{tab:Gueld:2} by shaded cells, are as follows. First, two nominal form classes, namely those established by the noun prefixes *mu- and *N- have a multiple affiliation with agreement classes, the former occurring with nouns of the agreement classes *1, *3, and *18 (cf. the above discussion in connection with (\ref{ex:Gueld:1}a) and (\ref{ex:Gueld:3}a) from \ili{Swahili}) and the latter with nouns of the classes *9 and *10. Second, two ``noun classes'' of the \ili{Bantu} tradition that establish single-class sets of transnumeral nominals should be subsumed under a single noun form and agreement class, because they do not diverge in either nominal prefix or concord. Their difference only concerns the syntactic occurrence of the respective nominal in that ``noun class'' *15 comprises infinitives, while ``noun class'' *17 is established by the class of general locatives.%
\footnote{For the record, class *15 is most likely a grammaticalization from class *17 via the path locative > purposive > infinitive (cf.\ \citealt{Haspelmath1989}).} In general one can conclude that the traditional identification and numbering of ``noun classes'' in \ili{Bantu} predominantly target agreement classes. As will be shown in \sectref{sec:Gueld:3}, this situation no longer holds for the application of the approach to many other \ili{Niger-Congo} languages, where the analysis of ``noun classes'' often, if implicitly, refers to nominal form classes.

Later approaches to \ili{Bantu} gender systems have introduced yet other conventions that may have enhanced philological comparability but blur cross-linguistic transparency. In particular, Bantuists (and some scholars like \citealt[166]{Welmers1973} dealing with other \ili{Niger-Congo} languages) make an additional ``noun class'' distinction of *1 vs.\ *1a (and possibly *2 vs.\ *2a). The first class of each pair comprises human nouns with the expected prefix and the latter contains prefixless kinship nouns and proper names. While descriptively adequate, this class differentiation is irrelevant for the inventory of agreement classes but more importantly hides the necessity of taking into account an additional nominal form class Ø that has no unique counterpart in the agreement system (cf.\ the above discussion in connection with (\ref{ex:Gueld:1}a) and (\ref{ex:Gueld:2}a) from \ili{Swahili}).%
\footnote{See \citet{VandeVelde2006} for an extensive recent discussion of such nouns in \ili{Eton} and \ili{Bantu} in general. We do not follow his proposal of considering them as ``genderless'' nouns, because gender is defined here by agreement and their predominant behavior in this respect clearly assigns them to the human gender.}


\begin{figure}[t]
\begin{tabular}{lrp{\llen}ll}
\lsptoprule
AGR &&  &\tknode{0}  NF&  Number \\
\midrule
X &&&  \tknode{7B1}    Ø &  SG, PL \\
\llap{*}1(a) &  u-, a- \tknode{7A2} &  & \tknode{7B2} \llap{*}mu- &  SG \\
\llap{*}3 & gu- \tknode{7A3} && \tknode{7B3}  X&  SG\\
\llap{*}18 &  mu- \tknode{7A4} &&  \tknode{7B4} X &  TN \\
\llap{*}2  & ba-  \tknode{7A5} & &  \tknode{7B5} \llap{*}ba-  & PL\\
\llap{*}4  & gi-  \tknode{7A6} & & \tknode{7B6} \llap{*}mi-  & PL\\
\llap{*}15/17 &  ku-  \tknode{7A7} & & \tknode{7B7} \llap{*}ku-  & TN, SG\\
\llap{*}5 &  di- \tknode{7A8} & & \tknode{7B8}  \llap{*}i̜-  & SG\\
\llap{*}6  & ga- \tknode{7A9}  & & \tknode{7B9} \llap{*}ma-  & TN, PL\\
\llap{*}14  & bu-  \tknode{7A10} & & \tknode{7B10} \llap{*}bu- &  TN, SG\\
\llap{*}7  & ki- \tknode{7A11} & & \tknode{7B11}  \llap{*}ki- &  SG\\
\llap{*}8  & bi̜-  \tknode{7A12} & & \tknode{7B12} \llap{*}bi̜- &  PL\\
\llap{*}9  & ji- \tknode{7A13}  & & \tknode{7B13} \llap{*}n- &  SG\\
\llap{*}10  & ji̜-  \tknode{7A14} & & \tknode{7B14} X  & PL\\
\llap{*}11  & du-  \tknode{7A15} & &  \tknode{7B15} \llap{*}du-  & SG\\
\llap{*}12  & ka- \tknode{7A16}  & &  \tknode{7B16} \llap{*}ka- &  SG\\
\llap{*}13  & tu- \tknode{7A17}  & & \tknode{7B17}  \llap{*}tu- &  PL\\
\llap{*}16  & pa- \tknode{7A18}  & & \tknode{7B18} \llap{*}pa-  &  TN\\
\llap{*}19  & pi̜- \tknode{7A19}  & & \tknode{7B19}  \llap{*}pi̜-  &  SG\\
\lspbottomrule
\end{tabular}

% lines connecting the nodes and labels
\tikz[remember picture, overlay] \draw[thick] (7A2.center) -- (7B1.center) ;
\tikz[remember picture, overlay] \draw[thick] (7A2.center) -- (7B2.center) ;
\tikz[remember picture, overlay] \draw[thick] (7A3.center) -- (7B2.center) ;
\tikz[remember picture, overlay] \draw[thick] (7A4.center) -- (7B2.center) ;
\tikz[remember picture, overlay] \draw[thick] (7A5.center) -- (7B5.center) ;
\tikz[remember picture, overlay] \draw[thick] (7A6.center) -- (7B6.center) ;
\tikz[remember picture, overlay] \draw[thick] (7A7.center) -- (7B7.center) ;
\tikz[remember picture, overlay] \draw[thick] (7A8.center) -- (7B8.center) ;
\tikz[remember picture, overlay] \draw[thick] (7A9.center) -- (7B9.center) ;
\tikz[remember picture, overlay] \draw[thick] (7A10.center) -- (7B10.center) ;
\tikz[remember picture, overlay] \draw[thick] (7A11.center) -- (7B11.center) ;
\tikz[remember picture, overlay] \draw[thick] (7A12.center) -- (7B12.center) ;
\tikz[remember picture, overlay] \draw[thick] (7A13.center) -- (7B13.center) ;
\tikz[remember picture, overlay] \draw[thick] (7A14.center) -- (7B13.center) ;
\tikz[remember picture, overlay] \draw[thick] (7A15.center) -- (7B15.center) ;
\tikz[remember picture, overlay] \draw[thick] (7A16.center) -- (7B16.center) ;
\tikz[remember picture, overlay] \draw[thick] (7A17.center) -- (7B17.center) ;
\tikz[remember picture, overlay] \draw[thick] (7A18.center) -- (7B18.center) ;
\tikz[remember picture, overlay] \draw[thick] (7A19.center) -- (7B19.center) ;

{\small Note: X = no independent class counterpart in the other class type.}


\caption{Mapping of agreement and nominal form classes in Proto-Bantu}
\label{fig:Gueld:7}
\end{figure}

\figref{fig:Gueld:7} shows the mapping of agreement and nominal form classes in Proto-\ili{Bantu} arising from \tabref{tab:Gueld:2} (including the additional ``noun class'' *1a). Overall, one-to-one trigger-target mapping as well as alliteration are salient but also have important exceptions. The different number of agreement classes and nominal form classes alone, namely 18 vs.\ 16, implies that the gender system and the deriflection system of Proto-\ili{Bantu} cannot turn out to be completely parallel. In this context, the symbol X in this and later schemas stands for the case where no unique counterpart exists for a class in the opposite class type. (The alignment between classes of different type by a horizontal or a sloping line is arbitrary in \figref{fig:Gueld:7}; in the case of historically rooted alliteration, it is useful to connect such etymologically ``proper'' counterparts by the horizontal line, which will be done in appropriate cases later on.)

\begin{figure}[t]

\begin{tabular}{lr>{\centering}p{\llen}l l r>{\centering}p{\llen}l}
\cmidrule{1-4}\cmidrule{6-8}
\addlinespace[-\aboverulesep]
\cmidrule[\heavyrulewidth]{1-4}\cmidrule[\heavyrulewidth]{6-8}
AGR  & SG \tknode{0} &  TN & \tknode{0} PL & & NF SG \tknode{0} & NF TN & \tknode{0} NF PL \\
\cmidrule{1-4}\cmidrule{6-8}
X & &&&& Ø \tknode{8C1} && \tknode{8D1} Ø \\
\llap{*}1(a) & u-, a- \tknode{8A2}& & & & MU- \tknode{8C2} & %
\tikz[remember picture,baseline=(8circ5.base)]\node[circle,inner sep=0pt,draw] (8circ5) {MU-} ; & \\
\llap{*}2 & && \tknode{8B3} ba- & &&& \tknode{8D3} BA-  \\
\llap{*}3 & gu- \tknode{8A4} &&&&X \tknode{0} \\
\llap{*}4 & && \tknode{8B5} gi- & &&& \tknode{8D5} MI-\\
\llap{*}15/17 & ku- \tknode{8A6} & %
\tikz[remember picture,baseline=(8circ1.base)]\node[circle,inner sep=0pt,draw] (8circ1) {ku-}; &  & & KU- \tknode{8C6} &%
\tikz[remember picture,baseline=(8circ6.base)]\node[circle,inner sep=0pt,draw] (8circ6) {KU-} ; & \\
\llap{*}5 & di- \tknode{8A7} &&&& \k{I}- \tknode{8C7} \\
\llap{*}6 & &%
\tikz[remember picture,baseline=(8circ2.base)]\node[circle,inner sep=0pt,draw] (8circ2) {ga-}; & \tknode{8B8} ga- && & %
\tikz[remember picture,baseline=(8circ7.base)]\node[circle,inner sep=0pt,draw] (8circ7) {MA-} ; & \tknode{8D8} MA- \\
\llap{*}14 & bu- \tknode{8A9} & %
\tikz[remember picture,baseline=(8circ3.base)]\node[circle,inner sep=0pt,draw] (8circ3) {bu-}; & & & BU- \tknode{8C9} & %
\tikz[remember picture,baseline=(8circ8.base)]\node[circle,inner sep=0pt,draw] (8circ8) {BU-} ; & \\
\llap{*}7 & ki- \tknode{8A10} &&&& KI- \tknode{8C10} \\
\llap{*}8 & & & \tknode{8B11} bi̜- && && \tknode{8D11} B\k{I}{}- \\
\llap{*}9 & ji- \tknode{8A12} &&& &X \tknode{0} \\
\llap{*}10 & & & \tknode{8B13} ji̜- & & N- \tknode{8C13} & & \tknode{8D13} N- \\
\llap{*}11 & du- \tknode{8A14} &&&& DU- \tknode{8C14} \\
\llap{*}12 & ka- \tknode{8A15} &&&& KA- \tknode{8C15} \\
\llap{*}13 & & & \tknode{8B16} tu- &&&& \tknode{8D16} TU- \\
\llap{*}19 & pi̜- \tknode{8A17} &&&& P\k{I} \tknode{8C17} \\
\llap{*}16 & & %
\tikz[remember picture,baseline=(8circ4.base)]\node[circle,inner sep=0pt,draw] (8circ4) {pa-} ; & & & & %
\tikz[remember picture,baseline=(8circ9.base)]\node[circle,inner sep=0pt,draw] (8circ9) {PA-} ; & \\
\llap{*}18 & & %
\tikz[remember picture,baseline=(8circ4.base)]\node[circle,inner sep=0pt,draw] (8circ4) {mu-} ; & && X \tknode{0} & \\
\cmidrule[\heavyrulewidth]{1-4}\cmidrule[\heavyrulewidth]{6-8}
\addlinespace[-\aboverulesep]
\cmidrule{1-4}\cmidrule{6-8}
\end{tabular}

% lines connecting the nodes and labels
% left panel
\tikz[remember picture, overlay] \draw[thick] (8A2.center) -- (8B3.center) ;
\tikz[remember picture, overlay] \draw[thick] (8A4.center) -- (8B5.center) ;
\tikz[remember picture, overlay] \draw[thick] (8A6.center) -- (8B8.center) ;
\tikz[remember picture, overlay] \draw[thick] (8A7.center) -- (8B8.center) ;
\tikz[remember picture, overlay] \draw[thick] (8A9.center) -- (8B8.center) ;
\tikz[remember picture, overlay] \draw[thick] (8A10.center) -- (8B11.center) ;
\tikz[remember picture, overlay] \draw[thick] (8A12.center) -- (8B13.center) ;
\tikz[remember picture, overlay] \draw[thick] (8A14.center) -- (8B13.center) ;
\tikz[remember picture, overlay] \draw[thick] (8A15.center) -- (8B16.center) ;
\tikz[remember picture, overlay] \draw[thick] (8A17.center) -- (8B16.center) ;
% right panel
\tikz[remember picture, overlay] \draw[thick] (8C1.center) -- (8D1.center) ;
\tikz[remember picture, overlay] \draw[thick] (8C1.center) -- (8D3.center) ;
\tikz[remember picture, overlay] \draw[thick] (8C2.center) -- (8D3.center) ;
\tikz[remember picture, overlay] \draw[thick] (8C2.center) -- (8D5.center) ;
\tikz[remember picture, overlay] \draw[thick] (8C6.center) -- (8D8.center) ;
\tikz[remember picture, overlay] \draw[thick] (8C7.center) -- (8D8.center) ;
\tikz[remember picture, overlay] \draw[thick] (8C9.center) -- (8D8.center) ;
\tikz[remember picture, overlay] \draw[thick] (8C10.center) -- (8D11.center) ;
\tikz[remember picture, overlay] \draw[thick] (8C13.center) -- (8D13.center) ;
\tikz[remember picture, overlay] \draw[thick] (8C14.center) -- (8D13.center) ;
\tikz[remember picture, overlay] \draw[thick] (8C15.center) -- (8D16.center) ;
\tikz[remember picture, overlay] \draw[thick] (8C17.center) -- (8D16.center) ;


{\small Note: X = no independent counterpart in the other class type.}

\caption{Gender system (left) vs.\ deriflection system (right) of Proto-Bantu}
\label{fig:Gueld:8}
\end{figure}

A full comparison of the gender and the deriflection systems in Proto-\ili{Bantu} as reconstructed from the hypothesized ``noun class'' behavior is shown in \figref{fig:Gueld:8}, which follows the presentation in Figures \ref{fig:Gueld:4} and \ref{fig:Gueld:6}. In the gender system on the left side of \figref{fig:Gueld:8}, at least some transnumeral noun groups marked by circles must be analyzed as establishing genders in their own right, because the respective agreement classes cannot be unambiguously associated with a single paired gender, as is the case for AGR6, AGR16, and AGR18 (AGR15/17 and AGR14 are arguably singularia tantum of two paired genders with AGR6 in the plural).

As can be expected, \figref{fig:Gueld:8} demonstrates considerable differences between the gender and the deriflection system, even more extensively than in \ili{Ikaan}, despite the still considerable one-to-one alliterative mapping shown in \figref{fig:Gueld:7}. While the gender system with 18 agreement classes is convergent in the above terms and comprises 10 paired genders and at least 3 single-class genders, the deriflection system with 16 nominal form classes is crossed and involves 12 types of morphological number alternations besides 5 types of transnumeral nouns.


Similar or even more dramatic cases of divergence between the gender system and the ``gender-like'' deriflection system are normal in \ili{Niger-Congo}, and the problems associated with the traditional ``noun class'' concept have been recognized in both language-specific and comparative research. The reader is referred to the revealing theoretical and methodological discussion in such studies as \citet{Guthrie1948} for \ili{Bantu}, and \citet{Voorhoeve1969} and \citet{DeWolf1971} for \ili{Benue-Congo}. As a consequence, \citet[33f]{Mieheforthcoming} explicitly states that ``the marking of nouns and the concord (agreement) systems in their formal and semantic multiplicity should be considered as independent paradigms with regard to their evolution.''

Nevertheless, the philological tradition is so strong that even the only approach known to us that uses the very same analytical concepts as ours yields an analysis that is far from being transparent, namely that by \citet{Sterk1978} for the \ili{Nupoid} language \ili{Gade}.

\begin{table}[b]
\begin{tabularx}{\textwidth}{XQXQX}
\lsptoprule
& ``Prefix'' =~nominal form class & ``Declension'' =~deriflection & ``Class'' =~agreement class & ``Gender''\\
\midrule
``form'' & $+$ & $+$ & $-$ & $-$\\
``concord'' & $-$ & $-$ & $+$ & $+$\\
``pairing'' & $-$ & $+$ & $-$ & $+$\\
\lspbottomrule
\end{tabularx}
{\small Note: ``...'' = Sterk's (\citeyear{Sterk1978}) term.}

\caption{Sterk's (\citeyear[25]{Sterk1978}) concepts for analyzing Gade ``noun classes''}
\label{tab:Gueld:3}
\end{table}

\largerpage
\tabref{tab:Gueld:3} betrays hardly any difference to our outline of analytical concepts in \tabref{tab:Gueld:1} of \sectref{sec:Gueld:1}. The only point is Sterk's overgeneralization of the singular-plural pairing of classes with count nouns in that his last line of the table prescribes the feature ``pairing'' for ``declension'' (a.k.a.\ deriflection) and gender, thus excluding single class patterns with transnumeral nouns.

The real drawback in his description is his complex numbering of ``classes'', which aims to cater simultaneously for their morphological shape and their agreement behavior. He writes (ibid.: 27):


\begin{quote}
We are now faced with the practical problem of how to classify \ili{Gade} nouns. Noun stems will have to be specified both for declension and for gender, since the one cannot always be predicted from the other. Rather than list noun stems in the lexicon with the double marking however, it is more convenient to devise a system which classifies them unambiguously, both for declension and for gender, with a \emph{single marker}. This will be done by assigning numbers to prefixes, with the proviso: not only will prefixes of differing phonological shape be assigned a different number, but even prefixes of the same shape will be given a different number if the nouns they form part of belong to different [agreement] classes. (emphasis and additions ours).
\end{quote}


\rephrase{Sterk's single-marker convention}{The single-marker convention proposed by Sterk}, which appears to be motivated by the equally conflating ``noun class'' concept, is the major reason that his presentation falls short of providing a transparent picture of \ili{Gade}'s nominal system (cf.\ also Sterk's (\citeyear{Sterk1976}) similarly complicated treatment of the \ili{Upper Cross} language \ili{Humono}). Our analysis concludes that \ili{Gade} has a complex deriflection system of more than 30 patterns (albeit many restricted to very few if not a single noun lexeme) based on 13 nominal form classes but a relatively simple system of three productive (and two inquorate) genders based on four regular agreement classes.

Comparing the situation in \ili{Ikaan}, Proto-\ili{Bantu}, and \ili{Gade} a potential generalization emerges: in all cases, the agreement-based gender system is simpler (or at least not more complex) than the deriflection system in size and structure \textendash{} this even if the basic inventory of agreement and nominal form classes shows the opposite picture, as is the case in Proto-\ili{Bantu}. More data supporting this observation follow in \sectref{sec:Gueld:3} regarding other \ili{Niger-Congo} languages.

The previous discussion has argued that the \ili{Niger-Congo} concept and term ``noun class'' is highly problematic. This is compounded by the fact that the term has come to bear different meanings in \ili{Niger-Congo} studies, depending on diverse language-specific situations. Thus, in languages that lost (most of) the inherited agreement, ``noun class'' may just refer to nominal form classes, as in some \ili{Gur} languages, for example, \ili{Moore} (\citealt{Canu1976}), or in the Idomoid language \ili{Igede} (\citealt{Abiodun1989}) (see also \citealt{Good2012a}: §4.2). In a parallel fashion, in the apparently rarer case of the loss of transparent noun affixes with retention of agreement, the term ``noun class'' tends to mean merely agreement class, as is the case to varying degrees in \ili{Wolof} from \ili{Atlantic} (\citealt{Babou2016}) and \ili{Mundabli} from \ili{Bantoid} \citep{Voll2017} (see also \citealt{Good2012a}: §4.3). Finally, the discussion in \sectref{sec:Gueld:3.2} below about \ili{Akan} shows that some authors even use ``noun class'' for deriflection (class). From a global typological perspective, yet another complication arises from the terminological tradition in other geographical areas: in Caucasian and partly Australian languages, the term ``noun class'' refers to gender. The same usage has been proposed by \citet{Aikhenvald2000} for typological investigation in general, the term ``gender'' being restricted to sex-based systems. We consider this proposal to be unfortunate because it not only diverges from Corbett's (\citeyear{Corbett1991}) earlier and widely accepted terminology but also disregards the fact that in \ili{Niger-Congo}, the largest language family on the globe where ``noun class'' plays a central role, it does conventionally not refer to gender (pace the statement in some relevant studies, e.g., \citealt[1]{Kilarski2013}). In view of the multiple ambiguity of the term ``noun class'', covering in fact all the four analytical concepts outlined in \sectref{sec:Gueld:1}, we do not use the term in any other meaning than the original philological one in \ili{Niger-Congo} and employ it in quotation marks for the sake of clarity.

\section{Examples for the treatment of individual Niger-Congo groups}
\label{sec:Gueld:3}



\subsection{Introduction}

As was said above, the approach to \ili{Niger-Congo} gender and deriflection systems in terms of ``noun classes'' has been and still is the rule. In the following we show that as a result analyses of individual languages and attempted reconstructions of language groups%
\footnote{Until now, (partial) reconstructions of gender and deriflection systems exist for relatively few of the numerous \ili{Niger-Congo} groups. In addition to \ili{Bantu}, we are aware of those for \ili{Gur} (\citealt{Manessy1967}, \citeyear{Manessy1975}; \citealt{Miehe2012}), \ili{Ghana-Togo-Mountain} (\citealt{Heine1968}, see \sectref{sec:Gueld:3.3}), \ili{Benue-Congo} (\citealt{DeWolf1971}), \ili{Mbaic} (\citealt{Bokula1971}, \citealt{Pasch1986}), \ili{Atlantic} \citep{Doneux1975}, Non-\ili{Bantu} \ili{Bantoid} \citep{Hyman1980}, \ili{Edoid} \citep{Elugbe1983}, \ili{Lower Cross} \citep{Connell1987}, and \ili{Guang} (\citealt{Manessy1987}, \citealt{Snider1988}, see \sectref{sec:Gueld:3.4}). In addition, comparative treatments exist on groups that are uncertain members of \ili{Niger-Congo} (see \citealt{Gueldemann2018}) but have typologically similar nominal systems such as \ili{Heibanic} \citep{Schadeberg1981a}, \ili{Talodic} \citep{Schadeberg1981b}, and \ili{Kru} \citep{Marchese1988}.} %
often deal predominantly or exclusively with the system of number inflection rather than gender. We demonstrate and elucidate this mistaken approach with data from \ili{Akan} (\sectref{sec:Gueld:3.2}), \ili{Guang} (\sectref{sec:Gueld:3.3}), and \ili{Ghana-Togo-Mountain} (\sectref{sec:Gueld:3.4}). These geographically close but structurally sufficiently diverse \ili{Niger-Congo} groups in West Africa that are commonly subsumed under the ambiguous concept of \ili{Kwa} (see \citealt{Gueldemann2018} for more discussion on the problematic genealogical classification) represent a convenience choice. The discussion would hardly differ by using other \ili{Niger-Congo} groups and our approach has indeed been applied with the same results to other relevant languages, for example, \ili{Kisi}, \ili{Wolof}, \ili{Fula}, and \ili{Laala} from \ili{Atlantic}, \ili{Miyobe} from \ili{Gur}, \ili{C'lela} and \ili{Gade} from \ili{Benue-Kwa}, and \ili{Mbane} from \ili{Ubangi}.

We will proceed in our analysis according to the framework outlined in \sectref{sec:Gueld:1}. For each language (or proto-language), we first present the agreement class system in the form of a table. This table represents each class by means of exponents in the most important agreement targets, records its behavior regarding number, and, if applicable, gives the default nominal form class. We number the language-specific classes by Arabic numbers either according to the source or our own arbitrary choice; these numbers are preceded by an acronym of the language in order to avoid any facile association with the comparative \ili{Bantu}{\textasciitilde}\ili{Niger-Congo} system. The gender system is established on the basis of the attested mapping of these agreement classes over the relevant number categories and presented in the form of a figure. Agreement classes are represented by one maximally distinct agreement target, similar to previous schemas; genders only receive a label in systems with few distinctions and reasonable clear semantics. Salient sets of transnumeral nouns are marked as usual by circles in the structural schemas; those that cannot be assigned to a paired-class gender in a straightforward way would establish separate single-class genders. Doubtful genders, including ``inquorate'' ones in terms of \citet[170--175]{Corbett1991}, that is, agreement-based sets of nouns whose small size is arguably insufficient to merit incorporation into the grammatical gender system, may be marked by broken lines or circles. This practice is at best approximate, as the available data are insufficient; notably because they usually do not give a full picture about lexical frequencies. In general, the following overviews of gender (and deriflection) patterns are ``structural'' systems that may have to be changed with more comprehensive information about the entire nominal lexicon of a language.

The description of the agreement and gender systems is followed by the investigation of nominal form classes and the resulting deriflection system. Nominal form classes, which are represented by an abstract thematic element in capital letters, are also given in a table that includes their number behavior and representative sample nouns. As far as possible, we take the Ø-marked class (e.g., loans, personal names, kinship terms) into account. The deriflection system is represented in a parallel fashion to the gender system.

Finally, in order to elucidate the relationship between gender and deriflection system, we discuss the discernible correspondences and mismatches between agreement and nominal form classes. These are schematized in figures similar to those given above. In doing so, we try to reflect, if appropriate, the original (alliterative) match between agreement and nominal form class, which is assumed to originate in an older \ili{Niger-Congo} state and whose best proxy at the present is still the relatively coherent Proto-\ili{Bantu} system.

The following discussion involves at several places an assessment of \ili{Niger-Congo} systems regarding a notion of complexity that differs from that focussed on in \sectref{sec:Gueld:2}, which was concerned with systemic organization. In line with Di Gar\-bo's (\citeyear[41, 179]{DiGarbo2014}) first principle of absolute complexity, the characterization here ascertains a system's number of genders (and deriflections). Our evaluation is done against the background of the widely assumed Proto-\ili{Niger-Congo} state, which, when modeled on \ili{Bantu}, would have involved around ten or even more distinctions in both domains, as well as Corbett's (\citeyear{Corbett2013}) typological approach, which assigns the label ``complex'' to gender systems with five or more distinctions. That is, we consider a \ili{Niger-Congo} system as reduced (or no longer as complex), if its inventory has been decreased to a value lower than Corbett's typological threshold for his highest degree of complexity. Note the partly misleading bias toward this typological standard, because a system with five genders like in \ili{Logba} (\ili{Ghana-Togo-Mountain}) is certainly reduced vis-à-vis the proto-state but still counts here as complex.


\subsection{Akan}
\label{sec:Gueld:3.2}

\ili{Akan} is the first linguistic entity to be discussed. It is a large language complex that is the core of a group of closely related languages called \ili{Akanic}, which in turn is classified under the \ili{Potou-Akanic} family \citep{Stewart2002}. \ili{Akan}'s most important dialects in Ghana are \ili{Akuapem}, \ili{Fante} and \ili{Asante} (\citealt[57]{Dolphyne1988}).

The evaluation of the synchronic nominal system of \ili{Akan} undertaken by various authors differs considerably, and none transparently captures the full picture of a system with complex number inflection and, in some dialects, a simple animacy-based gender system. We argue that this is due to a large extent to the problematic philological \ili{Niger-Congo} tradition outlined in \sectref{sec:Gueld:2}.

Earlier authors like \citet{Christaller1875}, \citet{Dolphyne1988}, etc.\ recognize nominal prefixes in \ili{Akan} but do not relate these to a nominal system of the \ili{Niger-Congo} type, thus failing to identify any possible grammatical aspect of ``noun classes''. Following Welmers' (\citeyear[4--5]{Welmers1971}) short notes, \citet{Osam1993} is possibly the first author who analyzes the nominal prefixes as vestiges of a formerly complex ``noun class'' system. Equally important is that the author also discusses agreement phenomena that are arguably remnants of the inherited \ili{Niger-Congo} gender system. Given the focus of this paper, these need to be outlined in more detail.

For one thing, there is number agreement between nouns and a sub-group of attributive adjectives in that the latter receive a prefix in the plural. The nasal prefixes on both the trigger and the target in example (\ref{ex:Gueld:4}b) suggest that there is correspondence in gender and number between the pluralized noun and the modifying adjective.

\protectedex{%
\ea
\label{ex:Gueld:4}
\langinfo{Akan}{}{\citealt[98, 87]{Osam1993}}\\
\begin{xlist}
\ex
\gll \textbf{a}{}-bofra  kakramba\\
     \textbf{\textsc{a}}{}-child  small\\
\glt `small child'
\ex
\gll \textbf{m}{}-bofra  \textbf{n}{}-kakramba\\
     \textbf{\textsc{n}}{}-child  \textbf{\textsc{pl}}{}-small\\
\glt `small children'
\end{xlist}
\z
}%

The author's explanations and additional examples as that in (\ref{ex:Gueld:5}) make it clear, however, that formal prefix identity as in (\ref{ex:Gueld:4}b) is coincidental. Although this is not stated explicitly, the available data suggest that plural marking on adjectives is lexicalized and thus independent of the noun, so that synchronically this phenomenon does not entail gender.

\protectedex{%
\ea
\label{ex:Gueld:5}
\langinfo{Akan}{}{\citealt[98]{Osam1993}}\\
\begin{xlist}
\ex
\gll \textbf{a}{}-kyen  \textbf{n}{}-kakramba\\
     \textbf{\textsc{a}}{}-drum  \textbf{\textsc{pl}}{}-small\\
\glt `small drums'
\ex
\gll \textbf{n}{}-tar  \textbf{e}{}-tuntum\\
     \textbf{\textsc{n}}{}-dress  \textbf{\textsc{p}}\textbf{\textsc{l}}{}-black\\
\glt `black dresses'
\end{xlist}
\z
}%

However, some \ili{Akan} dialects like \ili{Fante} and \ili{Bron} also display verbal subject cross-reference in which the agreement with the relevant nominal referent operates according to the feature of animacy, as shown in (\ref{ex:Gueld:6}) for singular number and systematized in the full picture of \tabref{tab:Gueld:4}.

\ea
\label{ex:Gueld:6}
\langinfo{Akan}{}{\citealt[93]{Osam1993}}
\begin{xlist}
\ex
\gll \textbf{ɔ}{}-bɛ-yera\\
     \textbf{\textsc{1}}\textsc{{}-fut}{}-be.lost\\
\glt `s/he will be lost'
\ex
\gll \textbf{ɛ}{}-bɛ-yera\\
     \textbf{\textsc{3}}\textsc{{}-fut}{}-be.lost\\
\glt `it will be lost'
\end{xlist}
\z

\begin{table}
\begin{tabularx}{.8\textwidth}{lXXX}
\lsptoprule
AGR & Number & Verb prefix & Semantics\\
\midrule
AK1 & SG & \textit{ɔ{}-}, \textit{o{}-}  = \textit{O-} & animate\\
AK2 & PL & \multicolumn{1}{X}{\itshape wɔ{}-, wo- = wO-} & animate\\
AK3 & SG, PL & \textit{ɛ{}-}\textit{, e{}-}  = \textit{E-} & inanimate\\
\lspbottomrule
\end{tabularx}

{\small Note: multiple forms due to vowel harmony.}

\caption{Agreement system of some Akan dialects (based on \citealt{Osam1993})}
\label{tab:Gueld:4}
\end{table}


Despite the data presented, Osam's (\citeyear[99--100, 102]{Osam1993}) major conclusions are that modern \ili{Akan} ``does not have a functioning noun class system'' nor ``a concordial system'', whereby he presumably refers to such elaborate and productive ones as in \ili{Bantu} and similar \ili{Niger-Congo} groups. From a typological perspective, however, \ili{Akan} dialects like \ili{Fante} and \ili{Bron} must be analyzed as having a gender system that is structurally of the parallel type and semantically driven by a distinction of animate vs.\ inanimate nouns, as shown in \figref{fig:Gueld:9}.

\begin{figure}[!htb]
\centering
\begin{tabular}{llp{\llen}l}
\lsptoprule
AGR & SG & & \tknode{0} PL \\
\midrule
\padding
AK1  & \textit{O-} \tknode{9A1} & \\
\padding
AK2 & & & \tknode{9B2} \textit{wO{}-} \\
\padding
AK3 & \textit{E-} \tknode{9A3} & & \tknode{9B3} \textit{E-} \\
\lspbottomrule
\end{tabular}

\tikz[remember picture, overlay] \draw[thick] (9A1.center) -- (9B2.center) %
node[midway,anchor=south] {AN} ;
\tikz[remember picture, overlay] \draw[thick] (9A3.center) -- (9B3.center) %
node[midway,anchor=south] {IAN};

\caption{Gender system of some Akan dialects (based on \citealt{Osam1993})}
\label{fig:Gueld:9}
\end{figure}

\citet{Bodomo2006} is another study dealing with the nominal system of \ili{Akan}. These authors explicitly contradict one of Osam's conclusions in identifying a functional ``noun class system'' on account of nominal affixation, which not only involves prefixes but also suffixes. As just another token of the theoretical and terminological confusion in \ili{Niger-Congo} studies, ``noun classes'' in their terms are sets of nouns showing the same singular/plural affix pairing, that is, classes of number inflection or deriflection in the above, and for that matter common typological, approach. The authors describe a complex system of 9 ``noun classes'' a.k.a.\ deriflections, which partly involve class pairs and subclasses. This is  schematized in \figref{fig:Gueld:10} (restricted to prefixes) and exemplified fully in \tabref{tab:Gueld:5}.


\begin{table}
\begin{tabularx}{\textwidth}{llXXl}
\lsptoprule
{``Noun Class''} & \multicolumn{3}{X}{Example(s)}\\
a.k.a.\ deriflection & Meaning & SG & TN & PL\\
\midrule
1: \textit{V-/N-} & \multicolumn{3}{X}{\bfseries {}$-$}\\
\quad  a: \textit{O-/N-} & `female' & \itshape \`{ɔ}-bàà && \itshape \`{m}-máá\\
\quad  b: \textit{A{}-/N-} & `cloth' & \itshape à-tààd\'{e}  && \itshape ǹ-tààdé\\
\quad  c: \textit{(V)-/N-} & `time' & \itshape \`{ɛ}-br\'{ɛ} && \itshape \`{m}-mr\'{ɛ}\\
2: \textit{Ø{}-/N-} & \multicolumn{3}{X}{\bfseries {}$-$}\\
& `mountain' & \itshape bép\'{ɔ} && \itshape m-mép\'{ɔ}\\
3: \textit{V-/A-} & \multicolumn{3}{X}{\bfseries {}$-$}\\
\quad  a: \textit{O-/A-} & `elephant' & \itshape \`{ɔ}-s\'{ʊ}n\'{ʊ} && \itshape à-s\'{ʊ}n\'{ʊ}\\
\quad  b: \textit{(V-){}-/A-} & `house' & \itshape è-fíé && \itshape à-fìe\\
4: \textit{Ø{}-/}\textit{A-} & \multicolumn{3}{X}{\bfseries {}$-$}\\
& `veranda' & \itshape bámá && \itshape à-bámá\\
5: \textit{(V-)/(A-)\_-nʊm} & \multicolumn{4}{X}{\bfseries Kinship}\\
\quad  a: \textit{V-/A-\_-nʊm} & `father' & {\itshape à-gyá} && {\itshape à-gyá-nʊḿ}\\
& `wife' & \itshape \`{ɔ}-yírí && \itshape à-yírí-n\'{ʊ}ḿ\\
\quad  b: \textit{Ø-/Ø-\_-nʊm} & `aunt' & \itshape sèwáá && \itshape sèwáá-n\'{ʊ}m\\
6: \textit{(O)-\_-ni/A-\_-fʊɔ} & \multicolumn{4}{l}{\bfseries Identity/occupation}\\
\quad  a: \textit{O-\_-ni/A-\_-fʊɔ} & `Christian' & \itshape ò-krǐstò-ní && \itshape à-krǐstò-fú\'{ɔ}\\
\quad  b: \textit{Ø{}-\_-ni/A-\_-}\textit{fʊɔ} & `teacher' & \itshape tíkyà-ní && \itshape a-tíkyà-f\'{ʊ}\'{ɔ}\\
7: \textit{(O)-\_(-ni)/N-\_-fʊɔ} & \multicolumn{3}{l}{\bfseries Identity}\\
\quad  a: \textit{O{}-\_}\textit{{}-ni/N-\_-fʊɔ} & `Muslim' & \itshape ò-k\`{r}èmò-ni && \itshape ǹ-k\`{r}èmò-f\'{ʊ}\'{ɔ}\\
\quad  b: \textit{O-\_{}-}\textit{Ø/N-}\textit{\_-fʊɔ} & `ghost' & \itshape \`{ɔ}-sámáń && \itshape ǹ-sàmàǹ-f\'{ʊ}\'{ɔ}\\
8: \textit{A-} & \multicolumn{3}{l}{\bfseries Deverbal derivation}\\
& `farming' & & \itshape à-d\'{ɔ}\\
9: \textit{N-{\textasciitilde}V-} & \multicolumn{3}{X}{\bfseries Mass}\\
\quad  a: \textit{N-} & `water' & & \itshape n-su\\
\quad  b: \textit{V{}-} & `fire' & & \itshape è-gyá\\
\lspbottomrule
\end{tabularx}

\caption{Deriflection system of Akan (based on \citealt[214--217]{Bodomo2006})}
\label{tab:Gueld:5}
\end{table}


\begin{figure}

\centering
\begin{tabular}{r>{\centering}p{\llen}l}
\lsptoprule
SG \tknode{0} & TN & \tknode{0} PL \\
\midrule
\itshape  O{}- \tknode{10A1} \\
\padding
 & %
 \tikz[remember picture,baseline=(6circ1.base)]\node[circle,inner sep=0pt,draw] (6circ1) {\textit{N-}} ; & \tknode{10B2} N- \\
\padding
  \textit{A-} \tknode{10A3} & %
 \tikz[remember picture,baseline=(6circ1.base)]\node[circle,inner sep=0pt,draw] (6circ1) {\textit{A-}} ; & \tknode{10B3} \textit{A-} \\
\padding
\textit{V-} \tknode{10A4} &  %
 \tikz[remember picture,baseline=(6circ1.base)]\node[circle,inner sep=0pt,draw] (6circ1) {\textit{V-}} ; & \\
\padding
\textit{Ø} \tknode{10A5} & & \tknode{10B5} \textit{Ø}	  \\
\lspbottomrule
\end{tabular}

\tikz[remember picture, overlay] \draw[thick] (10A1.center) -- (10B2.center) ;
\tikz[remember picture, overlay] \draw[thick] (10A1.center) -- (10B3.center) ;
\tikz[remember picture, overlay] \draw[thick] (10A3.center) -- (10B2.center) ;
\tikz[remember picture, overlay] \draw[thick] (10A4.center) -- (10B2.center) ;
\tikz[remember picture, overlay] \draw[thick] (10A4.center) -- (10B3.center) ;
\tikz[remember picture, overlay] \draw[thick] (10A5.center) -- (10B2.center) ;
\tikz[remember picture, overlay] \draw[thick] (10A5.center) -- (10B3.center) ;
\tikz[remember picture, overlay] \draw[dashed] (10A5.center) -- (10B5.center) ;

\caption{Deriflection system of Akan (based on \citealt{Bodomo2006})}
\label{fig:Gueld:10}
\end{figure}

As can be seen in \tabref{tab:Gueld:5}, some of the authors' ``noun classes'' a.k.a.\ deriflections, namely 5, 6 and 7, which all relate to various types of human nouns, involve suffixes in addition to prefixes. Except for the pattern 5b, these suffixes do not create deriflection types that do not already exist on account of the 5 prefix-based nominal form classes. For this reason we only integrate the new Ø/Ø prefix pattern  (see the broken line) in our analysis of the deriflection system in \figref{fig:Gueld:10}. This system involves 8 patterns for count nouns and three for transnumeral nouns. From a structural perspective, it is a complex crossed system because all types of singular noun forms except for the \textit{A}{}-class combine with the two productive plural form classes \textit{N}{}- and \textit{A}{}-.

As discussed above, only some varieties of \ili{Akan} have a parallel system of two genders. Here, the inventory of three agreement classes is so reduced that any correspondence between these and the numerous nominal form classes can only be limited. In fact, the only clear match in both form and meaning exists between AK1 with the exponent \textit{O{}-} and NF \textit{O-}; both mark (predominantly) animate singular nouns. Obviously, this situation diverges considerably from the picture involving ``noun classes'' of \ili{Bantu}-type languages, which involve both agreement and morphological form.

In summary, the \ili{Niger-Congo} tradition clearly fails to capture the structures encountered in \ili{Akan}. Its conceptual framework has even misled descriptive linguists, although the picture as such is not hard to understand as involving a complex, semantically sensitive deriflection system and in some dialects a far simpler agreement-based gender system steered by animacy. As for \citet{Osam1993}, he fails to clearly identify both phenomena in spite of providing most of the relevant empirical data. \citet[206]{Bodomo2006}, in turn, state that ``[a]n overview of \ldots\ nominal morphology shows that the most appropriate criterion that can be used to set up noun classes is number \textendash{} i.e.\ singular and plural \textendash{} categorization'', while ``concord marking \ldots\ is not a very sufficient criterion''. They thus acknowledge that mainstream \ili{Akan} has a system of overt noun classification by means of nominal morphology but fail to observe explicitly that this type of nominal categorization is crucially different from gender in general and the original \ili{Niger-Congo} system in particular (this apart from not dealing with the animacy-based gender system in some dialects).


\subsection{Guang}
\label{sec:Gueld:3.3}

\subsubsection{Introduction}

The second language group we deal with is the \ili{Guang} family, which like \ili{Akanic} belongs to the larger \ili{Potou-Akanic} lineage within \ili{Benue-Kwa}. \ili{Guang} languages are known for their elaborate nominal prefix system but are said to show little in the way of agreement.

\begin{quote}
In all the \ili{Guang} languages, singular and plural of nouns is [sic] indicated by prefixes. None exhibit concord systems, such as are found in many of the Central Togo languages [= \ili{Ghana-Togo-Mountain}, cf.\ \sectref{sec:Gueld:3.4}]. There is, however, at least a trace of number agreement between the noun and some types of adjectives in South \ili{Guang}, \ili{Gichode}, \ili{Krachi}, and some dialects of \ili{Nchumburu} \ldots\  (\citealt[82]{Dolphyne1988})
\end{quote}

Most attempts to define \ili{Guang} ``class'' systems are thus restricted to nominal form classes and disregard concord (and the potentially resulting genders). Our ongoing research aimed at a typologically informed survey of the \ili{Guang} family reveals that the picture, summarized in \tabref{tab:Gueld:6}, is in fact far more diverse.

\begin{table}
\begin{tabularx}{\textwidth}{Qll}
\lsptoprule
Languages & Gender agreement & Number inflection\\
\midrule
Chumburung,\il{Chumburung} \ili{Foodo}, \ili{Gichode}, \ili{Ginyanga}, \ili{Nawuri}
& complex & complex\\
\padding
\ili{Awutu}, \ili{Dwang}, \ili{Gonja}, \ili{Gua}, \ili{Krache}, \ili{Larteh}, \ili{Nkami}, \ili{Nkonya} & reduced & complex\\
\padding
\ili{Cherepon}, \ili{Dompo}, \ili{Nterato}, \ili{Kplang}, \ili{Nchumbulu}, \ili{Tchumbuli} & insufficient information & insufficient information\\
\lspbottomrule
\end{tabularx}

\caption{Overview over gender systems in Guang}
\label{tab:Gueld:6}
\end{table}

\begin{figure}
\begin{tabular}{lrp{\llen}l}
\lsptoprule
 &  SG \tknode{0} &&  \tknode{0}  PL \\
\midrule
GO1  & \textit{e-} \tknode{11A1} \\
\padding
GO2 & & & \tknode{11B2}  \textit{bo-} \\
\padding
GO3 &  \textit{ki-} \tknode{11A3} \\
\padding
GO4 & & & \tknode{11B4} \textit{a-} \\
\lspbottomrule
\end{tabular}

\tikz[remember picture, overlay] \draw[thick] (11A1.center) -- (11B2.center) %
node[midway,anchor=south] {AN} ;
\tikz[remember picture, overlay] \draw[thick] (11A3.center) -- (11B4.center) %
node[midway,anchor=south] {IAN} ;

\caption{Gender system of Gonja (based on \citealt{Painter1970})}
\label{fig:Gueld:11}
\end{figure}

\tabref{tab:Gueld:6} shows that gender agreement is indeed strongly reduced in several \ili{Guang} languages, largely to an animacy differentiation illustrated in \figref{fig:Gueld:11} with the case of \ili{Gonja}, which is parallel to the situation in the relevant \ili{Akan} dialects treated in \sectref{sec:Gueld:3.2}. However, several languages still possess quite complex gender systems, for example, \ili{Chumburung}, which we illustrate in \sectref{sec:Gueld:3.3.2}.



\subsubsection{Chumburung}
\label{sec:Gueld:3.3.2}

\ili{Chumburung}, according to the description by \citet[266ff]{Hansford1990}, is a \ili{Guang} language with a more canonical nominal system. Its agreement system concerns both the noun phrase in the form of quantifier agreement, as in (\ref{ex:Gueld:7}), and a variety of other morpho-syntactic contexts with anaphoric pronominal agreement, for example, the conjoined noun phrase in (\ref{ex:Gueld:8}). Other targets of the second type of concord are pronominal forms for `certain', `one of', `each, any', `which' and demonstratives \citep[184]{Hansford1990}; when these are used as modifiers within a noun phrase, they do not agree with their head. A similar situation holds for verbal subject and object cross-reference and relative clauses, as in (\ref{ex:Gueld:9}) \citep[450]{Hansford1990}. The full system of seven agreement classes is provided in \tabref{tab:Gueld:7}.

\ea
\label{ex:Gueld:7}
\langinfo{Chumburung}{}{\citealt[270, 201]{Hansford1990}}
\begin{xlist}
\ex
\gll \textbf{à}{}-wààgyà   d\`{ɩ}dáá   \textbf{á}{}-ny\'{ɔ}   m\`{ɔ}\\
     \textbf{\textsc{a}}\textsc{{}-}cloth(\textbf{6})  old \textbf{\textsc{6}}\textsc{{}-}two \textsc{dem}\\
\glt `these two old cloths'
\ex
\gll \textbf{\`{ɩ}}{}-wór\'{ɩ}    \textbf{\'{ɩ}}{}-ny\'{ɔ}   \textbf{\'{ɩ}}{}-ny\`{ɔ}\\
     \textbf{\textsc{i}}{}-book(\textbf{4})  \textbf{4}{}-two  \textbf{4}{}-two\\
\glt `pairs of two books' (distributive)
\end{xlist}
\z

\ea
\label{ex:Gueld:8}
\langinfo{Chumburung}{}{\citealt[266]{Hansford1990}}\\
\gll wààgyà   gyígyíí   nà   \textbf{ó}{}-pípéé\\
     Ø:cloth(\textbf{1})  black   and \textbf{\textsc{1}}{}-red\\
\glt `a black and red cloth [lit.: a black cloth and a red one]'
\z

\ea
\label{ex:Gueld:9}
\langinfo{Chumburung}{}{\citealt[267, 451]{Hansford1990}}\\
\gll k\`{ɩ}-b\'{ɩ}gyá  n\'{ɩ}   kí  í  gyí  sí  ó\\
     \textbf{\textsc{kv}}{}-side(\textbf{3})  \textsc{rel}  \textbf{3}  \textsc{ipfv}  eat  on  \textsc{rel}\\
\glt `... the side that will win'
\z

\begin{table}
\begin{tabularx}{\textwidth}{lXXXXX}
\lsptoprule
AGR & Number & SBJ & OBJ & Pronominal & NF default\\
\midrule
CH1 & TN, SG & \itshape ɔ{}-/o{}- & \itshape ˋ{}- & \itshape ɔ{}-/o{}- & $-$\\
CH2 & TN, PL & \itshape bʊ{}-/ba{}- & \itshape bá{}- & \itshape bʊ{}-/ba{}- & $-$\\
CH3 & TN, SG & \itshape kV{}- & \itshape k\'{ɩ}- & \itshape kʊ{}-/kɩ- & \itshape KV{}-\\
CH4 & TN, PL & \itshape i{}-/ɩ{}- & \itshape \'{ɩ}- & \itshape ɩ{}- & \itshape I{}-\\
CH5 & TN, SG & \itshape ka- & \itshape ká{}- & \itshape ka{}- & \itshape KA{}-\\
CH6 & TN, PL & \itshape a{}- & \itshape á{}- & \itshape a{}- & \itshape A{}-\\
CH7 & TN, PL & \itshape N- & \itshape ḿ{}- & \itshape ŋ{}-/m{}- & \itshape N{}-\\
\lspbottomrule
\end{tabularx}

\caption{Agreement class system of Chumburung (based on \citealt{Hansford1990})}
\label{tab:Gueld:7}
\end{table}

While Hansford does not give a schematic overview of the gender system, his description of the mapping of agreement classes over number categories allows one to establish the system in \figref{fig:Gueld:12} with six paired and at least four single-class genders.


\begin{figure}

\begin{tabular}{lrp{\llen}l}
\lsptoprule
  & SG \tknode{0} & TN & \tknode{0} PL \\
\midrule
CH2 & &   %
 \tikz[remember picture,baseline=(6circ1.base)]\node[circle,inner sep=0pt,draw,dashed] (6circ1) {\textit{bV-}} ; & \tknode{12B1} \textit{bV-} \\
\padding
CH1 &  \textit{O-} \tknode{12A2} & %
 \tikz[remember picture,baseline=(6circ1.base)]\node[circle,inner sep=1pt,draw] (6circ1) {\textit{O-}} ; & \\
\padding
CH4 &  & %
 \tikz[remember picture,baseline=(6circ1.base)]\node[circle,inner sep=2pt,draw] (6circ1) {\textit{I-}} ; & \tknode{12B3} \textit{I-} \\
\padding
CH3 &  \textit{kV-} \tknode{12A4} &  %
 \tikz[remember picture,baseline=(6circ1.base)]\node[circle,inner sep=0pt,draw] (6circ1) {\textit{kV{}-}} ; & \\
\padding
CH6 & & %
 \tikz[remember picture,baseline=(6circ1.base)]\node[circle,inner sep=2pt,draw] (6circ1) {\textit{a-}} ; & \tknode{12B5} \textit{a-} \\
\padding
CH5 &  \textit{ka-} \tknode{12A6} &  %
 \tikz[remember picture,baseline=(6circ1.base)]\node[circle,inner sep=0pt,draw,dashed] (6circ1) {\textit{ka{}-}} ; & \\
\padding
CH7 & & %
 \tikz[remember picture,baseline=(6circ1.base)]\node[circle,inner sep=1pt,draw,dashed] (6circ1) {\textit{N-}} ; & \tknode{12B7} \textit{N-} \\
\lspbottomrule
\end{tabular}

% lines connecting the nodes
\tikz[remember picture, overlay] \draw[thick] (12A2.center) -- (12B1.center) ;
\tikz[remember picture, overlay] \draw[thick] (12A2.center) -- (12B3.center) ;
\tikz[remember picture, overlay] \draw[thick] (12A2.center) -- (12B5.center) ;
\tikz[remember picture, overlay] \draw[thick] (12A4.center) -- (12B3.center) ;
\tikz[remember picture, overlay] \draw[thick] (12A4.center) -- (12B5.center) ;
\tikz[remember picture, overlay] \draw[thick] (12A6.center) -- (12B7.center) ;

\caption{Gender system of Chumburung (based on \citealt{Hansford1990})}
\label{fig:Gueld:12}
\end{figure}

When compared to the widely assumed \ili{Niger-Congo} proto-type, this complex crossed system is in several respects remarkable, which is largely due to the nature of agreement classes in \ili{Chumburung}. For one thing, all agreement classes occur with transnumeral nouns, so that at least some are not dedicated to a single number feature. For CH2, CH5, and CH7, one may avoid positing separate single-class genders by arguing that these nouns represent special transnumeral cases, namely singularia tantum or pluralia tantum that can be associated uniquely with particular paired genders, namely CH1/CH2 and CH5/CH7. However, this solution is not possible for similar nouns in the remaining four agreement classes, because it would be an ad-hoc decision at this stage to assign these nouns to one of the two or even three paired genders the relevant class partakes in. The last fact is another non-canonical finding in the present philological context, namely that only the three aforementioned classes, CH2, CH5, and CH7, have a unique counterpart in their opposite number feature and are thus dedicated to a paired gender. Overall, \ili{Chumburung} agreement classes only poorly meet the \ili{Niger-Congo} expectation that ``noun classes'' only have one number and one gender value.

The system of seven nominal form classes described for \ili{Chumburung}, including the group of prefixless nouns, are exemplified in \tabref{tab:Gueld:8}, while \figref{fig:Gueld:13} displays their mapping over number categories in the deriflection system.

\begin{table}
\begin{tabularx}{\textwidth}{lllX}
\lsptoprule

NF & Form &  & Examples\\
\midrule
\itshape Ø & {}$-$ & {SG} & {\textit{dáá} `elder brother', \textit{b\'{ʊ}r\'{ɩ}} `voice'} \\
&& TN & \textit{gyàbw\'{ɩ}\'{ɩ}} `honey', \textit{sáŋ} `time'\\
\itshape O- & \itshape o{}-/ɔ{}- & {SG} & {\textit{ó{}-}\textit{wúrè} `chief', \textit{\`{ɔ}{}-}\textit{d\`{ɔ}\'{ɔ}} `fishing net'}\\
&& TN & \textit{\`{ɔ}{}-}\textit{t\`{ʊ}r\'{ɩ}} `morning star'\\
\itshape I- & \itshape i{}-/ɩ{}- & {TN} & {\textit{í{}-}\textit{bírísí} `evil spirit(s)'}\\
&& PL & \textit{\'{ɩ}-}\textit{b\'{ʊ}r\'{ɩ}} `voices', \textit{\`{ɩ}-}\textit{d\`{ɔ}\'{ɔ}} `fishing net', \textit{\`{ɩ}-}\textit{s\'{ɩ}b\'{ɔ}} `ears', \textit{\'{ɩ}-}\textit{bá} `coming (PL)'\\
\itshape KV- & {\itshape ki{}-/kɩ{}-/} & {SG} & {\textit{kì{}-}\textit{yéé} `meat', \textit{k\`{ɩ}-}\textit{s\'{ɩ}b\'{ɔ}} `ear', \textit{k\'{ɩ}-}\textit{bá} `coming'} \\
& \itshape ku{}-/kʊ{}- & TN & \textit{kì{}-}\textit{tìrì} `poverty'\\
\itshape A- & \itshape a{}- & {TN} & \textit{à{}-}\textit{bán\'{ɩ}} `government' \\
&& PL & \textit{á{}-}\textit{dáá} `elder brothers', \textit{á{}-}\textit{wúrè} `chiefs', \textit{à{}-}\textit{yéé} `meats'\\
\itshape KA- & \itshape ka{}- & {SG} & {\textit{ká{}-}\textit{mé} `stomach'}\\
&& TN & \textit{ká{}-}\textit{nyíté} `patience', \textit{ká{}-}\textit{ky\'{ɩ}nà} `life'\\
\itshape N- & {\itshape n{}-/m{}-/} & {TN} & {\textit{\`{m}}\textit{{}-}\textit{b\`{ʊ}gyà} `blood', \textit{\`{m}}\textit{{}-}\textit{bèráá} `law'} \\
& \itshape ɲ{}-/ŋ{}- &PL & \textit{ḿ{}-}\textit{mé} `stomachs'\\
\lspbottomrule
\end{tabularx}

\caption{Nominal form class system of Chumburung}
\label{tab:Gueld:8}
\end{table}

\begin{figure}


\begin{tabular}{r>{\centering}p{\llen}l}
\lsptoprule
SG \tknode{0} & TN & \tknode{0} PL \\
\midrule
  \textit{Ø} \tknode{13A1} &   %
 \tikz[remember picture,baseline=(6circ1.base)]\node[circle,inner sep=2pt,draw] (6circ1) {\textit{Ø}} ; & \\
\padding
  \textit{O-} \tknode{13A2} &   %
 \tikz[remember picture,baseline=(6circ1.base)]\node[circle,inner sep=1pt,draw] (6circ1) {\textit{O-}} ; & \\
\padding
 &     %
 \tikz[remember picture,baseline=(6circ1.base)]\node[circle,inner sep=2pt,draw] (6circ1) {\textit{I-}} ; & \tknode{13B3} \textit{I-} \\
\padding
  \textit{KV-} \tknode{13A4} &  %
 \tikz[remember picture,baseline=(6circ1.base)]\node[circle,inner sep=0pt,draw] (6circ1) {\textit{KV-}} ;  & \\
\padding
 &     %
 \tikz[remember picture,baseline=(6circ1.base)]\node[circle,inner sep=1pt,draw] (6circ1) {\textit{A-}} ;  & \tknode{13B5} \textit{A-} \\
\padding
  \textit{KA-} \tknode{13A6} &  %
 \tikz[remember picture,baseline=(6circ1.base)]\node[circle,inner sep=0pt,draw] (6circ1) {\textit{KA-}} ; & \\
\padding
 &    %
 \tikz[remember picture,baseline=(6circ1.base)]\node[circle,inner sep=1pt,draw] (6circ1) {\textit{N-}} ;  & \tknode{13B7} \textit{N-} \\
\lspbottomrule
\end{tabular}

% lines connecting the nodes
\tikz[remember picture, overlay] \draw[thick] (13A1.center) -- (13B3.center) ;
\tikz[remember picture, overlay] \draw[thick] (13A1.center) -- (13B5.center) ;
\tikz[remember picture, overlay] \draw[thick] (13A2.center) -- (13B3.center) ;
\tikz[remember picture, overlay] \draw[thick] (13A2.center) -- (13B5.center) ;
\tikz[remember picture, overlay] \draw[thick] (13A4.center) -- (13B3.center) ;
\tikz[remember picture, overlay] \draw[thick] (13A4.center) -- (13B5.center) ;
\tikz[remember picture, overlay] \draw[thick] (13A6.center) -- (13B7.center) ;

\caption{Deriflection system of Chumburung (based on \citealt[156--161]{Hansford1990})}
\label{fig:Gueld:13}
\end{figure}

The deriflection system, presented by Hansford with example nouns, comprises 7 types of singular-plural pairings, and all nominal form classes also occur with transnumeral nouns. Although this crossed system is overall similar in structure and size to the gender system in \figref{fig:Gueld:12} with 6 paired and 4 single-class patterns, it is more complex than the latter on account of having 7 paired deriflections.

\begin{figure}

\begin{tabular}{lrp{\llen}ll}
\lsptoprule
&  AGR \tknode{0} &  & \tknode{0} NF &  Number\\
\midrule
&  X \tknode{0} &  & \tknode{14B1} \textit{Ø}  &  TN, SG \\
CH1 &  \textit{O-} \tknode{14A2} & & \tknode{14B2} \textit{O-}   & TN, SG\\
CH3 &  \textit{kV-} \tknode{14A3} & & \tknode{14B3} \textit{kV-}  &  TN, SG\\
CH5  & \textit{ka-} \tknode{14A4} & & \tknode{14B4} \textit{kA-} &  TN, SG\\
CH4 &  \textit{I-} \tknode{14A5} & & \tknode{14B5} \textit{I-}  & TN, PL\\
CH6 &  \textit{a-} \tknode{14A6} & & \tknode{14B6} \textit{A-}  & TN, PL\\
CH7 &  \textit{N-} \tknode{14A7} &  & \tknode{14B7} \textit{N-}  & TN, PL\\
CH2 &  \textit{bV-} \tknode{14A8} & & \tknode{0} X &  PL\\
\lspbottomrule
\end{tabular}

% lines connecting the nodes
\tikz[remember picture, overlay] \draw[thick] (14A2.center) -- (14B1.center) ;
\tikz[remember picture, overlay] \draw[thick] (14A2.center) -- (14B2.center) ;
\tikz[remember picture, overlay] \draw[thick] (14A2.center) -- (14B3.center) ;
\tikz[remember picture, overlay] \draw[thick] (14A2.center) -- (14B4.center) ;
\tikz[remember picture, overlay] \draw[thick] (14A3.center) -- (14B3.center) ;
\tikz[remember picture, overlay] \draw[thick] (14A4.center) -- (14B4.center) ;
\tikz[remember picture, overlay] \draw[thick] (14A5.center) -- (14B5.center) ;
\tikz[remember picture, overlay] \draw[thick] (14A6.center) -- (14B6.center) ;
\tikz[remember picture, overlay] \draw[thick] (14A7.center) -- (14B7.center) ;
\tikz[remember picture, overlay] \draw[thick] (14A8.center) -- (14B6.center) ;
\tikz[remember picture, overlay] \draw[thick] (14A8.center) -- (14B7.center) ;

{\small Note: X = no independent class counterpart in the other class type.}


\caption{Mapping of agreement and nominal form classes in Chumburung (based on \citealt[156--161]{Hansford1990})}
\label{fig:Gueld:14}
\end{figure}

The concrete differences between the systems of genders and deriflections are due to a number of mismatches between agreement and nominal form classes, as shown in \figref{fig:Gueld:14}. These exist in spite of the still considerable formal correspondence between the two sets that is expected from the inherited one-to-one alliterative mapping. A predictable mismatch is the existence of the Ø{}-nominal form class that has no independent match in the agreement system. Another difference arises from the loss of the reconstructable nominal form class counterpart of CH2; the relevant nouns are found today in two other nominal form classes in \textit{A-} (a potential reflex of the expected prefix *ba- through loss of the initial consonant) and \textit{N-}. Both points are related to another important phenomenon also found in other \ili{Guang} languages; namely that the semantic criterion of animacy overrides the inherited, more elaborate formal gender assignment. That is, all human nouns irrespective of their form class prompt agreement according to singular CH1 and plural CH2 (the nominal form class in \textit{I-} is the only one without human nouns). The power of this semantic criterion can also be seen when analyzing the agreement triggered by proper nouns: all singulars agree according to CH1; all plurals referring to humans, personified animals and supernatural beings belong to CH2 while the rest follows CH4 or CH6 \citep[166]{Hansford1990}.

\subsubsection{Proto-Guang}

The ``noun class'' system of the \ili{Guang} family has been subject to historical-comparative reconstruction independently but roughly at the same time by \citet{Manessy1987} and \citet{Snider1988}. We discuss their results in the following before the background and in accordance with the presentation of our \ili{Chumburung} analysis in the Figures \ref{fig:Gueld:12} and \ref{fig:Gueld:13}.

As already suggested by Manessy's term ``système classificatoire'' (instead of ``gender system''), this author takes both nominal form classes and agreement in the pronominal system of some languages into account, although the latter was at his time only available for two languages, namely \ili{Nkonya} (\citealt{Westermann1922}, \citealt{Reineke1966}) and \ili{Gonja} \citep{Painter1970}. For all other languages, he merely had access to wordlists that only rarely contained information on agreement. A yet greater problem of his analysis is that he follows the philological approach in explicitly (ibid.:~42) conflating noun form and agreement classes into a single \ili{Guang} reconstruction, given in the left schema of \figref{fig:Gueld:15}.

\citet{Snider1988} deduced the ``noun class'' system of Proto-\ili{Guang} by looking at the noun prefixes of nine of the 18 attested family members without mentioning at all possible agreement forms. He observed a major difference between Northern and Southern \ili{Guang}, the former being richer in nominal form classes, and concluded (ibid.:~138):

\begin{quote}
\ldots\ that proto-\ili{Guang} had a system at least as complex as the most complex present day \ili{Guang} languages and that the southern \ili{Guang} languages represent a collapsing of classes.
\end{quote}

The system he established for Proto-\ili{Guang} is displayed in the middle of \figref{fig:Gueld:15}; we have added the three single-class patterns mentioned by him when discussing the individual nominal form classes.

\begin{sidewaysfigure}

\begin{tabular}{ %
r>{\centering}p{\llen}l l %
r>{\centering}p{\llen}l l %
r>{\centering}p{\llen}l l %
}%
\multicolumn{4}{l}{\small \bfseries Proto-\ili{Guang} of \citet[42]{Manessy1987}} & %
\multicolumn{4}{l}{\small \bfseries Proto-\ili{Guang} of \citet[138]{Snider1988}} & %
\multicolumn{4}{l}{{\small \bfseries \ili{Chumburung} gender system}} \\
\cmidrule{1-3}\cmidrule{5-7}\cmidrule{9-12}
\addlinespace[-\aboverulesep]
\cmidrule[\heavyrulewidth]{1-3}\cmidrule[\heavyrulewidth]{5-7}\cmidrule[\heavyrulewidth]{9-12}
SG \tknode{0} & TN & \tknode{0} PL  & 	& SG \tknode{0} & TN & \tknode{0} PL  & 		& SG \tknode{0} & TN & \tknode{0} PL  & \\
\cmidrule{1-3}\cmidrule{5-7}\cmidrule{9-12}
&&&									& *\O \tknode{15C1} &		& & & &  \\
\padding
 & & \tknode{15B2} *bV- &					& & & \tknode{15D2} *ba- &	& & & \tknode{15F2} \textit{bV-} &  CH2 \\
 \padding
 *O- \tknode{15A3} & & &					& *O- \tknode{15C3} & & &	&  \textit{O-} \tknode{15E3} & %
 \tikz[remember picture,baseline=(15circ6.base)]\node[circle,inner sep=1pt,draw] (15circ6) {\textit{O-}} ; %
  & & CH1	\\
 \padding
 *E- \tknode{15A4} & & \tknode{15B4} *E- &  	& &%
 \tikz[remember picture,baseline=(15circ3.base)]\node[circle,inner sep=1pt,draw] (15circ3) {\textit{*I-}} ; %
  & \tknode{15D4} *I- &			& & %
 \tikz[remember picture,baseline=(15circ7.base)]\node[circle,inner sep=2pt,draw] (15circ7) {\textit{I-}} ; %
  & \tknode{15F4} \textit{I-} & CH4 \\
 \padding
 *kI- \tknode{15A5} & & &  				& *kI- \tknode{15C5} & & &		& \textit{kV-} \tknode{15E5} & %
 \tikz[remember picture,baseline=(15circ8.base)]\node[circle,inner sep=0pt,draw] (15circ8) {\textit{kV-}} ; %
  & & CH3 \\
 \padding
 *A- \tknode{15A6} &%
 \tikz[remember picture,baseline=(15circ1.base)]\node[circle,inner sep=0pt,draw] (15circ1) {\textit{*A-}} ; %
  & \tknode{15B6} *A- 	&  	& & %
 \tikz[remember picture,baseline=(15circ4.base)]\node[circle,inner sep=0pt,draw] (15circ4) {\textit{*A-}} ; %
  & \tknode{15D6} *A- &		& & %
 \tikz[remember picture,baseline=(15circ9.base)]\node[circle,inner sep=2pt,draw] (15circ9) {\textit{a-}} ; %
  & \tknode{15F6} \textit{a-} & CH6 \\
 \padding
 *kA- \tknode{15A7} & & &  				& *kA- \tknode{15C7} & & &		& \textit{ka-} \tknode{15E7} & & & CH5 \\
 \padding
  &%
 \tikz[remember picture,baseline=(15circ2.base)]\node[circle,inner sep=0pt,draw,dashed] (15circ2) {\textit{*N-}} ; %
  & \tknode{15B8} *N- &  				& & %
 \tikz[remember picture,baseline=(15circ5.base)]\node[circle,inner sep=0pt,draw,dashed] (15circ5) {\textit{*N-}} ; %
  & \tknode{15D8} *N- &		& & & \tknode{15F8} \textit{N-} & CH7 \\
 \padding
 *dI- \tknode{15A9} & & &  				& & & &					& & & &  \\
 \padding
 *ke- \tknode{15A10} & & &  				& & & &					& & & &  \\
\cmidrule[\heavyrulewidth]{1-3}\cmidrule[\heavyrulewidth]{5-7}\cmidrule[\heavyrulewidth]{9-12}
\end{tabular}

% lines connecting the nodes
% leftmost panel
\tikz[remember picture, overlay] \draw[thick] (15A3.center) -- (15B2.center) ;
\tikz[remember picture, overlay] \draw[thick] (15A3.center) -- (15B4.center) ;
\tikz[remember picture, overlay] \draw[thick] (15A3.center) -- (15B6.center) ;
\tikz[remember picture, overlay] \draw[thick] (15A4.center) -- (15B2.center) ;
\tikz[remember picture, overlay] \draw[thick] (15A4.center) -- (15B4.center) ;
\tikz[remember picture, overlay] \draw[thick] (15A5.center) -- (15B6.center) ;
\tikz[remember picture, overlay] \draw[thick] (15A6.center) -- (15B8.center) ;
\tikz[remember picture, overlay] \draw[thick] (15A7.center) -- (15B8.center) ;
\tikz[remember picture, overlay] \draw[thick] (15A9.center) -- (15B6.center) ;
\tikz[remember picture, overlay] \draw[thick] (15A10.center) -- (15B6.center) ;
\tikz[remember picture, overlay] \draw[thick] (15A10.center) -- (15B8.center) ;
% center panel
\tikz[remember picture, overlay] \draw[thick] (15C1.center) -- (15D4.center) ;
\tikz[remember picture, overlay] \draw[thick] (15C1.center) -- (15D6.center) ;
\tikz[remember picture, overlay] \draw[thick] (15C3.center) -- (15D2.center) ;
\tikz[remember picture, overlay] \draw[thick] (15C3.center) -- (15D8.center) ;
\tikz[remember picture, overlay] \draw[thick] (15C5.center) -- (15D4.center) ;
\tikz[remember picture, overlay] \draw[thick] (15C5.center) -- (15D6.center) ;
\tikz[remember picture, overlay] \draw[thick] (15C7.center) -- (15D8.center) ;
%rightmost panel
\tikz[remember picture, overlay] \draw[thick] (15E3.center) -- (15F2.center) ;
\tikz[remember picture, overlay] \draw[thick] (15E3.center) -- (15F4.center) ;
\tikz[remember picture, overlay] \draw[thick] (15E3.center) -- (15F6.center) ;
\tikz[remember picture, overlay] \draw[thick] (15E5.center) -- (15F4.center) ;
\tikz[remember picture, overlay] \draw[thick] (15E5.center) -- (15F6.center) ;
\tikz[remember picture, overlay] \draw[thick] (15E7.center) -- (15F8.center) ;

\caption{Noun classification systems of Proto-Guang and Chumburung}
\label{fig:Gueld:15}

\end{sidewaysfigure}

We briefly show in the following that both Proto-\ili{Guang} systems in \figref{fig:Gueld:15} are biased toward the situation in other West African class languages and/or the authors' assumptions about Proto-\ili{Niger-Congo}. Moreover, nominal form classes are the primary source for the analysis, even though agreement classes are taken into account to some extent. This bias and the conflation of all data into a single ``noun class'' system causes serious errors in their reconstruction results, so that they not only differ from each other but also both fail to yield a likely approximation to either the gender or the deriflection system of Proto-\ili{Guang}. The last point is evident from an inspection of the gender system in such modern languages as \ili{Chumburung} (repeated from \sectref{sec:Gueld:3.3.2} on the right side of \figref{fig:Gueld:15}).

The following can be observed regarding the (non)overlap between the two proto-systems. Manessy and Snider only agree on the three class pairs *kI-/A-, *ka-/N-, and *O-/bV-, all of which are also attested as genders in modern \ili{Chumburung}. Both \citet[27]{Manessy1987} and \citet[141]{Snider1988} reconstruct a plural prefix *bV- or *ba-, although they observe its exceptional status in that it only occurs as such in \ili{Gonja}; they claim it to belong to the proto-language because of its wide distribution in \ili{Niger-Congo} as well as its attestation as an agreement form for third-person plural (animate) in a range of \ili{Guang} languages.

Snider reconstructs a Ø-class but merely as part of the number inflection patterns *Ø/I- and *Ø/A- without noting that these reflect agreement-based genders that in the singular involve the old \ili{Niger-Congo} class *1, as can be observed in modern \ili{Chumburung} (his additional nominal prefix pairing *O-/N- is so far not attested as involving a separate gender). Although \citet{Manessy1987} appears to capture well the behavior of the old \ili{Niger-Congo} class *1, he does not posit a Ø-class for nouns. According to him, most prefixless nouns in one language show a \textit{kV}-prefix in another language, concluding that in the proto-language such nouns did not form a ``noun class'' \citep[20]{Manessy1987}; in our view this seems to be adequate with respect to agreement while not being the case for noun forms.

Another major divergence between the two reconstructions concerns all forms in \textit{kV-}. \citet[147--148]{Snider1988} reconstructs the prefixes *kA- and *kI- (representing \textit{ki-}, \textit{kɩ{}-}, \textit{ku-}, and \textit{kʊ-}). \citet[12]{Manessy1987} additionally posits *ke- (representing \textit{ke-}, \textit{kɛ-}, \textit{ko-}, and \textit{kɔ-}), assumed by Snider to be due to phonetically inaccurate data. All \ili{Guang} languages only have a binary distinction of \textit{kV}{}-forms in the agreement system but, due to the complexity of the vowel phonology, dispose of a wider range of relevant forms on nouns. Thus, Manessy's two class pairs based on a third *ke- do not seem to be warranted, because they are only attested in \ili{Gichode} (and probably \ili{Ginyanga}) as genders and deriflections in opposition to a \textit{gI}{}-class, so that putative *ke- may merely be a reflex of *kA-.

Manessy's Proto-\ili{Guang} reconstruction is problematic in several other respects. His pair *E-/bV- only exists as a gender and deriflection in \ili{Gonja} (see \figref{fig:Gueld:11}). He also posits a singular prefix *dI- (paired with plural *A-), although it is only attested in such a gender in \ili{Foodo} (which was not part of Snider's language sample). Manessy includes *dI- for Proto-\ili{Guang}, because there are nouns with a purported \textit{lV}{}-prefix in some other \ili{Guang} languages and the prefix is ``fort commune dans les langues à classes d'Afrique occidentale et que pour cette raison nous tenons pour ancienne [very common in the class languages of West Africa and for that reason we consider to be old]'' \citep[41]{Manessy1987}. His reconstructions *E-/E- and *A-/N- are not attested genders in any language and are also questionable as reconstructable deriflections. Finally, he fails to identify the pairing *kI-/E-.

A general conclusion about Manessy's and Snider's historical-comparative work on \ili{Guang} is that their philological approach generates reconstructions that reflect the agreement and resulting gender system inadequately. In particular, their focus on nominal form classes seems to result in proto-systems that are overly complex for the domain of genders.


\subsection{Ghana-Togo-Mountain}
\label{sec:Gueld:3.4}

\subsubsection{Introduction}

The \ili{Ghana-Togo-Mountain} languages (formerly known as Togo Remnant) are spoken in Ghana, Togo and Benin. Besides the relevant \ili{Guang} languages, they are well known within \ili{Kwa} for class systems that retain both rich agreement and noun prefix patterns. Historical comparisons across these languages are complicated by their unresolved genealogical classification in that they are viewed either as a single lineage according to the traditional view or as forming at least two families according to more recent research (cf.\ \citealt{Blench2009} for a relevant discussion). \tabref{tab:Gueld:9} shows the subclassification of the languages after \citet{Hammarstroem2018} and the profile of their noun categorization systems according to \citet{Gueldemann2016}.

\begin{table}[htb]
\begin{tabularx}{\textwidth}{lQXX}
\lsptoprule
& Language(s) & Gender agreement & Number inflection\\
\midrule
\multirow{3}{*}{\rotatebox[origin=r]{90}{\ili{Na-Togo}}} & Anii,\il{Anii} \ili{Adele}, \ili{Lelemi}, \ili{Siwu}, \ili{Sekpele}, \ili{Selee}, \ili{Logba} & complex & complex\\
\padding
& Boro\il{Boro} (†) & no information & no information\\
\padding
\multirow{3}{*}{\rotatebox[origin=r]{90}{\ili{Ka-Togo}}} & Avatime,\il{Avatime} \ili{Nyangbo}, \ili{Tafi}, \ili{Tuwuli}, Akebu & complex & complex\\
\padding
& Igo,\il{Igo} \ili{Animere} & reduced & complex\\
\padding
& Ikposo\il{Ikposo} & absent & absent\\
\lspbottomrule
\end{tabularx}

\caption{Inventory, classification and noun categorization profile of Ghana-Togo-Mountain languages}
\label{tab:Gueld:9}
\end{table}

As with \ili{Guang} in \sectref{sec:Gueld:3.3}, we will first present the synchronic gender system of one modern \ili{Ghana-Togo-Mountain} language before turning to historical approaches to the entire group.

\subsubsection{Lelemi}

We have chosen the \ili{Na-Togo} language \ili{Lelemi} (as described by \citealt{Allan1973} with a focus on the \ili{Baglo} variety) as an example, because it possesses a complex gender system and it has also been included in the typological gender survey by \citet{Corbett1991}.

\ili{Lelemi} nouns prompt agreement on a variety of targets such as determiners, as in (\ref{ex:Gueld:11}), ordinal numerals, the cardinal numeral `one', participles, as in (\ref{ex:Gueld:10}), and relative pronouns, as well as anaphoric subject cross-reference, as in (\ref{ex:Gueld:11}). As opposed to \citet[115]{Heine1968}, Allan's data do not provide evidence for adjectival agreement.


\ea\label{ex:Gueld:10}
\langinfo{Lelemi}{}{\citealt[178]{Allan1973}}\\
\gll \textbf{k\`{ɔ}}{}-làkpi  \textbf{k\`{ɔ}}{}-dun-di\\
     \textbf{\textsc{ko}}\textsc{{}-}snake(\textbf{6})  \textbf{\textsc{6}}\textsc{{}-}kill-\textsc{part}\\
\glt `a killed snake'
\z

\ea
\label{ex:Gueld:11}
\langinfo{Lelemi}{}{\citealt[240--241]{Allan1973}}\\

\begin{tabular}{lllll}
\itshape \textbf{\`{ɔ}}{}-nànà & \itshape \textbf{\'{ɔ}}{}-m\`{ɛ} & \itshape \textbf{\`{ɔ}}{}-dìa & `this man\\
\itshape \textbf{bà}{}-nànà & \itshape  \textbf{bá}{}-m\`{ɛ} & \itshape \textbf{bà}{}-dìa & `these men\\
\itshape \textbf{lɛ}{}-tɔ & \itshape  \textbf{l\'{ɛ}}{}-m\`{ɛ} & \itshape \textbf{l\`{ɛ}}{}-dìa & `these houses\\
\itshape \textbf{a}{}-nimì & \itshape \textbf{á}{}-m\`{ɛ} & \itshape \textbf{\`{a}}{}-dìa & `this rice\\
\itshape \textbf{kɔ}{}-di & \itshape \textbf{k\'{ɔ}}{}-m\`{ɛ} & \itshape \textbf{k\`{ɔ}}{}-dìa & `this cloth\\
\itshape \textbf{ke}{}-mo  & \itshape \textbf{ká}{}-m\`{ɛ} & \itshape \textbf{k\`{a}}{}-dìa & `this farm\\
\itshape \textbf{n}{}-tɛ & \itshape \textbf{b\'{ɔ}}{}-m\`{ɛ} & \itshape \textbf{b\`{ɔ}}{}-dìa & `this palm wine\\
\textbf{NF}{}-\textsc{x} & \textbf{AGR}{}-this & \textbf{AGR}{}-be.good & ... & is/are good'\\
\end{tabular}

\z


\tabref{tab:Gueld:10} summarizes the agreement system of \ili{Lelemi}. Different from \citet{Allan1973} we posit one more agreement class, LE4, for plural nouns with a prefix \mbox{\textit{LE-},} because these display a distinct set of concord exponents, which is intermediate between that of LE3 and LE5 (cf.\ bold-faced elements in the table).

\begin{table}[htb]

\begin{tabularx}{\textwidth}{lXXXXXX}
\lsptoprule

AGR & Number & DEM/REL

SBJ/PART* & POSS & OBJ & PRO & {NF}

default\\
\midrule
LE1 & TN, SG & \itshape ɔ{}-/u- & \itshape ŋwa & \itshape \`{ŋ} & \itshape àŋu &  $-$\\
LE2 & PL & \itshape ba-/be{}- & \itshape Bana & \itshape mà & \itshape àma &  $-$\\
LE3 & SG & \bfseries\itshape lɛ{}-/li- & \bfseries\itshape anya & \itshape nì & \itshape àni & \itshape LE{}-\\
LE4 & TN, PL & \bfseries\itshape lɛ{}-/li- & \bfseries\itshape anya & \bfseries\itshape nyà & \bfseries\itshape ànya & \itshape LE{}-\\
LE5 & TN, PL & \itshape a-/e{}- & \itshape ana & \bfseries\itshape nyà & \bfseries\itshape ànya & \itshape A{}-\\
LE6 & all & \itshape kɔ{}-/ku{}- & \itshape kuna & \itshape kù & \itshape àku & \itshape KO{}-\\
LE7 & TN, SG & \itshape ka{}-/ke{}- & \itshape kana & \itshape kà & \itshape àka & \itshape KA{}-\\
LE8 & TN, PL & \itshape bɔ{}-/bu- & \itshape anya & \itshape mù & \itshape àmu &  $-$\\
\lspbottomrule
\end{tabularx}
{\small Note: * forms vary tonally according to grammatical context.}

\caption{Agreement class system of Lelemi (based on \citealt{Allan1973})}
\label{tab:Gueld:10}
\end{table}

The gender system is not given by \citet{Allan1973} but can be deduced from the relevant behavior of agreement classes. \figref{fig:Gueld:16} shows that it comprises 9 paired and 7 single-class patterns.

\begin{figure}
\begin{tabular}{lr>{\centering}p{\llen}l}
\lsptoprule
&  SG \tknode{0} & TN & \tknode{0} PL \\
\midrule
LE4  &  & %
 \tikz[remember picture,baseline=(16circ1.base)]\node[shape=rounded rectangle,inner sep=2pt,draw] (16circ1) {\mbox{\textit{lE-/nyà}}} ; %
   & \tknode{16B1} \textit{lE-/nyà} \\
\padding
LE1  &  \textit{O-} \tknode{16A2} &  %
 \tikz[remember picture,baseline=(16circ2.base)]\node[circle,inner sep=1pt,draw] (16circ2) {\textit{O-}} ; & \\
 \padding
LE2   &  & %
 \tikz[remember picture,baseline=(16circ3.base)]\node[circle,inner sep=0pt,draw] (16circ3) {\textit{ba-}} ; %
   & \tknode{16B3} \textit{ba-} \\
\padding
LE3  &  \textit{lE-/nì} \tknode{16A4} \\
\padding
LE5  & &    %
 \tikz[remember picture,baseline=(16circ4.base)]\node[circle,inner sep=2pt,draw] (16circ4) {\textit{a{}-}} ; %
   & \tknode{16B5} \textit{a-/nyà} \\
\padding
LE7  &  \textit{ka-} \tknode{16A6} &  %
 \tikz[remember picture,baseline=(16circ5.base)]\node[circle,inner sep=0pt,draw] (16circ5) {\textit{ka-}} ; & \\
\padding
LE6  &  \textit{kO-} \tknode{16A7} &  %
 \tikz[remember picture,baseline=(16circ6.base)]\node[circle,inner sep=0pt,draw,dashed] (16circ6) {\textit{kO-}} ; & \tknode{16B7} \textit{kO-} \\
\padding
LE8  & &     %
 \tikz[remember picture,baseline=(16circ1.base)]\node[circle,inner sep=0pt,draw] (16circ1) {\textit{bO-}} ; %
   & \tknode{16B8}  \textit{bO-} \\
\lspbottomrule
\end{tabular}

% lines connecting the nodes
\tikz[remember picture, overlay] \draw[thick] (16A2.center) -- (16B1.center) ;
\tikz[remember picture, overlay] \draw[thick] (16A2.center) -- (16B3.center) ;
\tikz[remember picture, overlay] \draw[thick] (16A4.center) -- (16B3.center) ;
\tikz[remember picture, overlay] \draw[thick] (16A4.center) -- (16B5.center) ;
\tikz[remember picture, overlay] \draw[thick] (16A6.center) -- (16B3.center) ;
\tikz[remember picture, overlay] \draw[thick] (16A6.center) -- (16B7.center) ;
\tikz[remember picture, overlay] \draw[thick,dashed] (16A6.center) -- (16B8.center) ;
\tikz[remember picture, overlay] \draw[thick,dashed] (16A7.center) -- (16B3.center) ;
\tikz[remember picture, overlay] \draw[thick] (16A7.center) -- (16B5.center) ;

\caption{Gender system of Lelemi (based on \citealt{Allan1973})}
\label{fig:Gueld:16}
\end{figure}

Heine (\citeyear[114--115]{Heine1968}, \citeyear[197--198]{Heine1982}) has also presented an analysis of the noun classification system of \ili{Lelemi} with a focus on the \ili{Tetemang} variety, which in turn has been reanalyzed by \citet[173--175]{Corbett1991} from his typological perspective on gender. \figref{fig:Gueld:17} summarizes the results, including Corbett's argument that some agreement class pairs should be viewed as inquorate genders.


\begin{figure}

\begin{tabular}{lr>{\centering}p{\llen}l}
\lsptoprule
&  SG \tknode{0} & TN & \tknode{0} PL \\
\midrule
LE1 & \textit{o-} \tknode{17A1} \\
\padding
LE2 & & & \tknode{17B2} \textit{ba-} \\
\padding
LE3 &  \textit{le-} \tknode{17A3} \\
\padding
LE4/5 & & & \tknode{17B4} \textit{a-/(le-)} \\
\padding
LE7  & \textit{ka-} \tknode{17A5} \\
\padding
LE6 & \textit{ko-} \tknode{17A6} & & \tknode{17B6} \textit{ko-} \\
\padding
LE8 & \textit{bo-} \tknode{17A7} & & \\
\lspbottomrule
\end{tabular}

% lines connecting the nodes
\tikz[remember picture, overlay] \draw[thick] (17A1.center) -- (17B2.center) ;
\tikz[remember picture, overlay] \draw[thick] (17A1.center) -- (17B4.center)
%
node[near start,anchor=north] {?};
\tikz[remember picture, overlay] \draw[thick] (17A3.center) -- (17B4.center) ;
\tikz[remember picture, overlay] \draw[thick,dashed] (17A5.center) -- (17B4.center) ;
\tikz[remember picture, overlay] \draw[thick] (17A5.center) -- (17B6.center) ;
\tikz[remember picture, overlay] \draw[thick,dashed] (17A6.center) -- (17B2.center) ;
\tikz[remember picture, overlay] \draw[thick] (17A6.center) -- (17B4.center) ;
\tikz[remember picture, overlay] \draw[thick,dashed] (17A7.center) -- (17B2.center) ;
\tikz[remember picture, overlay] \draw[thick,dashed] (17A7.center) -- (17B4.center) ;

\caption{Gender system of Lelemi (based on \citealt{Heine1968} and \citealt{Corbett1991})}
\label{fig:Gueld:17}
\end{figure}

The considerable divergence between the gender systems in the Figures \ref{fig:Gueld:16} and \ref{fig:Gueld:17} may be partly accounted for by dialect differences, given that Allan and Heine focused on \ili{Baglo} and \ili{Tetemang}, respectively. It is clear, however, that some differences are due to diverse analytical approaches. One crucial point is the identification of the additional plural LE4 for which \citet[115]{Heine1968} also appears to present evidence with the demonstrative \textit{-mɛ} but which \citet[173]{Corbett1991} discards as a case of an overdifferentiated target. Another major difference in Heine's analysis of \ili{Lelemi} (albeit not in his family reconstruction, see \sectref{sec:Gueld:3.4.3}) is the non-recognition of single-class genders, although there are some likely candidates, notably with LE8.

A final but important point regarding the previous analyses of \ili{Lelemi} relates to the typologically oriented interpretation of the philological framework to \ili{Niger-Congo} noun classification. That is, the description of \ili{Lelemi}, couched by Heine (\citeyear{Heine1968}, \citeyear{Heine1982}) in this tradition, misled \citet[173--175]{Corbett1991} to a confusing analysis in that he calls the language's genders inappropriately ``agreement classes''. That the presentation of \ili{Niger-Congo} data in particular causes such problem appears to be significant, because in general this author has applied his cross-linguistic approach successfully to a wide range of structurally diverse and complex gender systems.


\begin{table}[!htb]

\begin{tabularx}{\textwidth}{lllX}
\lsptoprule

NF & Form(s) &  & Example(s)\\
\midrule
\itshape Ø & $-$ & {SG} & {\textit{wɛwɛ} `dog'} \\
 & & {TN} & {\textit{sìka} `money';} \textit{twif\`{ɔ}} `Twi speaking person/people' \\
%
\itshape O- & \itshape ɔ{}-/u{}- & {SG} & \textit{ù{}-}\textit{culi} `person'; \textit{\`{ɔ}-g}\textit{bà} `foot' \\
&&TN & \textit{ù{}-}\textit{bòja} `blood'\\
%
\itshape BA- & \itshape ba{}-/be{}- & PL & \textit{bà{}-}\textit{wɛwɛ} `dogs'; \textit{bè{}-}\textit{culi} `people'; \textit{bè{}-}\textit{kùkù} `owls'; \textit{bè{}-}\textit{se} `goats'; \textit{bà{}-}\textit{làkpi} `snakes'; \textit{be{}-}\textit{yu} `monkeys'\\
%
\itshape LE- & \itshape lɛ{}-/li{}- & {SG} & {\textit{lì{}-}\textit{kùkù} `owl'; \textit{lɛ}\textit{{}-}\textit{nimì} `eye'} \\
&&{TN} & {\textit{lɛ{}-}\textit{na} `meat'}\\
&&PL & \textit{l\`{ɛ}-g}\textit{bà} `feet'\\
%
\itshape A- & \itshape a-/e- & {SG} & \textit{è{}-}\textit{se} `goat' \\
&& {TN} & \textit{a{}-}\textit{ba} `mud' \\
&& PL & \textit{a{}-}\textit{nimì} `eyes'; \textit{e{}-}\textit{ji} `trees'\\
%
\itshape KO- & \itshape kɔ{}-/ku{}- & {SG} & {\textit{k\`{ɔ}}\textit{{}-}\textit{làkpi} `snake'; \textit{ku{}-}\textit{ji} `tree'} \\
&&{TN} & {\textit{ku{}-}\textit{tu} `soup'} \\
&& PL & \textit{k\`{ɔ}-}\textit{bwa} `hats'\\
%
\itshape KA- & \itshape ka{}-/ke- & {SG} & {\textit{ke{}-}\textit{yu} `monkey'; \textit{kà{}-}\textit{bwa} `hat'; \textit{ke{}-}\textit{mo} `farm'} \\
&&TN & \textit{ka{}-}\textit{na} `porridge'\\
%
\itshape N- & {\itshape m-/n-/ŋ-} & {TN} & {\textit{n{}-}\textit{tu} `water'; \textit{\`{ŋ}{}-}\textit{kpa} `life'} \\
&& PL & \textit{m{}-}\textit{mo} `farms', \textit{ǹ{}-}\textit{culi} `people (with NUM)'\\
%
\itshape BO- & \itshape bɔ{}-/bu- & TN & \textit{bɔ{}-}\textit{ŋwa} `cooking'\\
\lspbottomrule
\end{tabularx}

\caption{Nominal form class system of Lelemi (based on \citealt[97--124]{Allan1973})\protect\footnotemark{}}
\label{tab:Gueld:11}
\end{table}
%
\footnotetext{The tone marking in the table follows Allan's (\citeyear{Allan1973}) transcription: V́ high tone, V mid tone, V̀ low tone.}


\begin{figure}[!htb]

\begin{tabular}{>{\itshape}r>{\centering\itshape}p{\llen}>{\itshape}l}
\lsptoprule
\normalfont  SG \tknode{0} & \normalfont TN & \tknode{0} \normalfont PL \\
\midrule
  Ø \tknode{18A1} &  %
 \tikz[remember picture,baseline=(18circ1.base)]\node[circle,inner sep=2pt,draw] (18circ1) {Ø} ; & \\
\padding
  O- \tknode{18A2} &   %
 \tikz[remember picture,baseline=(18circ2.base)]\node[circle,inner sep=1pt,draw] (18circ2) {O-} ; & \\
\padding
     & & \tknode{18B3} BA-\\
\padding
  LE- \tknode{18A4} & %
 \tikz[remember picture,baseline=(18circ3.base)]\node[circle,inner sep=0pt,draw] (18circ3) {LE-} ; & \tknode{18B4} LE- \\
\padding
  A- \tknode{18A5} & %
 \tikz[remember picture,baseline=(18circ4.base)]\node[circle,inner sep=1pt,draw] (18circ4) {A-} ; & \tknode{18B5} A- \\
\padding
   KA- \tknode{18A6} & %
 \tikz[remember picture,baseline=(18circ5.base)]\node[circle,inner sep=0pt,draw] (18circ5) {KA-} ; & \\
\padding
  KO- \tknode{18A7} &  %
 \tikz[remember picture,baseline=(18circ6.base)]\node[circle,inner sep=0pt,draw] (18circ6) {KO-} ; & \tknode{18B7} KO- \\
\padding
 &    %
 \tikz[remember picture,baseline=(18circ7.base)]\node[circle,inner sep=1pt,draw] (18circ7) {N-} ; & \tknode{18B8} N- \\
\padding
  & %
 \tikz[remember picture,baseline=(18circ8.base)]\node[circle,inner sep=0pt,draw] (18circ8) {BO-} ; & \\
\lspbottomrule
\end{tabular}

% lines connecting the nodes
\tikz[remember picture, overlay] \draw[thick] (18A1.center) -- (18B3.center) ;
\tikz[remember picture, overlay] \draw[thick] (18A2.center) -- (18B3.center) ;
\tikz[remember picture, overlay] \draw[thick] (18A2.center) -- (18B4.center) ;
\tikz[remember picture, overlay] \draw[thick] (18A4.center) -- (18B3.center) ;
\tikz[remember picture, overlay] \draw[thick] (18A4.center) -- (18B5.center) ;
\tikz[remember picture, overlay] \draw[thick] (18A5.center) -- (18B3.center) ;
\tikz[remember picture, overlay] \draw[thick,dashed] (18A6.center) -- (18B3.center) ;
\tikz[remember picture, overlay] \draw[thick] (18A6.center) -- (18B7.center) ;
\tikz[remember picture, overlay] \draw[thick,dashed] (18A6.center) -- (18B8.center) ;
\tikz[remember picture, overlay] \draw[thick,dashed] (18A7.center) -- (18B3.center) ;
\tikz[remember picture, overlay] \draw[thick] (18A7.center) -- (18B5.center) ;

\caption{Deriflection system of Lelemi (based on \citealt[100]{Allan1973})}
\label{fig:Gueld:18}
\end{figure}

Turning to \ili{Lelemi}'s system of noun form and deriflection classes, Allan's information can be summarized as in \tabref{tab:Gueld:11} and \figref{fig:Gueld:18}.

Although \ili{Lelemi}'s crossed gender system is already complex, its deriflection system is yet more elaborate, due notably to an additional prefixless nominal form class and another one in \textit{N-}. It comprises 11 singular-plural affix pairings, albeit three of them inquorate. Nominal form classes are remarkable regarding their number behavior in that most of them are attested with more than one number value (only \textit{BA-} and \textit{BO-} are restricted to plural animates and transnumeral infinitives, respectively), and three of them are even attested in both singular and plural. Most of the discrepancies between gender and deriflection are thus due to the fact that agreement and nominal form classes show numerous patterns diverging from the expected biunique \ili{Niger-Congo} canon, as shown in \figref{fig:Gueld:19}.


\begin{figure}[!htb]

\begin{tabular}{lrp{\llen}ll}
\lsptoprule
 & AGR \tknode{0} &  & \tknode{0} NF &  Number* \\
\midrule
LE2 & \textit{ba-} \tknode{19A1} &  & \tknode{19B1} \textit{BA-} &   PL \\
   & X \tknode{0} &  & \tknode{19B2} \textit{Ø}  &  TN, SG, PL \\
LE1 &   \textit{O-} \tknode{19A3} &  & \tknode{19B3} \textit{O-}  &  TN, SG \\
LE5  &  \textit{a-} \tknode{19A4} &  &  \tknode{19B4} \textit{A-}  &  TN, SG, PL \\
LE3  &  \textit{lE-} \tknode{19A5} &  & \tknode{19B5} \textit{LE-}  &  SG \\
LE4  &  \textit{lE-} \tknode{19A6} &  & \tknode{0} X   & TN, PL \\
LE6 &   \textit{kO-} \tknode{19A7} &  & \tknode{19B7} \textit{KO-}  &  TN, SG, PL \\
LE7  &  \textit{ka-} \tknode{19A8} &  & \tknode{19B8} \textit{KA-} &   TN, SG \\
LE8  &  \textit{bO-} \tknode{19A9} &  & \tknode{19B9} \textit{BO-}  &  TN, (PL) \\
  &  X \tknode{0} &  & \tknode{19B10} \textit{N-}  &  TN, PL \\
\lspbottomrule
\end{tabular}

% lines connecting the nodes
\tikz[remember picture, overlay] \draw[thick] (19A1.center) -- (19B1.center) ;
\tikz[remember picture, overlay] \draw[thick] (19A1.center) -- (19B2.center) ;
\tikz[remember picture, overlay] \draw[thick] (19A3.center) -- (19B2.center) ;
\tikz[remember picture, overlay] \draw[thick] (19A3.center) -- (19B3.center) ;
\tikz[remember picture, overlay] \draw[thick] (19A3.center) -- (19B4.center) ;
\tikz[remember picture, overlay] \draw[thick] (19A4.center) -- (19B4.center) ;
\tikz[remember picture, overlay] \draw[thick] (19A5.center) -- (19B5.center) ;
\tikz[remember picture, overlay] \draw[thick] (19A6.center) -- (19B5.center) ;
\tikz[remember picture, overlay] \draw[thick] (19A7.center) -- (19B7.center) ;
\tikz[remember picture, overlay] \draw[thick] (19A8.center) -- (19B8.center) ;
\tikz[remember picture, overlay] \draw[thick] (19A9.center) -- (19B9.center) ;
\tikz[remember picture, overlay] \draw[thick] (19A9.center) -- (19B10.center) ;

{\small Note: X = no independent class counterpart in the other class type.\\
* may join behavior for both AGR and NF}


\caption{Mapping of agreement and nominal form classes in Lelemi (based on \citealt[128]{Allan1973})}
\label{fig:Gueld:19}
\end{figure}

\subsubsection{Proto-Ghana-Togo-Mountain}
\label{sec:Gueld:3.4.3}

The noun classification systems of \ili{Ghana-Togo-Mountain} languages have been subject to historical-comparative analysis by \citet{Heine1968}. Since the very genealogical unity of the group is disputed, Heine's results are in principle controversial. In this context, however, we focus on another problem of his reconstruction, namely that he closely follows the problematic philological approach to \ili{Niger-Congo} ``noun classes'', which obscures a transparent treatment of gender and nominal deriflection. \citet[112]{Heine1968} writes:

\begin{quote}
Ein Nominalklassensystem liegt vor, wenn\\
a) Nominalklassen bestehen, d.h.\ die Nomina durch Affixe in Klassen ein\-ge\-teilt werden,\\
b) Paarigkeit der Klassenaffixe vorhanden ist, d.h.\ einem sg-Affix ein be\-stimmtes pl-Affix entspricht bzw. umgekehrt, und wenn\\
c) nach einer Nominalklassenkonkordanz verfahren wird, d.h.\ wenn den Nominalklassenaffixen an verschiedenen grammatischen Kategorien re\-gel\-mä\-ßig zugeordnete Klassen-Zeichen entsprechen.\\
{[}We speak of a noun class system if a) there are noun classes, that is, nouns are sorted by affixes into different classes; b) the class affixes occur in pairs, that is, a certain singular affix corresponds to a certain plural affix and vice versa; and if c) there is noun class concord, that is, if the noun class affixes correlate regularly with class exponents on different grammatical categories.{]}
\end{quote}

Heine's awareness of the importance of agreement is reflected in his data presentation for single languages (ibid.: 113--123) as well as the exclusion of three languages from the reconstruction that according to him (ibid.: 276--277) no longer display class concord, namely \ili{Ikposo}, \ili{Igo}, and \ili{Animere} (it turns out that this holds in fact only for the first language). Nevertheless, he focuses predominantly on the nominal affix system and often conflates agreement and noun forms, which makes it hard to distinguish the two. Finally, when reconstructing the ``noun class'' system of the entire group (ibid.: 187--211), he almost exclusively discusses the noun affixes; only in rare, unclear cases does he resort to the role of agreement forms.

A final point, which has also been made in \sectref{sec:Gueld:3.3} regarding the comparative work on \ili{Guang}, concerns the reconstruction bias toward Proto-\ili{Bantu}. Heine's proto-system, schematized in \figref{fig:Gueld:20}, demonstrates that the inventory and numbering of the majority of his ``noun classes'' are, to the extent possible, clearly modeled on and also implicitly justified (ibid.: 187) by the conflated Proto-\ili{Bantu} system, whose two components were shown in \figref{fig:Gueld:8} of \sectref{sec:Gueld:2}.


\begin{figure}

\begin{tabular}{lrp{\llen}ll}
\lsptoprule
 &  SG \tknode{0} &  TN & \tknode{0} PL & \\
\midrule
1/3  &  *o- \tknode{20A1} &  & \\
   &  &  & \tknode{20B2} *ba-  &  2 \\
   &  &  & \tknode{20B3} *i-    &  4 \\
7  &  *ki- \tknode{20A4} &  &  & \\
   &  &  & \tknode{20B5} *bi-  &  8 \\
5  & *li- \tknode{20A6} &  &  & \\
    &  &  & \tknode{20B7} *a-  &  6/10 \\
9  &  *ku- \tknode{20A8} &  &  \tknode{20B8}  *ku-  &  15 \\
13  &  *ka- \tknode{20A9} &  &  & \\
    &  &  *bu-  & \tknode{20B10} *bu-  &  14 \\
11  &  &    *N-  &  & \\
12  &  &    *ti-  &  & \\
\lspbottomrule
\end{tabular}

% lines connecting the nodes
\tikz[remember picture, overlay] \draw[thick] (20A1.center) -- (20B2.center) ;
\tikz[remember picture, overlay] \draw[thick] (20A1.center) -- (20B3.center) ;
\tikz[remember picture, overlay] \draw[thick] (20A4.center) -- (20B5.center) ;
\tikz[remember picture, overlay] \draw[thick] (20A6.center) -- (20B7.center) ;
\tikz[remember picture, overlay] \draw[thick] (20A8.center) -- (20B7.center) ;
\tikz[remember picture, overlay] \draw[thick] (20A9.center) -- (20B8.center) ;
\tikz[remember picture, overlay] \draw[thick] (20A9.center) -- (20B10.center) ;

\caption{``Noun class'' system of Proto-Ghana-Togo-Mountain by \citet[187]{Heine1968}}
\label{fig:Gueld:20}
\end{figure}

Since Heine's (\citeyear{Heine1968}) work many studies dealing to different degrees with the noun classification systems of individual \ili{Ghana-Togo-Mountain} languages have appeared. Despite the much more complete data available today it remains hard to reconstruct a robust proto-system, irrespective of the classificatory status of the group. This is because most language-specific treatments are still biased toward nominal form classes and deriflections and neglect agreement, which is crucial for determining the gender system. That is, we have come across studies for only three of the 16 languages where the agreement and resulting gender systems receive primary attention by the respective authors, namely \citet{Zaske2007} on \ili{Anii}, \citet{Essegbey2009} on \ili{Nyangbo}, and Agbetsoamedo (\citeyear{Agbetsoamedo2014a}, \citeyear{Agbetsoamedo2014}) on \ili{Selee}, while in all other descriptions this domain plays a secondary role, is overly conflated with nominal form classes, or is lacking altogether.

\section{Summary}
\label{sec:Gueld:4}


We have outlined the traditional approach to the noun categorization systems of the \ili{Niger-Congo} type found in a large number of African languages and argued that it is in need of revision for the sake of better language-specific synchronic as well as historical-comparative analyses. This holds in addition to the comparative bias toward the \ili{Bantu} system, which tends to conceal a large part of the existing diversity across \ili{Niger-Congo} languages.

One bias in the ``noun class'' framework is the strong focus on the affix status of class exponents. One consequence in the realm of nominal form classes is the overall analytical neglect of nouns without class affixes despite their important and partly diagnostic role in the nominal system.

Another crucial problem of the current \ili{Niger-Congo} approach is the stereotypical view about agreement and nominal form classes in that the large majority of ``noun classes'' are assumed to be functionally dedicated to a specific gender and number value. As shown in the discussion of Proto-\ili{Bantu} in \sectref{sec:Gueld:2}, this situation is not even universal in the group that was the inspiration for this assumption. However, the degree of deviation from this hypothetical prototype can be much higher, so that this overgeneralized view should give way to a more neutral approach. In particular, this phenomenon throws a different light on the underlying number system in that the overall importance of transnumeral nouns seems to be higher than commonly assumed. That is, the data should no longer be dealt with according to a simple and universal singular-plural distinction.

The last and most important drawback of the traditional \ili{Niger-Congo} framework is that its central concept of ``noun class'' conflates two independent linguistic phenomena associated with nouns: gender agreement between a nominal trigger and its target and deriflection reflected in morphological and/or phonological regularities of nouns. Their unified treatment has several negative effects for the current investigation of this domain. These are in particular an inappropriate focus on deriflection systems, a resulting neglect of a transparent and comprehensive analysis of agreement-based gender, and finally an impeded investigation of the exact relationship between the two distinct components, including their complex interdependency.

The disadvantages of the ``noun class'' concept negatively impact the transparency and even adequacy of language-specific descriptions. In the worst case, it may be impossible to establish the inventory of a language's gender distinctions and its semantic and formal basis in spite of a lengthy treatment of ``noun classes''. As discussed above, this is not restricted to a case like the heavily restructured \ili{Akan} treated in \sectref{sec:Gueld:3.2}, for which scholars go into great detail about its classificatory morphology on nouns but fail to explicitly identify the occasional existence of an animacy-based gender system.

Synchronic descriptive problems inevitably carry over to the historical reconstruction of noun classification in \ili{Niger-Congo}, as shown for the \ili{Guang} and \ili{Ghana-Togo-Mountain} groups in \sectref{sec:Gueld:3.3} and \sectref{sec:Gueld:3.4}, respectively. The general bias toward the \ili{Bantu} family aside, available proto-systems are not only unrealistic vis-à-vis the attested modern data but simply difficult to interpret linguistically in mixing distinct grammatical phenomena in a single paradigm.

Last but not least, it is hard to impossible for typologists to integrate a considerable amount of \ili{Niger-Congo} data, in particular on complex systems, in cross-linguistic surveys on gender due to the intractable amalgamation of gender and deriflection. The typological incompatibility and thus ``opaqueness'' of many \ili{Niger-Congo} descriptions deprives this research domain of interesting cases the analysis of which is necessary in order to arrive at meaningful cross-linguistic generalizations.

We venture that the cross-linguistic framework outlined in \sectref{sec:Gueld:1} is universally viable for language-specific, historical-comparative, and typological analyses. The restricted data presented here suggest several generalizations that are worth testing against a wider range of data. For example, the observation made in \citet{Gueldemann2000} that agreement classes need not be dedicated to specific gender and number values is demonstrably relevant for a much larger number of languages, and it can also be extended in \ili{Niger-Congo} to nominal form classes. As proposed in \citet{Gueldemann2000}, the degree of this functional insensitivity of classes is reflected in the ratio between genders and agreement classes (or, for that matter, between deriflections and nominal form classes). In typological comparison, this promises to serve as a good proxy for assessing basic structural differences between systems.

There is another conclusion that may turn out to be cross-linguistically significant, even though the data presented here are admittedly limited. That is, in languages with gender-sensitive noun morphology these deriflection systems are regularly more complex, or at least not simpler, than the associated gender systems in terms of inventory as well as systemic structure as per \citet{Heine1982} and \citet{Corbett1991}.

For \ili{Niger-Congo} languages, one can assume that the two subsystems of this nominal domain were originally very similar. This suggests for this group that deriflection systems tend to be more conservative than gender systems. With respect to the former, the transfer of individual or entire groups of nouns from one to another nominal form class, the merger of nominal form classes, and the resulting effects on deriflections are certainly rampant in the family. However, the changes in agreement-based gender marking are recurrently even more frequent and drastic, up to the reorganization, or even loss, of the entire system.

As long as the divergences between the two subsystems of gender and deriflection are minor, they will not differ dramatically in terms of their classification of nouns into sets. However, quite a few cases in \ili{Niger-Congo} are different. For example, \ili{Akan}, dealt with in \sectref{sec:Gueld:3.2}, possesses a binary system of animate vs.\ inanimate gender but an elaborate deriflection system with more and different categorizing distinctions. Languages of this type inform the new topic of so-called ``concurrent systems'' of noun classification, as investigated recently by  \citet{Fedden2017} but for which the authors failed to recognize the relevance of \ili{Niger-Congo}. Thus, a more detailed and typologically sound investigation of some of its languages where deriflection and gender have grown apart is a very worthwhile undertaking for the future.

In summary, this paper attempts to make two major contributions to the treatment of gender. First, the linguistic analysis of \ili{Niger-Congo}-type noun classification systems should be better aligned with a sound cross-linguistic perspective. The detrimental philological approach, which is of a substantial rather than merely terminological nature, is not necessitated by any linguistic structures in \ili{Niger-Congo}, however quirky they may appear from a cross-linguistic view. Second, we make a new proposal for a universally applicable framework for gender systems, especially useful if gender interacts intimately with the morpho(phono)logy of nouns. The approach based on the four analytical concepts outlined in \sectref{sec:Gueld:1} could not be fully expounded here by means of a wider language sample. However, its viability has been shown for the specific gender-system profile of the important group of \ili{Niger-Congo} languages. It has also been applied successfully to structurally quite different languages from such families as \ili{Kx'a} and \ili{Tuu} in southern Africa, \ili{Kadu} and \ili{Cushitic} in northeastern Africa, and yet others. Hence, we venture to review the approach to gender from a wider typological perspective in line with the present framework.


\section*{Acknowledgments}
This paper or parts thereof were previously presented at the International Workshop on ``Grammatical Gender and Linguistic Complexity'' at the Department of Linguistics of Stockholm University, 20--21 November 2015; at the Linguistics Department of the University of California Berkeley, 30 March 2016; the International Conference ``Toward Proto-\ili{Niger-Congo} II'' at LLACAN, Paris, 1--3 September 2016; at the Department of African and Ethiopian Studies of Hamburg University, 24 January 2018; and at the  Diversity Linguistics Seminar at Leipzig University, 1 February 2018. We are grateful for the fruitful feedback by the respective audiences. Our thanks also go to the extensive and productive comments by the editors of this volume, two anonymous reviewers, and Martin Haspelmath. Last but not least, we gratefully acknowledge the funding received from the German Research Foundation (DFG) for the project 'Noun classification in Africa between gender and declension' within which the greater part of our research presented here was carried out.

\section*{Special abbreviations}

\noindent The following abbreviations are not found in the Leipzig Glossing Rules:
\medskip

\begin{tabular}{llll}
AGR 	&	 agreement class 	&	NUM 	&	  numeral	\\
AN  	&	 animate & PART 	&	  participle	\\
CONC  	&	 pronominal concord &	PERF  	&	 perfect	\\
D 	&	  distal 	&	PRO  	&	 pronoun	\\
IAN  	&	 inanimate &	TN 	&	  transnumeral	\\
 NF 	&	  nominal form class && \\

\end{tabular}
\medskip

\noindent Arabic numbers represent agreement classes while Roman numbers represent genders.

\sloppy
\printbibliography[heading=subbibliography,notkeyword=this]

\end{document}
