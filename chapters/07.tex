\documentclass[output=collectionpaper]{langsci/langscibook}

\title{Gender in Walman%
%\footnote{}%  % Acknowledements moved to end of chapter
}

\author{%
Matthew S. Dryer
\affiliation{University at Buffalo}
}%

% \chapterDOI{} %will be filled in at production

\abstract{%
\label{firstpage:Dryer}
In this paper, I describe gender and gender-like phenomena in Walman, a language of the Torricelli family spoken on the north coast of Papua New Guinea. I discuss three topics. One of these is the two clear instances of gender in Walman, masculine and feminine. I discuss the formal realization of gender in Walman and the factors governing the choice of masculine versus feminine gender.\\
There are also two gender-like phenomena in Walman, namely pluralia tantum nouns and diminutive. Pluralia tantum nouns in Walman are different from pluralia tantum nouns in European languages in that what makes them grammatically plural is not their form, but the fact that they control plural agreement. What makes pluralia tantum gender-like is that there are twice as many pluralia tantum nouns in our data as there are nouns that are lexically masculine.\\
The second gender-like phenomenon in Walman is a diminutive category, which is coded in the same way as feminine singular, masculine singular, and plural. What makes it unlike phenomena that are normally considered instances of gender in other languages is the fact that there are no lexically diminutive nouns and any noun can be associated with diminutive agreement.
\medskip

\textbf{Keywords:}
gender, masculine, feminine, diminutive, pluralia tantum, Walman, Torricelli
}%


\maketitle
\begin{document}


\section{Introduction}

The goal of this paper is to give a description of gender in Walman, a language in the Torricelli family spoken in Papua New Guinea. I understand gender to denote a morphosyntactic category in a language based on a division among nouns in the language and on agreement phenomena related to this division. There are two unambiguous instances of genders in Walman, namely masculine and feminine. But there are also two other gender-like phenomena in the language, namely pluralia tantum nouns and a diminutive category. I will describe the first of these phenomena in some detail in this paper, discussing ways in which it is like or unlike clear instances of gender. My discussion of the diminutive category will be briefer, since it is discussed in more detail elsewhere, in \citet{Dryer2016} and \citet{DryerUnderrevision}.

In \sectref{sec:Dry:2}, I provide a brief grammatical sketch, primarily describing inflectional categories that vary for gender. In \sectref{sec:Dry:3}, I describe the factors governing the choice between masculine and feminine gender. In \sectref{sec:Dry:4}, I describe pluralia tantum nouns in Walman and in \sectref{sec:Dry:5}, I briefly describe the Walman diminutive.

\section{Brief grammatical sketch}
\label{sec:Dry:2}

This section focuses primarily on the coding of gender in Walman, along with the coding of number, person, and diminutiveness. See \citet{DryerInpreparation} for a description of other features of Walman.

Verbs in Walman inflect for both subject and object (and in some applicative constructions, for two objects). The subject affixes are word-initial prefixes consisting of single consonants, as in (\ref{ex:Dry:1}), where the verb \textit{mara} `come' bears a \textsc{1sg} subject prefix \textit{m-} and the verb \textit{nawa} `call' bears a \textsc{2sg} subject prefix \textit{n-}.

\ea \label{ex:Dry:1}
\gll Kum \textbf{m-}ara eni chi \textbf{n-}awa.\\
\textsc{1sg} \textbf{\textsc{1sg.subj}}-come because \textsc{2sg} \textbf{\textsc{2sg.subj}}-call\\
\glt `I came because you called.'
\z

Example (\ref{ex:Dry:2}) contains two occurrences of the \textsc{1pl} subject prefix \textit{k-}.

\ea \label{ex:Dry:2}
\gll Akou \textbf{k}-anan \textbf{k}-ara komoru.\\
finish \textbf{\textsc{1pl}}-go.down \textbf{\textsc{1pl-}}come evening\\
\glt `Then we walked home in the afternoon.'
\z

The \textsc{2pl} subject prefix \textit{ch}- is illustrated in (\ref{ex:Dry:3}).%
\footnote{Our orthography for Walman employs three digraphs, <ch> for [tʃ], <ng> for [ŋ], and <ny> for [ɲ].}

\ea \label{ex:Dry:3}
\gll
Chim	 \textbf{ch}-orou nyien?\\
\textsc{2pl} \textbf{\textsc{2pl}}-go where\\
\glt `Where are you (plural) going?'
\z

Example (\ref{ex:Dry:4}) contains two occurrences of the \textsc{3pl} subject prefix \textit{y-}.

\ea \label{ex:Dry:4}
\gll Ri	pelen	\textbf{y}-anan	\textbf{y}-okorue	wul.\\
\textsc{3pl} dog \textbf{\textsc{3pl}}-go.down \textbf{\textsc{3pl}}-bathe water\\
\glt `Then the dogs went in for a wash.'
\z

As mentioned above, there are two clear cases of gender in Walman, masculine and feminine; this distinction is realized only in the \textsc{3sg}. Example (\ref{ex:Dry:5}) illustrates the \textsc{3sg.m} subject prefix \textit{n-} (again occurring twice).

\ea \label{ex:Dry:5}
\gll Runon	\textbf{n}-rukuel 	\textbf{n}-anan	nyuey.\\
\textsc{3sg.m} \textbf{\textsc{3sg.m}}-run \textbf{\textsc{3sg.}}\textsc{m}-go.down sea\\
\glt `He ran to the beach.'
\z

And (\ref{ex:Dry:6}) illustrates the \textsc{3sg.f} subject prefix \textit{w-}.

\ea \label{ex:Dry:6}
\gll Nakol	kkuk	\textbf{w}-anan.\\
house broken \textbf{\textsc{3sg.f}}-go.down\\
\glt `The house fell down.'
\z

There is also a diminutive subject prefix \textit{l-}, illustrated on \textit{lakor} `drown' in (\ref{ex:Dry:7}).

\ea \label{ex:Dry:7}
\gll Nyanam	mon	ro-l,	ampa	rul \textbf{l}-akor	wul.\\
child \textsc{neg} tall-\textsc{dimin} \textsc{fut} \textsc{3.dimin} \textbf{\textsc{3.dimin}}-drown water\\
\glt `The child is small, she will drown.'
\z

Although the diminutive is like masculine and feminine in being restricted to singular, it involves a distinct notion of `singular', as discussed in \sectref{sec:Dry:5} below.

There is also a set of object affixes that occur on transitive verbs, though they occur in three different positions within the verb. The first and second person object affixes are prefixes that immediately follow the subject prefixes. These prefixes are unspecified for number and are illustrated in (\ref{ex:Dry:8}) by the first person object prefix \textit{p}{}- and in (\ref{ex:Dry:9}) by the second person object prefix \textit{ch}{}-.

\ea \label{ex:Dry:8}
\gll Kum	m-alma,	chim	ch-\textbf{p}-chami. \\
\textsc{1sg} \textsc{1sg}{}-die \textsc{2pl} \textsc{2pl}-\textbf{\textsc{1obj}}-bury\\
\glt `When I die, bury me.'
\z

\ea \label{ex:Dry:9}
\gll Opucha	mol	w-\textbf{ch}-any	chi?\\
thing which \textsc{3sg.f}-\textbf{\textsc{2obj}}-happen.to \textsc{2sg}\\
\glt `What happened to you?'
\z

A reflexive/reciprocal prefix /r/ occurs in the same slot as the first and second person object prefixes, as illustrated by the verb \textit{yrklwaro} `they deceived each other' in (\ref{ex:Dry:10}).

\ea \label{ex:Dry:10}
\gll Kamte-n	ngo-n	w-ri	Walis	n-aro-n nyemi	kasim	y-\textbf{r}-klwaro\\
person-\textsc{m} one-\textsc{m} \textsc{gen}-\textsc{3pl} Walis \textsc{3sg.m}-and-\textsc{3sg.m} friend friend \textsc{3pl}-\textbf{\textsc{refl/recip}}-deceive\\
\glt  `A man from Walis Island and his friend deceived each other.'
\z

The third person object affixes are generally suffixes, though with a minority of verbs they are infixes. Examples (\ref{ex:Dry:11}) and (\ref{ex:Dry:12}) illustrate the \textsc{3pl} and \textsc{3sg.m} object suffixes respectively.

\ea \label{ex:Dry:11}
\gll Kum	m-ete-\textbf{y}	wuel	chomchom.\\
\textsc{1sg} \textsc{1sg}-see-\textbf{\textsc{3pl.obj}} pig many\\
\glt `I saw many pigs.'
\z

\ea \label{ex:Dry:12}
\gll Ru	w-lro-\textbf{n}	runon.\\
\textsc{3sg.f} \textsc{3sg.f}-like-\textbf{\textsc{3sg.m:obj}} \textsc{3sg.m}\\
\glt `She likes him.'
\z

The form of the third person object affixes is, with one exception, the same as the corresponding subject prefixes. For example, /n/ is the form of both the \textsc{3sg.m} subject prefix, as in (\ref{ex:Dry:5}) above, and the \textsc{3sg.m} object affix, as in (\ref{ex:Dry:12}). The one difference between the third person subject prefixes and third person object affixes is in \textsc{3sg.f}, where the subject prefix is \textit{w}{}-, as in (\ref{ex:Dry:12}), while the object affix is phonologically null, as illustrated by the form \textit{mete} `see' in (\ref{ex:Dry:13}) (contrasting, for example, with the presence of an overt object suffix for \textsc{3pl} in the form \textit{metey} in (\ref{ex:Dry:11})).

\ea  \label{ex:Dry:13}
\gll Kum	m-ete-\textbf{ø}	chuto	nyanam.\\
\textsc{1sg} \textsc{1sg}-see-\textbf{\textsc{3sg.f}} woman child\\
\glt `I saw a young girl.'
\z

With some verbs, the third person object affixes are infixes, as in the form \textit{yanpu} `kill' in (\ref{ex:Dry:14}), where the \textsc{3sg.m} object affix -\textit{n}{}- is an infix inside the verb stem -\textit{apu} `kill'.

\ea  \label{ex:Dry:14}
\gll Rim	y-a<\textbf{n}>pu	ampatu	mon	nngkal.\\
\textsc{3pl} \textsc{3pl}-kill\textbf{<\textsc{3sg.m}}> ground.wallaby \textsc{neg}  small\\
\glt `They killed a big wallaby.'
\z

Inflection for gender, as well as number and diminutiveness, also occurs on some adnominal words, including a small subset of adjectives, a subset of demonstratives and two numeral words meaning `one'.%
\footnote{There are five adjectives that inflect for gender: \textit{lapo} `large', \textit{nyopu} `good', \textit{woyue} `bad', \textit{wwe} `bad', and \textit{kolue} `short'. The meanings associated with these correspond closely to the adjectival concepts found in languages with small adjective inventories \citep{Dixon1977}. One might expect to find adjectives meaning `small' or `long' in this set. The Walman adjective for `small', \textit{nngkal}, does not inflect for gender but does for number; the plural form is \textit{nngkam}. The meaning of `long, tall' in Walman is expressed by a sequence of two words \textit{ro rani}, where \textit{ro} exists separately as an adnominal word meaning `piece of' and does inflect for gender, so the two word sequence (feminine \textit{ro rani}, masculine \textit{ron rani}) can be described as functioning as an adjective and hence as a sixth adjective that inflects for gender.} The form of affixes indicating gender, number, or diminutiveness on adnominal words is the same as those used for object affixes on verbs. In (\ref{ex:Dry:15}), for example, we find the masculine affix -\textit{n-} as an infix in the demonstrative \textit{panten} and as a suffix on the adjective \textit{lapon} `big' (here used predicatively).

\ea  \label{ex:Dry:15}
\gll Ngolu	pa<\textbf{n}>ten	\textbf{n}-o	lapo-\textbf{n}.\\
cassowary that\textbf{<\textsc{m}}> \textbf{\textsc{3sg.m}}-be big-\textbf{\textsc{m}}\\
\glt `That cassowary is large.'
\z

Like the third person object affixes on verbs, feminine gender is phonologically null on adnominal words, as illustrated by the feminine forms \textit{paten} `that' in (\ref{ex:Dry:16}) and \textit{lapo} `big' in (\ref{ex:Dry:17}).

\ea \label{ex:Dry:16}
\gll Mon	chi	n-a<ø>ko	wul	\textbf{pa}<\textbf{ø}>\textbf{ten}.\\
\textsc{neg} \textsc{2sg} \textsc{2sg}-eat<\textsc{3sg.f}> water \textbf{that}<\textbf{\textsc{f}}>\\
\glt `You shouldn't drink that water.'
\z

\ea \label{ex:Dry:17}
\gll Wako	\textbf{lapo}-\textbf{ø}	w-ara.\\
boat \textbf{large-\textsc{f}} \textsc{3sg}-come\\
\glt `A big ship has come.'
\z

In (\ref{ex:Dry:18}), we get a plural suffix \textit{-y} on \textit{lapoy} `good'.

\ea  \label{ex:Dry:18}
\gll Nypeykil	\textbf{lapo}-\textbf{y}	y-an	olun	olun. \\
tree.\textsc{pl} \textbf{big-\textsc{pl}} \textsc{3pl}-be.at side side\\
\glt  `There are big trees on both sides of the road.'
\z

There is no gender distinction in the plural. Note that the position of these affixes is similar to the position of corresponding object affixes in being typically suffixes (as in \textit{lapon} in (\ref{ex:Dry:15}) and \textit{nyopuy} in (\ref{ex:Dry:16})), but with some words infixes (as in \textit{panten} in (\ref{ex:Dry:15})).

There are also two words for `one' that inflect for gender, number, and diminutiveness, illustrated by \textit{alpan} `one' in (\ref{ex:Dry:19}).

\ea  \label{ex:Dry:19}
\gll Kamte-n	\textbf{alpa}-\textbf{n}	n-epin	n-ara.\\
person-\textsc{m} \textbf{one-\textsc{m}} \textsc{3sg.m}-go.ahead \textsc{3sg.m}-come\\
\glt `One man came ahead of the others'
\z

  Not all adnominal words inflect. In fact most adjectives do not. For example the adjective \textit{chapa} `fat' is invariant, as illustrated in (\ref{ex:Dry:20}) (where the form would be a masculine form \textit{chapan} if it did inflect).

\ea \label{ex:Dry:20}
\gll Runon	n-o	\textbf{chapa}.\\
\textsc{3sg.m} \textsc{3sg.m}-be \textbf{fat}\\
\glt `He is fat.'
\z

Finally, the third person pronouns themselves vary for number, gender, and diminutiveness, as illustrated by the pronouns for \textsc{3sg.m,} \textit{runon,} in (\ref{ex:Dry:20}) and \textsc{3sg.f}, \textit{ru}, in (\ref{ex:Dry:12}) above.

The only morphology found on nouns is plural marking.%
\footnote{There are a few words that might be analysed as nouns that inflect for gender, since they involve a contrast that is formally identical to gender inflection on many adnominal words. First, there is a noun \textit{kamten} `man' with plural \textit{kamtey} for which we have a few instances of a feminine form \textit{kamte} and a diminutive form \textit{kamtel} in elicited data, but none in texts. Second, there are a few pairs of kin terms differing in that the one denoting a male ends in an /n/ while the corresponding one denoting a female lacks the /n/, like \textit{wlapon} `older brother of a man' and \textit{wlapo} `older sister of a woman'.} %
However, plural marking occurs with a relatively small number of nouns; most nouns lack distinct plural forms. The set of nouns with distinct plural forms includes most kinship terms and a few other nouns denoting humans, plus seventeen inanimate nouns. There seems little way to predict which inanimate nouns have distinct plural forms. Some are nouns denoting body parts (e.g. \textit{kampotu} `knee', plural \textit{kamtikiel}). Others include \textit{nyikie} `piece of wood', plural \textit{nyikiel}; \textit{nymuto} `star', plural \textit{nymteykil}; and \textit{tomuel} `stone', plural \textit{tmleykiel}. The process of plural formation is fairly irregular. There are no plural forms for nouns denoting non-human animals. Whether a noun has a distinct plural form or not has no effect on agreement patterns. For nouns lacking distinct plural forms, differences in number are carried only on agreeing words. For example, what conveys the difference in number in (\ref{ex:Dry:21}) and (\ref{ex:Dry:22}) is the subject prefix on the verb (\textit{w-} for \textsc{3sg} feminine in (\ref{ex:Dry:20}), \textit{y-} for \textsc{3pl} in (\ref{ex:Dry:22})); the form of the noun \textit{pelen} `dog' is the same in the two examples.

\ea \label{ex:Dry:21}
\gll Pelen	\textbf{w}-aykiri.\\
dog \textbf{\textsc{3sg.f}}-bark\\
\glt  `The dog (female) is barking.'
\z

\ea \label{ex:Dry:22}
\gll Pelen	\textbf{y}-aykiri. \\
dog \textbf{\textsc{3pl}}-bark\\
\glt  `The dogs are barking.'
\z

Among other grammatical features of Walman illustrated by the above examples is the fact that the language lacks case marking to distinguish arguments in a clause and the fact that the most frequent word order is SVO (though SOV exists as a not uncommon alternative order). Apart from the subject and object affixes described above, the only other verb morphology is an applicative suffix and a largely obsolete imperative form of verbs.

\section{Principles Of Gender Assignment}
\label{sec:Dry:3}

In (\ref{ex:Dry:23}) is a summary of the principles governing the choice between masculine and feminine gender in Walman.

\ea%23
    \label{ex:Dry:23}
\begin{xlist}
\ex  All nouns denoting humans and some larger animals are either masculine or feminine, depending on the sex of the referent

\ex  All nouns denoting inanimate objects are feminine%
\footnote{As discussed below in \sectref{sec:Dry:3}, there are many nouns denoting inanimate objects which are pluralia tantum nouns. These nouns are neither masculine nor feminine.}

\ex  Nouns denoting a few quasi-animate natural phenomena, such as \textit{nganu} `sun', are masculine

\ex  Nouns denoting most animals appear to have relatively arbitrary gender
\end{xlist}
\z

The first principle, given in (\ref{ex:Dry:23}a), is that all nouns denoting humans and some larger animals can be either masculine or feminine, depending on the sex of the referent.%
\footnote{The only nouns denoting animals for which we have clear evidence on this are the nouns \textit{pelen} `dog' and \textit{wuel} `pig'. There are some other nouns, like \textit{slaoi} `rat', where some instances in our data control masculine agreement and others control feminine agreement, but we need to investigate to determine whether this alternation is governed by the presumed sex of the referent (or some other factors).} %
For example, the noun \textit{pelen} `dog' controls feminine subject agreement in (\ref{ex:Dry:24}), but masculine subject agreement in (\ref{ex:Dry:25}).

\ea \label{ex:Dry:24}
\gll O	pelen	tu	\textbf{w}-ata	ke?\\
and dog \textsc{perf} \textbf{\textsc{3sg.f}}-bite.\textsc{2obj} Q\\
\glt `Did the dog bite you?'
\z

\ea \label{ex:Dry:25}
\gll Kum	wuel	mingrieny	tu	pelen	\textbf{n}-a<y>ko.\\
\textsc{1sg} pig meat \textsc{perf} dog \textbf{\textsc{3sg.m}}-eat<\textsc{3pl}>\\
\glt `My pig's meat has been eaten by the dog.'
\z

Most nouns denoting humans are inherently masculine or feminine, but only because they necessarily denote someone who is male or female respectively. For example, in (\ref{ex:Dry:26}), the noun \textit{ngan} `father' controls masculine subject agreement on \textit{nroko} `take' while \textit{nyue} `mother' controls feminine subject on \textit{wrulu} `cut'.

\ea \label{ex:Dry:26}
\gll Ngan	\textbf{n}-r-oko	rele,	nyue	\textbf{w}-r-ulo woruen.\\
father \textbf{\textsc{3sg.m}}-\textsc{refl}-take beard mother \textbf{\textsc{3sg.f}}-\textsc{refl}-cut hair\\
\glt `The father shaves, the mother trims her hair.'
\z

The second principle is that nouns denoting inanimate objects are feminine. This is illustrated in (\ref{ex:Dry:27}), where \textit{chakonu} `road' controls \textsc{3sg.f} agreement on the verb \textit{wo} `be'.

\ea \label{ex:Dry:27}
\gll Chakonu	\textbf{w}-o	mail.\\
road \textbf{\textsc{3sg.f}}-be crooked\\
\glt `The road is not straight.'
\z

This principle is also illustrated in examples above, for \textit{nakol} `house' in (\ref{ex:Dry:6}), for \textit{opucha} `thing' in (\ref{ex:Dry:9}), and for \textit{wul} `water' in (\ref{ex:Dry:16}).

What could be interpreted as an exception to this principle is stated above in (\ref{ex:Dry:23}c): nouns denoting a few quasi-animate natural phenomena are masculine. This is illustrated for \textit{snar} `moon' in (\ref{ex:Dry:28}), where it controls masculine subject agreement, and for \textit{onyul} `earthquake' in (\ref{ex:Dry:29}), where it controls masculine object agreement.

\ea \label{ex:Dry:28}
\gll Snar	\textbf{n}-reliel.\\
moon \textbf{\textsc{3sg.m}}-shine\\
\glt `The moon is shining.'
\z

\ea \label{ex:Dry:29}
\gll Kum	m-rere-\textbf{n}	onyul	nngkal.\\
\textsc{1sg} \textsc{1sg}-feel-\textbf{\textsc{3sg.m}} earthquake small\\
\glt `I felt a small earthquake.'
\z

There are two other nouns of this sort that consistently control masculine agreement, namely \textit{nganu} `sun' and \textit{knum} `whirlpool, riptide'. Note that \textit{nganu} `sun' can also mean simply `day' and controls masculine agreement with this meaning as well, as in (\ref{ex:Dry:30}), where it controls masculine agreement on the adnominal word \textit{ngon} `one', as reflected by the masculine suffix -\textit{n}.%
\footnote{A chamul is a partly human, partly supernatural being in traditional Walman culture. Example (\ref{ex:Dry:30}) employs an idiom \textit{{}-ekele chamul} `to play a flute to call one's chamul'.}

\ea \label{ex:Dry:30}
\gll Nganu	ngo-\textbf{n}	ru	w-ekele-n	chamul w-ru.\\
sun one-\textbf{\textsc{m}} \textsc{3sg.f} \textsc{3sg.f}-pull-\textsc{3sg.m} Chamul \textsc{gen}-\textsc{3sg.f}\\
\glt `One day she played a flute to call her Chamul.'
\z

There are two other nouns of this sort that can control masculine agreement, but only when they occur in idioms, not when they occur with their literal meaning. One is the noun \textit{olokol} `mountain', which is normally a pluralia tantum noun, controlling plural agreement, as in (\ref{ex:Dry:31}), where it controls plural inflection on \textit{alpay} `one' and \textsc{3pl} subject agreement on the verb \textit{yiliel} `go towards sea'.%
\footnote{Normally \textit{olokol} refers to an entire mountain range, since the salient mountains near Walman-speaking villages are the Torricelli Mountains, a mountain range that is roughly parallel to the coast, where there is not a clear delineation between individual mountains. In (\ref{ex:Dry:31}), however, it is clear from the text this comes from that a single mountain is being referred to.}

\ea \label{ex:Dry:31}
\gll {\ldots}	olokol	alpa-\textbf{y}	konu	\textbf{y}-iliel	Matapau. \\
{} mountain one-\textbf{\textsc{pl}} only \textbf{\textsc{3pl}}-go.seaward Matapau\\
\glt `... there was just one mountain coming down at Matapau.'
\z

However, this noun also occurs with the verb -\textit{oruel} `explode' in an idiom meaning `to thunder', as in (\ref{ex:Dry:32}), and in this idiom it controls masculine subject agreement on the verb.

\ea \label{ex:Dry:32}
\gll Olokol	\textbf{n}-oruel.\\
mountain \textbf{\textsc{3sg.m}}-explode\\
\glt `It thundered.'
\z

In other contexts with the verb -\textit{oruel}, this noun triggers plural subject agreement, but in these cases, the meaning is literal rather than idiomatic, as illustrated in (\ref{ex:Dry:33}).

\ea \label{ex:Dry:33}
\gll Olokol	\textbf{y}-oruel.\\
mountain \textbf{\textsc{3pl}}-explode\\
\glt `The mountain exploded (i.e.\ a volcano).'
\z

The second noun that controls masculine agreement in an idiom but not in its literal meaning is the noun \textit{anako} `sky', which combines either with the verb -\textit{ol} `break' or with the verb \textit{ochoro} `split open' as alternative ways to express the meaning `to thunder', as illustrated with the verb -\textit{ol} in (\ref{ex:Dry:34}).

\ea \label{ex:Dry:34}
\gll Anako	\textbf{n}-ol	komoru.\\
sky \textbf{\textsc{3sg.m}}-break evening\\
\glt `It thundered in the (late) afternoon.'
\z

Outside of this idiom, the noun \textit{anako} `sky' controls feminine agreement, as illustrated in (\ref{ex:Dry:35}).

\ea \label{ex:Dry:35}
\gll Lasi	anako	\textbf{w}-arau	\textbf{w}-orou	wor.\\
immediately sky \textbf{\textsc{3sg.f}}-go.up \textbf{\textsc{3sg.f}}-go high\\
\glt `The sky immediately went high up.'
\z

Although these nouns denote things that are considered inanimate in Western cultures, I characterize them as quasi-animate, since they all denote things that are associated with autonomous movement or force, something generally associated with animate beings. However, not all nouns that might be considered instances of autonomous movement or force control masculine agreement, as illustrated for \textit{loun} `cloud' in (\ref{ex:Dry:36}) and for \textit{nyuey} `sea' in (\ref{ex:Dry:37}), which are both feminine, as reflected by the \textsc{3sg} subject prefixes \textit{w}{}- on the verbs.

\ea \label{ex:Dry:36}
\gll Loun	\textbf{w}-alplo-n	nganu\\
cloud \textbf{\textsc{3sg.f}}-cover-\textsc{3sg.m} sun\\
\glt `The cloud is hiding the sun.'
\z

\ea \label{ex:Dry:37}
\gll Nyuey	\textbf{w}-oko-n	n-orou	\textbf{w}-elie-n {n-ekiel \ldots} \\
sea \textbf{\textsc{3sg.f}}-take-\textsc{3sg.m} \textsc{3sg.m}-go \textbf{\textsc{3sg.f}}-throw-\textsc{3sg.m} \textsc{3sg.m}-go.landward \\
\glt `The sea carried him until it threw him up on the beach ...'
\z

Another noun, \textit{chepili} `thunder, lightning', always controls plural agreement, as in (\ref{ex:Dry:38}), where it controls \textsc{3pl} subject agreement on \textit{yol} `break', \textit{yanan} `go down' and \textit{yaypu} `kill'.%
\footnote{The first three words in (\ref{ex:Dry:37}) constitute an idiom meaning `for there to be lightning', where the literal meaning is `it shoots women'. Note that this idiom obligatorily has the \textsc{3sg.f} pronoun \textit{ru} as subject.}

\ea \label{ex:Dry:38}
\gll Ru	w-ao-y	nyiki,	lasi	chepili \textbf{y}-ol	mpang,	\textbf{y}-anan,	\textbf{y}-a<y>pu kamte-y	eni	y-a<ø>ko	wkaray w-aro-ø	ngotu,	y-alma	mpor.\\
\textsc{3sg.f} \textsc{3sg.f}-shoot-\textsc{3pl} woman.\textsc{pl} immediately  thunder \textbf{\textsc{3pl}}-break loud.noise \textbf{\textsc{3pl}}-go.down \textbf{\textsc{3pl}}-kill<\textsc{3pl}>  person-\textsc{pl} \textsc{rel} \textsc{3pl}-eat<\textsc{3sg.f}> white.cuscus
\textsc{3sg.f}-and-\textsc{3sg.f} coconut \textsc{3pl}-die all\\
\glt `There was lightning and immediately thunder cracked ``mpang'' and came down and killed all the people who had eaten the cuscus with coconut.'
\z

The only nouns in Walman for which gender appears to be arbitrarily assigned are those denoting other animals, especially non-mammals. For example, \textit{alan} `red and green parrot' is masculine, as reflected by the masculine subject prefixes on the verbs \textit{nka} `fly' and \textit{nekiel} `go inland, go towards land' in (\ref{ex:Dry:39}).

\ea \label{ex:Dry:39}
\gll Alan	yapa	\textbf{n}-ka	\textbf{n}-ekiel.\\
parrot that \textbf{\textsc{3sg.m}}-fly \textbf{\textsc{3sg.m}}-go.landward\\
\glt  `That parrot is flying inland.'
\z

Similarly \textit{wraul} `toad' is feminine, as reflected in (\ref{ex:Dry:40}) by the feminine object agreement on \textit{nete} `see', the feminine agreement on the adjective \textit{lapo} `large', and the feminine subject agreement on \textit{wekele} `make'.

\ea \label{ex:Dry:40}
\gll Lasi	runon	Tenten	n-ete-\textbf{ø}	wraul lapo-\textbf{ø}	oluel	\textbf{w}-ekele	w-an kra	nyumuen.\\
immediately \textsc{3sg.m} Tenten \textsc{3sg.m}-see-\textbf{\textsc{3sg.f}} toad big-\textbf{\textsc{f}} nest \textbf{\textsc{3sg.f}}-make \textsc{3sg.f}-be.at sugarcane middle\\
\glt  `A man Tenten suddenly saw a large toad making a nest in the middle of the sugarcane.'
\z

For a number of reasons, it is not really possible to demonstrate convincingly that gender is arbitrary for most animals. First, for many species, we have not actually seen instances of the animals, but depend on descriptions by speakers. Second, one can never know for sure whether there are unknown characteristics of particular animals that play a role in determination of gender (such as size, sound, or behaviour). And third, there may be roles that animals play in Walman culture and history that we are not aware of that influence gender. In general, however, native speakers do not have explanations for particular gender assignment for these nouns.

The lack of an obvious semantic basis for gender assignment for animals can be illustrated by looking at the gender of nouns denoting various species of snakes. In (\ref{ex:Dry:41}), I list the genders for the six nouns (or two-word nominal expressions) in our data denoting different species of snake.

\ea   \label{ex:Dry:41}
Snakes

\begin{tabularx}{0.9\textwidth}{lp{1.8cm}X}
\multicolumn{3}{l}{\textsc{Masculine}} \\
& \textit{ani konu} & snake, light brown-orange-red, about a metre long, very dangerous\\
& \textit{nayko iyoy} & small snake, lives along coast, not dangerous, eats crabs\\
& \textit{layat} & type of python, very big and long, pretty patterned skin, lives in trees, not really dangerous\\
\end{tabularx}

\begin{tabularx}{0.9\textwidth}{lp{1.8cm}X}
\multicolumn{3}{l}{\textsc{Feminine}}\\
  & \textit{kilekile} & death adder, about a foot long, black with white dots, very dangerous\\
& \textit{mekey} & ground python, brown with white belly, not poisonous, can be very big\\
& \textit{nyieu} & very big and long, light blue and shiny, lives in bush, not dangerous to people but swallows small animals\\
\end{tabularx}
\z

Two obvious differences among snakes that might play a role in determining gender are size and how dangerous they are (defined by how serious their snake bite is). The list of snakes in (\ref{ex:Dry:41}) includes three pythons, which share the features of being large and not being dangerous: two are masculine, while one is feminine. Of the three smaller snakes, two are very dangerous: one of these is masculine, the other feminine. Thus neither size nor how dangerous they are provides a basis for predicting gender. There may be other factors, of course, but the most obvious ones do not seem relevant. Note that \textit{ani konu} is literally `male snake', so the masculine gender for this two-word nominal expression is explained by the fact that \textit{konu} means `male'. In addition the first word in \textit{nayko iyoy} is a form that looks like a form of the verb -\textit{ako} `eat', with a \textsc{3sg.m} prefix and a \textsc{3pl} object infix, while the second word (\textit{iyoy}) is a noun meaning `crab' so that the apparent literal meaning of \textit{nayko iyoy} is `he eats crabs'; thus the fact that \textit{nayko} begins with what looks like a \textsc{3sg.m} subject prefix may be relevant to the fact that this snake is masculine.

We find a similar situation with insects and similar lower animals. The list in (\ref{ex:Dry:42}) is a list of all the species of such animals in our data (excluding a few whose gender we lack data on).

\ea    \label{ex:Dry:42}
Insects and the like (spiders, lice, leeches, worms, centipedes, millipedes)\\

\begin{tabularx}{0.9\textwidth}{lp{1.8cm}X}
\multicolumn{3}{l}{\textsc{Masculine}}\\
& \textit{achakol} & housefly\\
& \textit{kayikiel} & fruitfly\\
& \textit{kaimung} & firefly\\
& \textit{kanal} & sago grub\\
& \textit{melkil} & bee, wasp\\
& \textit{mile} & leech\\
& \textit{paral tkay} & flying ant\\
& \textit{ppu} & small green or brown grasshopper-like creature\\
& \textit{slmako} & bluebottle fly\\
& \textit{srnyako} & beetle which comes around in evening, makes loud sound\\
& \textit{tmpinie} & worm (general term)\\
\end{tabularx}

\begin{tabularx}{0.9\textwidth}{lp{1.8cm}X}
\multicolumn{3}{l}{\textsc{Feminine}}\\
& \textit{atal} & scorpion\\
& \textit{inrer} & very small mosquito (hard to see, smaller than sandfly) that bites people in evening, especially in marshy areas of bush\\
& \textit{klu} & `fly which is very tiny and which makes nest in holes in wood'\\
& \textit{krunu} & centipede\\
& \textit{nymuchuto} & spider\\
& \textit{nymulol} & louse\\
& \textit{paral} & ant\\
& \textit{pirinyue} & cockroach\\
& \textit{posur} & termite, white ant (does not live in houses, builds mounds)\\
& \textit{puseksek} & a type of grasshopper that is large, brown or green, and can fly, and which make a noise like ``seksek''\\
& \textit{waykelie} & millipede\\
& \textit{woru} & mosquito\\
\end{tabularx}
\z

The nouns listed in (\ref{ex:Dry:42}) denoting species which bite or sting humans include three masculine nouns (\textit{melkil} `bee, wasp', \textit{mile} `leech', and \textit{paral tkay} `flying ant') and seven feminine nouns (\textit{atal} `scorpion', \textit{inrer} `very small mosquito', \textit{krunu} `centipede', \textit{nymuchuto} `spider', \textit{nymulol} `louse', \textit{paral} `ant', and \textit{woru} `mosquito'), so being something that bites or stings is not a predictor of gender. Of the two species whose stings are most painful, one is masculine (\textit{melkil} `bee, wasp') while the other is feminine (\textit{krunu} `centipede', the local variety of which reportedly has an especially painful sting). Of the smaller species in (\ref{ex:Dry:42}), one is masculine (\textit{kayikiel} `fruit fly') while three are feminine (\textit{inrer} `very small mosquito', \textit{klu} `very tiny fly', and \textit{woru} `mosquito'). Nor is there any other obvious feature distinguishing the masculine nouns in (\ref{ex:Dry:42}) from the feminine nouns.

If there is any feature that correlates at least weakly with gender among other  animals, it is that nouns denoting more aggressive species are somewhat more often masculine while nouns denoting less aggressive species are somewhat more often feminine. A correlation with aggressiveness seems most apparent with species of birds, listed in (\ref{ex:Dry:43}).

\ea
\label{ex:Dry:43}
Birds \\*
\begin{tabularx}{0.9\textwidth}{lp{1.8cm}X}
\multicolumn{3}{l}{ \textsc{Masculine}}\\
& \textit{alan} & red and green parrot\\
& \textit{aron} & eagle that is large and grey and white and that is found in the jungle\\
& \textit{mmpul} & hawk with reddish brown body and white head\\
& \textit{ngolu} & cassowary\\
& \textit{semier} & type of bush fowl\\
& \textit{tarkau} & osprey\\
& \textit{tualiau} & type of bush fowl, brown, small, the size of a chicken\\
& \textit{wamol} & hornbill\\
& \textit{wawiel} & crow\\
& \textit{yiwos} & very small hawk, brown, lives at coast\\
\end{tabularx}

\begin{tabularx}{0.9\textwidth}{lp{1.8cm}X}
\multicolumn{3}{l}{ \textsc{Feminine}}\\
& \textit{kmaynum} & blue bird about the size of a chicken, has no decoration\\
& \textit{le} & bird of paradise\\
& \textit{pinie} & tiny bird, blue with white around neck\\
& \textit{polmonu} & guria pigeon\\
& \textit{rampanyau} & willy wagtail\\
& \textit{solponyou} & swallow\\
& \textit{yup} & white cockatoo\\
\end{tabularx}
\z

All of the nouns denoting what I believe are the most aggressive species are masculine: \textit{nganu} (`cassowary'), \textit{aron} (a type of eagle), \textit{mmpul} (a type of hawk), \textit{yiwos} (another type of hawk), \textit{tarkau} (`osprey'), and \textit{wawiel} (`crow').

Most of the nouns denoting aquatic animals are feminine. This includes nine out of twelve species of fish, two species of crab, crayfish, and two aquatic mammals (\textit{alpariak} `dolphin', \textit{yuel} `seal'). One of the three masculine nouns for a species of fish is the noun \textit{wuey} for `shark', which fits the weak correlation between aggressiveness and masculine gender. There is one noun, \textit{nyelekel}, that can denote either of two species of snail. This noun is masculine when it denotes one species, feminine when it denotes the other species. The feminine one lives in water, while the masculine one apparently does not.

Some nouns denoting larger animals can be either masculine or feminine, but one of the two genders is the default. While it is apparently the case that the default gender is generally used when the sex of the referent is unknown, this is not always the case. For example the default gender of the noun \textit{ngolu} `cassowary' is masculine and although it can be feminine when the referent is female, feminine gender is not obligatory when the referent is clearly female. In (\ref{ex:Dry:44}), for example, this noun controls masculine subject agreement on the verb, despite the fact that the semantics of the sentence implies that the referent is female.

\ea \label{ex:Dry:44}
\gll \textbf{Ngolu}	\textbf{n}-ikie-ø	meten.\\
\textbf{cassowary} \textbf{\textsc{3sg.m}}-put-\textsc{3sg.f} egg\\
\glt `A cassowary has lain an egg.'
\z

However, this noun can be feminine, as in (\ref{ex:Dry:45}), where it controls feminine object agreement.\footnote{The possibility of feminine agreement in (\ref{ex:Dry:45}) may be due to the fact that it is the meat (i.e.\ an inanimate object) that is being denoted here, rather than the living bird. However we have more than one other instance in our data of a noun phrase denoting cassowary meat triggering masculine agreement.}

\ea \label{ex:Dry:45}
\gll ...	y-e<\textbf{ø}>tiki	\textbf{ngolu}.\\
{} \textsc{3pl}-cook.over.fire\textbf{<\textsc{3sg.f}}> \textbf{cassowary}\\
\glt `[She is still with her brothers] cooking (a) cassowary.'
\z

There are a few other uses of masculine gender in Walman that are more unusual. For example, the noun \textit{won} can mean `chest', but it is far more common as part of a large number of idioms where this meaning is less evident. In its meaning `chest', it is feminine, as in (\ref{ex:Dry:46}).

\ea \label{ex:Dry:46}
\gll Won	mnon	\textbf{w}-o	lapo-\textbf{ø}.\\
chest \textsc{3sg.m:gen} \textbf{\textsc{3sg.f}}-be big-\textbf{\textsc{f}}\\
\glt `His chest is large.'
\z

When \textit{won} occurs in idioms, it is masculine, as in (\ref{ex:Dry:47}) and (\ref{ex:Dry:48}), where in both cases \textit{won} controls masculine subject agreement on the verb. The idiom in (\ref{ex:Dry:47}) for `angry' is literally `heart be fast'.

\ea \label{ex:Dry:47}
\gll Ru	won	\textbf{n}-o	kisiel	prie.\\
\textsc{3sg.f} heart \textbf{\textsc{3sg.m}}-be fast completely\\
\glt `She is very angry.'
\z

I gloss \textit{won} in idioms as `heart', not in the sense of the body part, but in a more abstract sense that could alternatively be glossed `mind' or `soul'. One reason that I gloss it as `heart' is that it is clearly cognate to the word for the body part heart in a number of other languages in the Torricelli family.

The idiom in (\ref{ex:Dry:48}) for `be happy' is literally `heart follows', where the one who is happy is grammatically the object of the verb, as reflected by the \textsc{3pl} object suffix on the verb. Note that the object pronoun \textit{ri} in (\ref{ex:Dry:48}) is clause-initial; the normal word order in this and a couple of other idiomatic constructions with an inanimate subject and an animate object is OSV.

\ea    \label{ex:Dry:48}
\gll Ri	won	\textbf{n}-rowlo-\textbf{y}.\\
\textsc{3pl} heart \textbf{\textsc{3sg.m}}-follow-\textbf{\textsc{3pl}}\\
\glt `They are happy.'
\z

In (\ref{ex:Dry:49}), \textit{won} functions as the object of the verb in an idiom meaning `take a deep breath' (literally `pulls heart hard'); in this idiom, the verb obligatorily occurs with masculine object inflection, agreeing with \textit{won}.

\ea \label{ex:Dry:49}
\gll Kum	won	m-ekele-\textbf{n}	tetiet.\\
\textsc{1sg} heart \textsc{1sg}-pull-\textbf{\textsc{3sg.m}} hard\\
\glt `I took a deep breath.'
\z

Another word that is feminine in its literal meaning but masculine in idioms is \textit{puna} `brain'. In (\ref{ex:Dry:50}), \textit{puna} controls feminine subject agreement in its literal meaning, while in (\ref{ex:Dry:51}), it controls masculine object agreement in an idiom -\textit{ekelen puna} `to snore' (literally `to pull one's brain').\footnote{Note that in all the examples I have discussed where a noun is a different gender in an idiom from its gender outside of idioms are cases where the noun is masculine in the idiom but feminine outside of idioms. This appears to be due to the fact that the relevant nouns denote inanimate objects outside of idioms and thus are feminine outside of idioms.}

\ea \label{ex:Dry:50}
\gll Kum	puna	\textbf{w}-o	cheliel.\\
\textsc{1sg} brain \textbf{\textsc{3sg.f}}-be hot\\
\glt `My brain hurts.'
\z

\ea \label{ex:Dry:51}
\gll Chi	n-ekele-\textbf{n}	puna	kisiel.\\
\textsc{2sg} \textsc{2sg}-pull-\textbf{\textsc{3sg.m}} brain fast/loud\\
\glt `You were snoring loudly.'
\z

A final instance of a word that is obligatorily masculine is the interrogative pronoun \textit{mon} `who', illustrated in (\ref{ex:Dry:52}). It is not possible to use a verb form \textit{chaltawro} in (\ref{ex:Dry:52}), with \textsc{3sg.f} object agreement, even in contexts where it is assumed that someone is looking for a woman, although \textsc{3pl} agreement would be possible if it is assumed that more than one person is being looked for.

\ea \label{ex:Dry:52}
\gll Chim	ch-altawro-\textbf{n}	mon?\\
\textsc{2pl} \textsc{2pl}-look-\textbf{\textsc{3sg.m}} who\\
\glt `Who are you looking for?'
\z

\textit{Mon} thus behaves as a masculine noun.\footnote{There is no interrogative pronoun in Walman meaning `what'. Rather, there is an interrogative adnominal word \textit{mol} and and the expression for `what' is \textit{opucha mol} literally `what thing'. The gender of noun phrases with \textit{mol} is determined by the gender of the noun (or the sex of the referent).}

\section{Pluralia tantum nouns}
\label{sec:Dry:4}

I analyse nouns in Walman which are always grammatically plural as pluralia tantum nouns (\citealt[233ff]{Corbett2012}; \citealt{Acquaviva2008}).%
\footnote{Many linguists distinguish a singular expression \textit{plurale tantum} from a plural expression \textit{pluralia tantum}. But there is considerable inconsistency in the literature in the use of these expressions, so I avoid the expression \textit{plurale tantum} and urge other linguists to do likewise. In this paper, I treat the expression \textit{pluralia tantum} as grammatically similar to the words \textit{masculine} and \textit{feminine}.} %
While the category of pluralia tantum nouns in other languages is not usually considered a gender, what makes it gender-like in Walman is the sheer number of pluralia tantum nouns. In our current data, there are about twice as many pluralia tantum nouns as there are masculine nouns.%
\footnote{Our current data includes 81 instances of pluralia tantum nouns, but only 40 instances of masculine nouns. Since there are a number of nouns denoting animals whose gender we have not yet had opportunity to check, it is likely that the ratio of pluralia tantum nouns to masculine nouns will be less than 2 to 1.} %
What this means is that apart from nouns which can be either masculine or feminine depending on the sex of the referent, every noun in Walman is masculine, feminine, or pluralia tantum. In this sense, pluralia tantum is like a gender.

In many languages, what characterizes pluralia tantum nouns is that they are plural in form (e.g., \textit{scissors} in English). In Walman, however, what characterizes pluralia tantum nouns is not their form, but the fact that they always trigger plural agreement. An example of a pluralia tantum noun is \textit{nyi} `fire'. In (\ref{ex:Dry:53}), it triggers \textsc{3pl} subject agreement on the verb \textit{yiri} `stand up, rise' and \textit{yreliel} `shine, for a fire to blaze'.

\ea \label{ex:Dry:53}
\gll Nyi	\textbf{y}-iri	pa,	nyi	\textbf{y}-reliel.\\
fire \textbf{\textsc{3pl}}-stand.up \textsc{ptcl} fire \textbf{\textsc{3pl}}-shine\\
\glt `The fire rose, it was ablaze.'
\z

In (\ref{ex:Dry:54}), the same noun triggers \textsc{3pl} object agreement on \textit{noysusur} `move' and \textsc{3pl} subject agreement on \textit{yesi} `go outside'.

\ea \label{ex:Dry:54}
\gll Runon	n-o<\textbf{y}>susur	nyi	\textbf{y}-esi	chalien.\\
\textsc{3sg.m} \textsc{3sg.m}-move\textbf{<\textsc{3pl}}> fire \textbf{\textsc{3pl}}-go.outside outside\\
\glt `He moved the fire outside.'
\z

And in (\ref{ex:Dry:55}), the same noun triggers \textsc{3pl} object agreement on the verb \textit{kaoy} `shoot' (here used in the sense of `light' in `light a fire'), as well as plural agreement on the numeral \textit{ngony} `one'.

\ea \label{ex:Dry:55}
\gll Kipin	k-ao-\textbf{y}	nyi	ngo-\textbf{ny}.\\
\textsc{1pl} \textsc{1pl}-shoot-\textbf{\textsc{3pl}} fire one-\textbf{\textsc{pl}}\\
\glt `We lit a fire.'
\z

In (\ref{ex:Dry:56}), the pluralia tantum noun \textit{apar} `platform, shelf, bed' triggers plural agreement on the demonstrative \textit{payten} and \textsc{3pl} subject agreement on the verb \textit{yo} `be'.

\ea  \label{ex:Dry:56}
\gll Apar	pa<\textbf{y}>ten	\textbf{y}-o	rachi.\\
bed that\textbf{<\textsc{pl}}> \textbf{\textsc{3pl}}-be strong\\
\glt `That bed is strong.'
\z

Just as there are semantic factors that partially account for gender in Walman, there are also semantic factors that probably account for at least some pluralia tantum nouns in Walman. Like pluralia tantum nouns in many languages, there is something about many pluralia tantum nouns in Walman that can be conceived as denoting more than one thing. In the case of \textit{nyi} `fire', there are multiple flames. In the case of \textit{apar} `bed, shelf', there are multiple pieces of wood. Other pluralia tantum nouns that denote objects that contain multiple pieces of wood include \textit{chauchau} `door', \textit{salriet} `steps', and \textit{watakol} `raft, coffin'. Pluralia tantum nouns that contain multiple threads (or similar material) include \textit{chrikiel} `net', \textit{ranguang} `clothes' and \textit{kmem} `rope for tying logs together to form a raft'. The noun \textit{tim} `dew' is pluralia tantum and could be construed as involving multiple drops. The noun \textit{yikiel} `language, story, statement, word' is pluralia tantum and one could think of most of these uses as involving multiple words.

However, there are many nouns that can just as easily be conceived of as denoting something with multiple pieces that are not pluralia tantum nouns, including \textit{yie} `bilum, string bag', \textit{wuwu} `basket made from spines of nipa palm fronds for trapping fish', and \textit{amen} `type of basket made from coconut leaves, used for fishing'. Conversely, there are pluralia tantum nouns where it is less obvious that they consist of multiple instances of something, such as \textit{nganyi} `urine',  \textit{almat} `fog', \textit{ei} `lime (white powder produced from grinding up shells, used when chewing betelnut)'. All three of these nouns are mass nouns, but mass nouns do not appear to be pluralia tantum nouns with any greater frequency than count nouns. For example, \textit{wul} `water' and \textit{tantan} `sand' are mass nouns, but are grammatically feminine (as illustrated for \textit{wul} `water' in (\ref{ex:Dry:16}) above by the feminine object agreement on the verb \textit{nako} `eat' and the feminine form of the demonstrative \textit{paten}).

One of the more interesting classes of pluralia tantum nouns are ones denoting body parts. The majority of these nouns denote body parts that occur in pairs. However, these nouns trigger plural morphology even when only one of the two parts is denoted, as in (\ref{ex:Dry:57}), where \textit{chkuel} `eye' triggers plural agreement on both \textit{ngony} `one' and \textit{yo} `be'.

\ea  \label{ex:Dry:57}
\gll Chi	chkuel	ngo-\textbf{ny}	tu	y-o	ngul.\\
\textsc{2sg} eye one-\textbf{\textsc{pl}} \textsc{perf} \textsc{3pl}-be blind\\
\glt `One of your eyes is blind.'
\z

Other pluralia tantum nouns denoting body parts that occur in pairs include \textit{kam} `lungs', \textit{kayal} `foot', \textit{kawa} `heel', \textit{kopun} `buttock', \textit{nyiminy} `breast', \textit{wi} `palm of hand, hand not including fingers', \textit{mkuel} `ear', and \textit{wili} `shoulder'. However, some pluralia tantum nouns refer to body parts that are not normally regarded as paired, such as \textit{repicha} `mouth', \textit{chpurum} `upper lip', \textit{saykil} `liver', \textit{ngoul} `womb' and \textit{kal} `afterbirth'. There are also some body part nouns in Walman which occur in pairs but which are not pluralia tantum nouns; however, in each case, these are nouns that have distinct plural forms, such as \textit{kampotu} `knee' (plural \textit{kamtikiel}).

Note that while pluralia tantum nouns can be conceived of as denoting things with multiple parts, they can still denote single objects, that is, single objects with multiple parts. In other words, they can be semantically singular, as reflected by the fact that they can be modified by either of two words meaning `one' with plural inflection, as in (\ref{ex:Dry:58}) and (\ref{ex:Dry:59}), as well as (\ref{ex:Dry:31}), (\ref{ex:Dry:55}) and (\ref{ex:Dry:57}) above.

\ea \label{ex:Dry:58}
\gll Kum	ranguang	alpa-\textbf{ny}.\\
\textsc{1sg} clothing one-\textbf{\textsc{pl}}\\
\glt `I have one shirt.'
\z

\ea \label{ex:Dry:59}
\gll Kum	m-oko-y	chrikiel	ngo-\textbf{ny}.\\
\textsc{1sg} \textsc{1sg}-take-\textsc{3pl} net one-\textbf{\textsc{pl}}\\
\glt `I brought one net.'
\z

Some nouns are optionally pluralia tantum. For example, the noun \textit{tokun} `knot' can be used with singular agreement to denote a single knot, but with plural agreement to denote either a single knot or more than one knot. Some nouns are pluralia tantum with one sense, but not with another. For example, the noun \textit{wukul} denotes either the sail of a boat or the soft bark flap of coconut tree, which is like a cloth and which is used to strain the sago dust out of the water in making sago. It is pluralia tantum with the first of these senses, but not with the second. A more complex example is illustrated by the noun \textit{kiri}, which means either `sago flour' or `sago pancake'. On the first of these meanings, it is optionally pluralia tantum, while on the second it is always pluralia tantum. This is particularly interesting since it is semantically a mass noun with the first sense, but a count noun with the second; one might have expected it to be more likely pluralia tantum when a mass noun.

In the preceding section, I described a few nouns which are masculine in certain idioms but feminine outside of idioms. We are also aware of at least one case of a noun which does not occur outside of idioms, but which is feminine in one idiom but pluralia tantum in two other idioms. The word \textit{apum} combines with \textit{kakol} `skin' to mean `body', as in (\ref{ex:Dry:60}), where \textit{loyol apum kakol wru} `a sugar-glider's body' triggers feminine agreement on the verb \textit{wo} `be'.

\ea \label{ex:Dry:60}
\gll Loyol	apum	kakol	w-ru	\textbf{w}-o nngkal-nngkal,	chei	w-ru	ro-ø	rani.\\
sugar.glider body skin \textsc{gen}-\textsc{3sg.f} \textbf{\textsc{3sg.f}}-be small-small tail \textsc{gen}-\textsc{3sg.f} piece-\textsc{f} long\\
\glt `A sugar-glider's body is small but its tail is long.'
\z

However, the same word \textit{apum} occurs in two idioms where it behaves as a pluralia tantum noun, controlling plural subject agreement on the verb. One of these idioms, \textit{apum yo sopuer} `to feel tired', is illustrated in (\ref{ex:Dry:61}), while the other, \textit{apum yo mayay} `to feel ashamed', is illustrated in (\ref{ex:Dry:62}).%
\footnote{The adjectives \textit{sopuer} `tired' and \textit{mayay} `ashamed' can also be used with the experiencer as subject, as illustrated in (i) for \textit{sopuer} `tired'.

\begin{exe}
\exi{(i)}
{
\gll Kum	m-o	sopuer\\
\textsc{1sg} \textsc{1sg}-be tired\\
\glt `I'm tired.'
}
\end{exe}

\noindent We do not know if there is a difference in meaning between these non-idiomatic uses of these adjectives and the idioms in (\ref{ex:Dry:61}) and (\ref{ex:Dry:62}).}

\ea \label{ex:Dry:61}
\gll Kum	apum	\textbf{y}-o	sopuer.\\
\textsc{1sg} body \textbf{\textsc{3pl}}-be tired\\
\glt  `I am feeling lethargic.'
\z

\ea \label{ex:Dry:62}
\gll Runon	apum	\textbf{y}-o	mayay.\\
\textsc{3sg.m} body \textbf{\textsc{3pl}}-be shy\\
\glt  `He feels ashamed.'
\z

The idiomatic uses in (\ref{ex:Dry:61}) and (\ref{ex:Dry:62}) involve psychological states while the use in (\ref{ex:Dry:60}) does not. This is probably not a coincidence since the idioms in (\ref{ex:Dry:61}) and (\ref{ex:Dry:62}) resemble the idioms in (\ref{ex:Dry:47}) and (\ref{ex:Dry:48}), where the noun \textit{won} `heart' controls masculine agreement and the meaning involves psychological states.

There are also a few nouns which are singularia tantum nouns that do not appear to be mass nouns. One such noun is \textit{woru} `mosquito', which always triggers feminine singular agreement, as in (\ref{ex:Dry:64}), where it controls feminine singular subject agreement on the verb \textit{wanpu} `attack'.

\ea    \label{ex:Dry:64}
\gll Kon	woru	chomchom	\textbf{w}-a<n>pu.\\
night mosquito many/much \textbf{\textsc{3sg.f}}-attack<\textsc{3sg.m}>\\
\glt `At night, many mosquitoes bit him.'
\z

While examples like (\ref{ex:Dry:64}) are consistent with \textit{woru} being a mass noun, the meaning of (\ref{ex:Dry:65}), where \textit{woru} functions as object of \textit{mkawlo} `count', but still triggers singular agreement, implies that it is a count noun.

\ea \label{ex:Dry:65}
\gll Kum	m-kawlo-\textbf{ø}	woru.\\
\textsc{1sg} \textsc{1sg}-count-\textbf{\textsc{3sg.f}} mosquito\\
\glt  `I counted the mosquitoes.'
\z

While pluralia tantum in Walman behaves in some ways like a gender, I make no claim that it \textit{is} a gender, though I am not aware of any strong arguments against this position. Note that if we were to consider pluralia tantum a gender, I would not be suggesting that plural is a gender, only that the forms used with pluralia tantum nouns are the same as those used for all plurals regardless of gender. A more detailed description of the kinds of nouns that are often pluralia tantum in Walman is given in \citet{DryerInpreparation}.

\section{Diminutive}
\label{sec:Dry:5}

In this section, I describe the Walman diminutive, illustrated in (\ref{ex:Dry:7}) above, and discuss ways in which it is both like and not like a gender.%
\footnote{My discussion in this section is brief since I discuss the Walman diminutive in more detail in \citet{DryerUnderrevision} and \citet{Dryer2016}.} %
\citet[149]{Corbett2012} argues that the Walman diminutive is indeed a gender, though a non-canonical one. In \citet{Dryer2016}, I discuss possible reasons not to consider it a gender.

Unlike diminutives in most languages, the Walman diminutive is inflectional (rather than derivational) in that diminutive affixes occur in the same morphological positions as affixes coding gender and number. In (\ref{ex:Dry:66}), for example, we get diminutive subject prefixes on the verbs \textit{lan} `be at' (here functioning as a progressive auxiliary verb) and \textit{loruen} `cry'.

\ea \label{ex:Dry:66}
\gll Nyanam	nngkal	pa	\textbf{l}-an	\textbf{l}-oruen.\\
child small that \textbf{\textsc{3.dimin}}-be.at \textbf{\textsc{3.dimin}}-cry\\
\glt `The small child was crying.'
\z

And in (\ref{ex:Dry:67}), we get diminutive agreement on the demonstrative \textit{palten}, on the verb \textit{lo} `be' and on the adjective \textit{lapol} `large'.

\ea \label{ex:Dry:67}
\gll Pelen	pa<\textbf{l}>ten	\textbf{l}-o	lapo-\textbf{l}.\\
dog that<\textbf{\textsc{dimin}}> \textbf{\textsc{3.dimin}}-be large-\textbf{\textsc{dimin}}\\
\glt `That puppy is large.'
\z

All words that can inflect for gender and number can also inflect for diminutiveness.

What makes diminutive significantly different from masculine and feminine gender is that there are no nouns that are lexically diminutive, that is, there are no nouns which obligatorily trigger diminutive agreement.%
\footnote{There is one word that may be (or may be considered) lexically diminutive that I discuss in \citet{DryerUnderrevision}, viz.\ \textit{kamtel}, the diminutive form of \textit{kamten} `man'. However, as discussed in \citep{DryerUnderrevision}, there are reasons to consider this the diminutive form of a single lexical item rather than a distinct lexical item.} %
In principle, any noun can be associated with diminutive agreement. For example, the noun \textit{chu} `wife' is normally feminine, but in (\ref{ex:Dry:68}), it triggers diminutive subject agreement on the verb \textit{lalma} `die' in the relative clause \textit{ni lalma pa} `who died there' modifying \textit{chu}.

\ea \label{ex:Dry:68}
\gll Runon	n-akrowon	chu	ni	\textbf{l}-alma	pa.\\
\textsc{3sg.m} \textsc{3sg.m}-think wife \textsc{rel} \textbf{\textsc{3.dimin}}-die there\\
\glt `He mourned his dear wife who had died there.'
\z

The semantics associated with the Walman diminutive is similar to the semantics associated with derivational diminutives in other languages. It can simply denote a smaller size than normal, as in (\ref{ex:Dry:69}), where it triggers diminutive object agreement on the verb \textit{malwul} `buy'.

\ea \label{ex:Dry:69}
\gll Kum	m-a<\textbf{l}>wul	selenyue.\\
\textsc{1sg} \textsc{1sg}-buy\textbf{<\textsc{3.dimin}}> axe\\
\glt `I bought a small axe.'
\z

However, it more often denotes the young of a species, as in (\ref{ex:Dry:66}) and (\ref{ex:Dry:67}) above, or expresses endearment, as in (\ref{ex:Dry:68}) above.

Apart from the fact that there are apparently no lexically diminutive nouns in Walman, another reason for thinking that the Walman diminutive is not a gender is that one can get agreement mismatches in the sense that one target of agreement for a given controller is masculine or feminine while another target of the same controller is diminutive, suggesting that a given noun phrase can be masculine or feminine but at the same time diminutive. For example, in (\ref{ex:Dry:71}), the noun phrase \textit{wuel woyuel} `the naughty pig' is masculine, triggering masculine subject agreement on the verb \textit{narul} `run away', but at the same time diminutive in that the adjective \textit{woyuel} `bad' exhibits diminutive inflection.

\ea \label{ex:Dry:71}
\gll Wuel	woyue-\textbf{l}	\textbf{n}-arul.\\
pig bad-\textbf{\textsc{dimin}} \textbf{\textsc{3sg.m}}-run.away\\
\glt `The naughty little male pig ran away.'
\z

The reverse is also possible, with masculine inflection on the adjective and diminutive agreement on the verb, as in (\ref{ex:Dry:72}).

\ea \label{ex:Dry:72}
\gll Wuel	woyue-\textbf{n}	\textbf{l}-arul.\\
pig bad-\textbf{\textsc{masc}} \textbf{\textsc{3.dimin}}-run.away\\
\glt `The naughty little male pig ran away.'
\z

Whether the Walman diminutive should be treated as a gender is a complex question and depends to a large extent how one interprets the question, as discussed by \citet{Dryer2016}. For more detailed description of the Walman diminutive, see \citet{DryerUnderrevision} and \citet{DryerInpreparation}.

\section{Conclusion}

In this paper, I have described gender in Walman. The choice between the two clear instances of gender, masculine and feminine, is largely predictable semantically, though this is partly due to the fact that inanimate nouns are always feminine. The only nouns whose gender is apparently arbitrary are ones denoting animals. I have also briefly described two other gender-like phenomena in Walman, pluralia tantum and diminutive. I do not take a stand here on whether these two phenomena are genders or not. My goal has simply been to illustrate ways in which they are gender-like and ways in which they are not gender-like. In the case of pluralia tantum nouns, they are more gender-like than similar categories in other languages, simply because there are so many of them. In the case of the diminutive, it is like a gender to the extent that it is coded in the same morphological positions as masculine and feminine, but not like a gender in that there appear to be no lexically diminutive nouns.

\section*{Acknowledgments}
I acknowledge funding supporting field work by myself and Lea Brown on Walman from the Endangered Languages Documentation Programme and from the National Science Foundation (in the United States). We began field work on Walman in 2001 and are currently preparing a detailed description of the language \citep{DryerInpreparation}. See \citet{Brown2008} for some basic features of Walman. I am indebted to anonymous reviewers and particularly to Lea Brown for comments on an earlier draft of this paper.

\section*{Abbreviations}

\begin{tabular}{llll}
%\lsptoprul}e
 \textsc{dimin} & diminutive& \textsc{pl} & plural\\
 \textsc{f} & feminine &  \textsc{ptcl} & particle\\
 \textsc{fut} & future&  Q & marker of polar question\\
 \textsc{gen} & genitive&  \textsc{recip} & reciprocal\\
 \textsc{m} & masculine&  \textsc{refl} & reflexive\\
 \textsc{neg} & negative&  \textsc{rel} & relative clause marker\\
 \textsc{obj} & object & \textsc{subj} & subject\\
 \textsc{perf} & perfect\\
%\lspbottomrule
\end{tabular}

\printbibliography[heading=subbibliography,notkeyword=this]


\label{lastpage:Dryer}
\end{document}
